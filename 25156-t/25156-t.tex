% %%%%%%%%%%%%%%%%%%%%%%%%%%%%%%%%%%%%%%%%%%%%%%%%%%%%%%%%%%%%%%%%%%%%%%% %
%                                                                         %
% The Project Gutenberg EBook of An Introduction to Nonassociative Algebras, by 
% R. D. Schafer                                                           %
%                                                                         %
% This eBook is for the use of anyone anywhere at no cost and with        %
% almost no restrictions whatsoever.  You may copy it, give it away or    %
% re-use it under the terms of the Project Gutenberg License included     %
% with this eBook or online at www.gutenberg.org                          %
%                                                                         %
%                                                                         %
% Title: An Introduction to Nonassociative Algebras                       %
%                                                                         %
% Author: R. D. Schafer                                                   %
%                                                                         %
% Release Date: April 24, 2008 [EBook #25156]                             %
%                                                                         %
% Language: English                                                       %
%                                                                         %
% Character set encoding: ASCII                                           %
%                                                                         %
% *** START OF THIS PROJECT GUTENBERG EBOOK NONASSOCIATIVE ALGEBRAS ***   %
%                                                                         %
% %%%%%%%%%%%%%%%%%%%%%%%%%%%%%%%%%%%%%%%%%%%%%%%%%%%%%%%%%%%%%%%%%%%%%%% %

\def\ebook{25156}
\newtoks\PGheader
{\catcode`\#11\relax\catcode`\L\active\obeylines\obeyspaces%
\global\PGheader={%
The Project Gutenberg EBook of An Introduction to Nonassociative Algebras, by 
R. D. Schafer

This eBook is for the use of anyone anywhere at no cost and with
almost no restrictions whatsoever.  You may copy it, give it away or
re-use it under the terms of the Project Gutenberg License included
with this eBook or online at www.gutenberg.org


Title: An Introduction to Nonassociative Algebras

Author: R. D. Schafer

Release Date: April 24, 2008 [EBook #25156]

Language: English

Character set encoding: ASCII

*** START OF THIS PROJECT GUTENBERG EBOOK NONASSOCIATIVE ALGEBRAS ***
}}
\AtBeginDocument{\CreditsLine{%
Produced by David Starner, David Wilson, Suzanne Lybarger
and the Online Distributed Proofreading Team at
http://www.pgdp.net
}}

%%%%%%%%%%%%%%%%%%%%%%%%%%%%%%%%%%%%%%%%%%%%%%%%%%%%%%%%%%%%%%%%%%%%%%%%%%%
%%                                                                       %%
%% Packages and substitutions:                                           %%
%%                                                                       %%
%% memoir:   Advanced book class. Required.                              %%
%% memhfixc: Part of memoir; needed to work with hyperref. Required.     %%
%% amsmath:  AMS mathematics enhancements. Required.                     %%
%% amssymb:  extra AMS mathematics symbols. Required.                    %%
%% amsthm:   configurable theorem environments. Required.                %%
%% hyperref: Hypertext embellishments for pdf output. Required.          %%
%%           Driver option needs to be set explicitly.                   %%
%% indentfirst: Standard package to indent first line following chapter/ %%
%%              section headings. Recommended.                           %%
%%                                                                       %%
%%                                                                       %%
%% Producer's Comments: A fairly straightforward text, with modern       %%
%%                      notation and very few unusual layouts, other     %%
%%                      than indented equation tags and fixed tabstops   %%
%%                                                                       %%
%% Things to Check:                                                      %%
%%                                                                       %%
%% hyperref driver option matches workflow: OK                           %%
%% color driver option matches workflow (color package is called         %%
%%    by hyperref, so may rely on color.cfg): OK                         %%
%% Spellcheck: OK                                                        %%
%% Smoothreading pool: No                                                %%
%% LaCheck: OK                                                           %%
%% Lprep/gutcheck: OK                                                    %%
%% PDF pages:  81                                                        %%
%% PDF page size: 499 x 709pt (b5)                                       %%
%% PDF bookmarks: created but closed by default                          %%
%% PDF document info: filled in                                          %%
%% 1 overfull hbox due to lengthy inline formula (unavoidable)           %%
%%                                                                       %%
%% pdflatex x3                                                           %%
%%                                                                       %%
%% Compile History:                                                      %%
%%                                                                       %%
%% Apr 07: dcwilson.                                                     %%
%%         Compiled with pdfLaTeX THREE times.                           %%
%%         MiKTeX 2.7, Windows XP Pro                                    %%
%%                                                                       %%
%%                                                                       %%
%%%%%%%%%%%%%%%%%%%%%%%%%%%%%%%%%%%%%%%%%%%%%%%%%%%%%%%%%%%%%%%%%%%%%%%%%%%

\listfiles

\makeatletter

\documentclass[b5paper,12pt,twoside,openany,onecolumn]{memoir}[2005/09/25]

\setlrmarginsandblock{2.3cm}{2.6cm}{*}
\setulmarginsandblock{3.1cm}{2.2cm}{*}
\setlength{\headsep}{1cm}
\setlength{\footskip}{0.6cm}
\fixthelayout
\typeoutlayout
%
% font stuff
% this is the section most likely to require modification
%
% Courier, for the PG licence stuff
\DeclareRobustCommand\ttfamily % Courier, for the PG licence stuff
        {\not@math@alphabet\ttfamily\mathtt
         \fontfamily{pcr}\fontencoding{T1}\selectfont}

\setlength{\overfullrule}{\z@} % hide black boxes for PPV version!

% mathematics
\usepackage[leqno]{amsmath}[2000/07/18]
\usepackage[psamsfonts]{amssymb}[2002/01/22]
\usepackage{amsthm}[2004/08/06]

% The original uses uppercase; we use bold instead
\newtheoremstyle{schafer}% name
  {}%      Space above, empty = `usual value'
  {}%      Space below
  {\upshape}% Body font
  {\parindent}%         Indent amount (empty = no indent, \parindent = para indent)
  {\bfseries}% Thm head font
  {.}%        Punctuation after thm head
  {.5em}%     Space after thm head: " " = normal interword space;
        %       \newline = linebreak
  {#1\if!#3!\else\ \DPanchor{#1:#3}\fi\thmnote{#3}}% Thm head spec and destination for hyperlinking
\theoremstyle{schafer}
\newtheorem*{theorem}{Theorem}% we hardcode the number in the note
\newtheorem*{proposition}{Proposition}% we hardcode the number in the note
\newtheorem*{lemma}{Lemma}% we hardcode the number in the note
\newtheorem*{corollary}{Corollary}% we hardcode the number in the note
\newtheorem*{exercise}{Exercise}% we hardcode the number in the note
\makeatletter
\renewenvironment{proof}[1][\proofname]{\par
  \normalfont \topsep6\p@\@plus6\p@\relax
  \trivlist
  \item[\hskip\labelsep\indent
        \itshape % obviously this differs form the original typescript, but makes proofs easier to find
    #1\@addpunct{:}]\ignorespaces
}{%
  \endtrivlist
}
\DeclareMathOperator{\sgn}{sgn}
\DeclareMathOperator{\diag}{diag}
\DeclareMathOperator{\trace}{trace}
\newcommand{\type}[1]{\mathrm{#1}}
% to kill the fraktur, use \newcommand{\algebra}[1]{#1} instead
\newcommand{\algebra}[1]{\mathfrak{#1}}
% note \H and \L are already defined in such a way that redefining them is almost impossible
% we need the normal \S, and \SS is like \H and \L
\newcommand{\A}{\algebra{A}}
\newcommand{\B}{\algebra{B}}
\newcommand{\C}{\algebra{C}}
\newcommand{\D}{\algebra{D}}
\newcommand{\E}{\algebra{E}}
\newcommand{\G}{\algebra{G}}
\newcommand{\HH}{\algebra{H}}
\newcommand{\I}{\algebra{I}}
\newcommand{\J}{\algebra{J}}
\newcommand{\LL}{\algebra{L}}
\newcommand{\M}{\algebra{M}}
\newcommand{\N}{\algebra{N}}
\newcommand{\R}{\algebra{R}}
\newcommand{\Ss}{\algebra{S}}
\newcommand{\T}{\algebra{T}}
\newcommand{\Q}{\algebra{Q}}
\newcommand{\Z}{\algebra{Z}}
% we only want inline equations to break at an operator where
% we explicitly allow it (such as in the excessively long
% inline sequences of equalities used in the typescript)
\binoppenalty=\@M
\def\allowbreaks{\binoppenalty700\relax}
%
\medmuskip = 4mu plus 2mu minus 2mu % stop these shrinking to nothing
\def\dotsc{\allowbreak\ldots}
\let\dotm\cdot

% for (possibly aligned) equations with tags and/or commentary
% should be replaced with amsmath equivalent if I can find one!
\newskip\@Centering \@Centering=0pt plus 1000pt minus 1000pt
%   eqn number right-aligned at 2.25 normal parindent less 1 quad
% & (skip, default 2em) LHS
% & RHS
% & commentary text starting at [distance from right margin, default 10em] or right-aligned if wider
% \\
% \begin{myalign}[optional commentary tabstop, measured from right margin](optional equation indent)
\newenvironment{myalign}{\[\let\\\cr\let\tag\myaligntag
  \intertext@\@ifnextchar[{\my@lign}{\my@lign[10em]}}{\]\aftergroup\ignorespaces}
\def\my@lign[#1]{\@ifnextchar({\my@l@gn[#1]}{\my@l@gn[#1](2em)}}
\def\my@l@gn[#1](#2)#3\end{\displ@y \tabskip=\displaywidth
  \halign to\displaywidth{\kern-\displaywidth
  \hbox to2.25\absparindent{\hfil$\@lign##$\quad}\tabskip=\z@
  &\kern#2\hfil$\@lign\displaystyle{##}$\tabskip=\z@
  &$\@lign\displaystyle{{}##}$\hfill\tabskip=\@Centering
  &\setbox0=\hbox{\@lign##}\ifdim\wd0>#1\kern#1\llap{\box0}\else
    \hbox to#1{\box0\hfil}\fi\tabskip=\z@\crcr
  #3\crcr}\end}
\def\Intertext#1{\ifvmode\else\\\fi\noalign{\nobreak\noindent\strut#1}}
% for simple tags in myalign; includes hyper-destination
\def\myaligntag#1{\hbox{(#1)\DPanchor{eqn:\thechapter#1}}}
% to hyperlink to an equation: \tagref [chapter.](eqn)
\def\tagref{\@ifnextchar[{\t@gref}{\t@gref[\thechapter]}}
\def\t@gref[#1](#2){(\hyperlink{eqn:#1#2}{#2})}

% A quasi-verbatim environment for boilerplate, slightly less drastic than alltt
% Spaces, linebreaks, $ , % and # will appear as typed
% but unlike full verbatim, commands will still be interpreted and long lines will wrap
% (comments documenting the boilerplate text need to use | as the comment character)
% uses slightly non-standard obeylines and active space.
% The optional argument can be used to specify an explicit font size for the boilerplate.
% If no optional argument is provided, a fontsize will be computed to allow--as nearly as
% possible--73 fixed-width characters per \textwidth (the longest line in the PG license
% had 73 characters at one stage)
{\catcode`\^^M=\active % these lines must end with %
  \global\def\PGobeylines{\catcode`\^^M\active \def^^M{\null\par}}}%
{\obeyspaces%
\global\def\PGb@ilerplate[#1]{\def\PGb@ilerplateHook{#1}\catcode`\%11\relax%
\catcode`\$11\relax\catcode`\#11\relax\catcode`\|=14\relax%
\pretolerance=\@m\hyphenpenalty=5000%
\rightskip=\z@\@plus20em\relax%
\frenchspacing\ttfamily\PGb@ilerplateHook%
\def {\noindent\null\space}%
\parindent=\z@\PGobeylines\obeyspaces}}
\def\PGboilerplate{%
 \@ifnextchar[{\PGb@ilerplate}{\PGb@ilerplate[\PGAutoFit{73}]}}
\let\endPGboilerplate\empty
% \PGAutoFit adjusts the fontsize so a specified number of
% fixed-width characters will fit in the current \textwidth
\def\PGrem@pt#1.#2Q@!!@Q{#1}
\def\PGAutoFit#1{\setbox\z@=\hbox{m}\dimen@=\wd\z@\relax
  \multiply\dimen@#1\relax \dimen@i=\dimen@\relax
  \dimen@=\textwidth\relax
  \dimen@ii=\f@size pt  \advance\dimen@ii0.5pt
  \expandafter\multiply\expandafter\dimen@\expandafter\PGrem@pt\the\dimen@ii Q@!!@Q
  \expandafter\divide\expandafter\dimen@\expandafter\PGrem@pt\the\dimen@i Q@!!@Q
  \dimen@i=\dimen@\multiply\dimen@i12\divide\dimen@i10
  \fontsize{\strip@pt\dimen@}{\strip@pt\dimen@i}\ttfamily\selectfont}
{\catcode`\L\active
\gdef\PGlicencelink{\catcode`\L\active\letL\PGlinklicence}}
\def\PGlinklicence{\@ifnextchar i{\PG@lli}{L}}
\def\PG@lli#1{\@ifnextchar c{\PG@llii}{Li}}
\def\PG@llii#1{\@ifnextchar e{\PG@lliii}{Lic}}
\def\PG@lliii#1{\@ifnextchar n{\PG@lliv}{Lice}}
\def\PG@lliv#1{\@ifnextchar s{\PG@llv}{Licen}}
\def\PG@llv#1{\@ifnextchar e{\PG@llvi}{Licens}}
\def\PG@llvi#1{\hyperlink{PGlicence}{License}}

% half-title, title and copyright pages
\aliaspagestyle{title}{empty}
\setlength{\droptitle}{-6pc}
\pretitle{\begin{center}\LARGE\bfseries}
\renewcommand{\maketitlehooka}{\vglue\z@\@plus\@ne fill}
\posttitle{\par\end{center}}
\preauthor{\begin{center}\large}
\postauthor{\par\end{center}}
\def\affiliation#1{\renewcommand{\maketitlehookc}{\begin{center}\small
  \textsc{#1}\par\end{center}}}
\def\subtitle{\def\SubTitle}
\predate{\vfill\begin{center}\SubTitle\end{center}
  \vspace{\z@\@plus1.5fill}\begin{center}\Small\itshape}
\postdate{\par\end{center}\vspace{-1.5em}}
\let\transcribersnotes\@empty
\let\transcribersNotes\@empty
\newcommand{\transcribersnote}[1]{%
  \@ifnotempty{#1}{\g@addto@macro\transcribersnotes{#1\par}%
    \@xp\@ifempty\@xp{\transcribersNotes}%
      {\renewcommand{\transcribersNotes}{note}}
      {\renewcommand{\transcribersNotes}{notes}}}}
\newcommand{\CreditsLine}[1]{\newcommand{\thePr@ductionTeam}{#1}}

\def\makehalftitlepage{% the boilerplate header
  \begingroup
  \pagestyle{empty}
  \pagenumbering{Alph} % to ensure unique hyperref page anchors
  \begin{PGboilerplate}[\tiny] % 8pt for B5
    \PGlicencelink
    \the\PGheader
  \end{PGboilerplate}
  \cleartorecto
  \endgroup}

\def\makecopyrightpage{% production credits and transcriber's notes
  \begingroup\pagestyle{empty}
  \null\vfil
  \begin{center}
    \thePr@ductionTeam
  \end{center}
  \vfil\vfil
  \vbox{\Small\hsize=.75\textwidth\parindent=\z@\parskip=.5em
  \textit{Transcriber's \transcribersNotes}\par\medskip\raggedright
  \transcribersnotes\par}
  \newpage\endgroup}

% chapters and sections
\usepackage{indentfirst}[1995/11/23]
\makechapterstyle{schafer}{%
  \setlength{\beforechapskip}{\z@}
  \setlength{\midchapskip}{\z@}
  \renewcommand{\chaptitlefont}{\normalfont\Large\scshape}
  \renewcommand{\chapnumfont}{\centering\normalfont\Large\scshape}
  \let\printchaptername\empty\let\chapternamenum\empty
  \let\afterchapternum\quad
  \renewcommand{\thechapter}{\@Roman\c@chapter.}
  \renewcommand{\printchaptertitle}[1]{\chaptitlefont ##1\ignorespaces}
  \setlength{\afterchapskip}{0.15pc}
  \renewcommand{\afterchaptertitle}{\nobreak\vskip\afterchapskip\null}
  }
\chapterstyle{schafer}
\setlength{\parindent}{1.8em}
\setlength{\leftmargini}{1.8em}
% for cross-hyperreferences to chapters
\def\chaplink#1{\hyperlink{chapter.#1}{\@Roman{#1}}}
\AtBeginDocument{\let\PGph@ntomsecti@n\phantomsection
  \def\phantomsection{\leavevmode\PGph@ntomsecti@n\ignorespaces}}

% headers and footers
\copypagestyle{frontstuff}{headings}
\makeevenhead{frontstuff}{\normalfont\SMALL\thepage}{\normalfont
    \SMALL\MakeUppercase{\leftmark}}{}
\makeoddhead{frontstuff}{}{\normalfont\SMALL\MakeUppercase{\rightmark}}%
    {\normalfont\SMALL\thepage}
\makepsmarks{frontstuff}{%
      \let\@mkboth\markboth
      \def\chaptermark##1{%
        \markboth{\MakeUppercase{##1}}{\MakeUppercase{##1}}}%
    }
\pagestyle{frontstuff}
\copypagestyle{chapter}{plain}
\makeevenfoot{chapter}{}{\normalfont\SMALL\thepage}{}
\makeoddfoot{chapter}{}{\normalfont\SMALL\thepage}{}
\copypagestyle{mainstuff}{headings}
\makepsmarks{mainstuff}{%
      \let\@mkboth\markboth
      \def\chaptermark##1{%
        \markboth{\MakeUppercase{##1}}{\MakeUppercase{##1}}}%
    }
\makeevenhead{mainstuff}{\normalfont\SMALL\thePAGE}{\normalfont
    \SMALL\MakeUppercase{\leftmark}}{}
\makeoddhead{mainstuff}{}{\normalfont
    \SMALL\MakeUppercase{\rightmark}}{\normalfont\SMALL\thePAGE}

\copypagestyle{licence}{headings}
\makeevenhead
  {licence}{\normalfont\SMALL\thepage}{\normalfont\SMALL LICENSING.}{}
\makeoddhead
  {licence}{}{\normalfont\SMALL LICENSING.}{\normalfont\SMALL\thepage}

% to deal with the scanned page breaks
% add a "draft" option to the documentclass invocation
% to see the scan numbers
\ifdraftdoc
\def\PG#1 #2.png#3
{\marginpar{\noindent\null\hfill\Small #2.png}}
\def\PGx#1 #2.png#3
{}
\else
\def\PG#1 #2.png#3
{}
\let\PGx\PG
\fi

% PDF stuff: links, document info, etc
% Originally coded with a PostScript workflow in mind;
% if default driver given in hyperref.cfg is not suitable,
% add appropriate explicit option to hyperref call
\usepackage[final,colorlinks]{hyperref}[2003/11/30]
% we check if the driver is the useless (for pdf) "hypertex",
% and if so we force dvips instead and issue a warning
\def\@tempa{hypertex}
\ifx\@tempa\Hy@defaultdriver
  \GenericWarning{*** }{***\MessageBreak
   Inappropriate driver for hyperref specified: assuming dvips.\MessageBreak
   You should amend the source code if using another driver.\MessageBreak
   \expandafter\@gobble\@gobble}
  \Hy@SetCatcodes\input{hdvips.def}\Hy@RestoreCatcodes
\fi
\usepackage{memhfixc}[2004/05/13]
\providecommand\ebook{1wxyz}% temporary, will be overridden by WWer
\hypersetup{pdftitle=The Project Gutenberg eBook \#\ebook: An Introduction to Nonassociative Algebras,
  pdfsubject=NSF Institute in Algebra,
  pdfauthor=Richard D. Schafer,
  pdfkeywords={David Starner, Suzanne Lybarger, David Wilson,
               Project Gutenberg Online Distributed Proofreading Team},
  pdfstartview=Fit,
  pdfstartpage=1,
  pdfpagemode=UseNone,
  pdfdisplaydoctitle,
  bookmarksopen,
  bookmarksopenlevel=1,
  linktocpage=false,
  pdfpagescrop=0 0 499 709, b5paper, % b5 176x250mm
  pdfpagelayout=TwoPageRight, % this is Acrobat 6's "Facing"
  plainpages=false, linkcolor=\ifdraftdoc blue\else black\fi,
  menucolor=\ifdraftdoc blue\else black\fi,
  citecolor=\ifdraftdoc blue\else black\fi,
  urlcolor=\ifdraftdoc magenta\else black\fi}

% slight modification of hyperref's command,
% for adding explicit bookmarks and destinations
\newcommand\DPpdfbookmark[3][0]{\rlap{\hyper@anchorstart{#3}\hyper@anchorend
     \Hy@writebookmark{}{#2}{#3}{#1}{toc}}}
\newcommand\DPanchor[1]{\rlap{\hyper@anchorstart{#1}\hyper@anchorend}}
\newcommand\DPlabel[1]{\DPanchor{#1}\label{#1}}
% hyperref seems to use the most recent section anchor instead of page
% anchors in a \pageref, so we have to fix that; using AtBeginDocument
% so hyperref doesn't clobber it
\AtBeginDocument{\def\T@pageref#1{\@safe@activestrue\expandafter\@setpgref
  \csname r@#1\endcsname\@secondoffive{#1}\@safe@activesfalse}
  \def\@setpgref#1#2#3{\ifx #1\relax\protect\G@refundefinedtrue\nfss@text
    {\reset@font\bfseries ??}\@latex@warning{Reference `#3' on page \thepage
     \space undefined}\else\expandafter\Hy@setpgref@link
     #1\@empty\@empty\@nil{#2}\fi}
  \def\Hy@setpgref@link#1#2#3#4#5#6\@nil#7{\begingroup\toks0={\hyper@@link
   {#5}{page.#2}}\toks1=\expandafter{#7{#1}{#2}{#3}{#4}{#5}}\edef\x
   {\endgroup\the\toks0{\the\toks1}}\x}}

% for itemized lists without hanging indent
\def\Itemize{\leftmargini\z@\list{}{\labelwidth\parindent
    \itemindent2\parindent\advance\itemindent\labelsep\parsep\z@\itemsep6\p@\@plus3\p@
    \def\makelabel##1{\rlap{\LeftBracket##1\RightBracket}\hss}%
    \let\LeftBracket(\let\RightBracket)}}%
\let\endItemize\endlist
\def\IItemize{\leftmargini\z@\list{}{\labelwidth1.5\parindent\advance\labelwidth-\labelsep
    \itemindent2.5\parindent
    \def\makelabel##1{\rlap{\LeftBracket##1\RightBracket}\hss}%
    \let\LeftBracket\empty\let\RightBracket\empty}}%
\let\endIItemize\endlist

% for the bibliography
\def\bysame{\leavevmode\hbox to3em{\hrulefill}\thinspace}
\nobibintoc
\renewcommand{\prebibhook}{\DPpdfbookmark[0]{Bibliography}{Biblio*1}\markboth{\bibname}{\bibname}}
\renewcommand{\bibsection}{\bigskip\small\prebibhook}
\setbiblabel{#1.\hfill}
\setlength\bibindent{1.8em}
\def\@openbib@code{\advance\leftmargin\bibindent
      \itemindent -\bibindent
      \listparindent \itemindent}
% to change from commas to semicolons in multiple citations
\def\@citex[#1]#2{\leavevmode
  \let\@citea\@empty
  \@cite{\@for\@citeb:=#2\do
    {\@citea\def\@citea{;\penalty\@m\ }%
     \edef\@citeb{\expandafter\@firstofone\@citeb\@empty}%
     \if@filesw\immediate\write\@auxout{\string\citation{\@citeb}}\fi
     \@ifundefined{b@\@citeb}{\hbox{\reset@font\bfseries ?}%
       \G@refundefinedtrue
       \@latex@warning
         {Citation `\@citeb' on page \thepage \space undefined}}%
       {\hbox{\csname b@\@citeb\endcsname}}}}{#1}}

% bits and pieces
\emergencystretch=12pt
\let\Small\footnotesize
\let\SMALL\scriptsize
\let\ThoughtBreak\bigskip

\makeatother

\begin{document}
\PG--File: 001.png---\******\*******\************\******\------------------

\iffalse

\begin{center}
AN INTRODUCTION TO

NONASSOCIATIVE ALGEBRAS

\vspace{1cm}

R. D. Schafer

Massachusetts Institute of Technology

\vspace{4cm}

An Advanced Subject-Matter Institute in Algebra  \\
%[** Typo?---------^. The next page doesn't have the hyphen.]
Sponsored by  \\
The National Science Foundation

\vspace{2cm}

Prepared for Multilithing by  \\
Ann Caskey


\vspace{2cm}

The Department of Mathematics  \\
Oklahoma State University  \\
Stillwater, Oklahoma  \\
1961
\end{center}
\fi

\makehalftitlepage
\frontmatter

\title{AN INTRODUCTION TO\\
NONASSOCIATIVE ALGEBRAS}

\author{R.\,D. Schafer}
\affiliation{Massachusetts Institute of Technology}
\subtitle{An Advanced Subject-Matter Institute in Algebra\\
Sponsored by\\
The National Science Foundation}

\date{Stillwater, Oklahoma, 1961}

\maketitle

\newpage

\transcribersnote{This e-text was created from scans of the multilithed
book published by the Department of Mathematics at Oklahoma State University
in 1961. The book was prepared for multilithing by Ann Caskey.}

\transcribersnote{The original was typed rather than typeset, which somewhat
limited the symbols available; to assist the reader we have here adopted the
convention of denoting algebras etc by fraktur symbols, as followed by the
author in his substantially expanded version of the work published under the
same title by Academic Press in 1966.}

\transcribersnote{\SMALL Minor corrections to punctuation and spelling and minor modifications to layout
are documented in the \LaTeX\ source.}

\makecopyrightpage
\cleartorecto



\PG--File: 002.png---\******\*******\************\******\------------------




\null\vfil\DPpdfbookmark[0]{Preface}{Preface*1}
These are notes for my lectures in July, 1961, at the Advanced Subject
Matter Institute in Algebra which was held at Oklahoma State University in
the summer of 1961.

Students at the Institute were provided with reprints of my paper,
\emph{Structure and representation of nonassociative algebras} (Bulletin of the
American Mathematical Society, vol.~61 (1955), pp.~469--484), together with
copies of a selective bibliography of more recent papers on non\-associative
algebras. These notes supplement \S\S3--5 of the 1955 Bulletin \hyperlink{cite.Ref64}{article},
bringing the statements there up to date and providing detailed proofs of
a selected group of theorems. The proofs illustrate a number of important
techniques used in the study of nonassociative algebras.

\bigskip
\begin{flushright}
\textsc{R.\,D.\ Schafer}
\end{flushright}

\bigskip
\begin{flushleft}
\small
Stillwater, Oklahoma  \\
July 26, 1961
\end{flushleft}

\mainmatter

\PGx--File: 003.png---\******\******\********\********\---------------------
\chapter{Introduction} % I.

By common consent a ring $\R$ is understood to be an additive abelian
group in which a multiplication is defined, satisfying
\begin{myalign}
\tag{1} &&(xy)z = x(yz)           &for all $x,y,z$ in $\R$
\Intertext{and}
\tag{2} &&(x+y)z = xz+yz,\qquad z(x+y) = zx+zy \\ % break not in original
  &&&for all $x,y,z$ in $\R$,
\end{myalign}
while an algebra $\A$ over a field $F$ is a ring which is a vector space over
$F$ with
\begin{myalign}
\tag{3} &&\alpha(xy) = (\alpha x)y = x(\alpha y)  &for all $\alpha$ in $F$, $x,y$ in $\A$,\\
\end{myalign}
so that the multiplication in $\A$ is bilinear. Throughout these notes,
however, the associative law \tagref(1) will fail to hold in many of the algebraic
systems encountered. For this reason we shall use the terms ``ring'' and
``algebra'' for more general systems than customary.

We define a \emph{ring} $\R$ to be an additive abelian group with a second
law of composition, multiplication, which satisfies the distributive
laws \tagref(2). We define an \emph{algebra} $\A$ over a field $F$ to be a vector space
over $F$ with a bilinear multiplication (that is, a multiplication satisfying
\tagref(2) and \tagref(3)). We shall use the name \emph{associative ring} (or \emph{associative
algebra}) for a ring (or algebra) in which the associative law \tagref(1) holds.

In the general literature an algebra (in our sense) is commonly
referred to as a \emph{nonassociative algebra} in order to emphasize that \tagref(1)
is not being assumed. Use of this term does not carry the connotation
that \tagref(1) fails to hold, but only that \tagref(1) is not assumed to hold. If \tagref(1)
is actually not satisfied in an algebra (or ring), we say that the algebra
(or ring) is \emph{not associative}, rather than nonassociative.

As we shall see in \chaplink{2}, a number of basic concepts which are familiar
from the study of associative algebras do not involve associativity in any
\PG--File: 004.png---\************\******\********\******\-----------------
way, and so may fruitfully be employed in the study of nonassociative
algebras. For example, we say that two algebras $\A$ and $\A'$ over $F$ are
\emph{isomorphic} in case there is a vector space isomorphism $x \leftrightarrow x'$ between
them with
\begin{myalign}
\tag{4} && (xy)' = x'y'&for all $x, y$ in $\A$.
\end{myalign}

Although we shall prove some theorems concerning rings and infinite-dimensional
algebras, we shall for the most part be concerned with finite-dimensional
algebras. If $\A$ is an algebra of dimension $n$ over $F$, let
$u_1, \dotsc, u_n$ be a basis for $\A$ over $F$. Then the bilinear multiplication in $\A$
is completely determined by the $n^3$ \emph{multiplication constants} $\gamma_{ijk}$ which
appear in the products
\begin{myalign}
\tag{5} &&
  u_i u_j = \sum_{k=1}^n \gamma_{ijk} u_k,   &$\gamma_{ijk}$ in $F$.
\end{myalign}
We shall call the $n^2$ equations \tagref(5) a \emph{multiplication table}, and shall
sometimes have occasion to arrange them in the familiar form of such a
table:
\[
\begin{array}{c| ccccc}
         & u_1    & \dots  & u_j     & \dots  & u_n \rule{0pt}{3ex}  \\
\hline
  u_1    &        &        & \vdots  \\
  \vdots &        &        & \vdots  \\
  u_i
&\multicolumn{5}{c}{\dots\ \ \sum \gamma_{ijk} u_k\ \ \dots}  \\
  \vdots &        &        & \vdots  \\
  u_n    &        &        & \vdots
\end{array}
\]

\PG--File: 005.png---\*******\******\********\******\----------------------
The multiplication table for a one-dimensional algebra $\A$ over $F$ is
given by $u_1^2 =\gamma u_1 (\gamma = \gamma_{111})$. There are two cases: $\gamma=0$ (from which
it follows that every product $xy$ in $\A$ is $0$, so that $\A$ is called a \emph{zero
algebra}), and $\gamma \ne 0$. In the latter case the element $e = \gamma^{-1}u_1$ serves
as a basis for $\A$ over $F$, and in the new multiplication table we have
$e^2 = e$. Then $\alpha \leftrightarrow \alpha e$ is an isomorphism between $F$ and this one-dimensional
algebra $\A$. We have seen incidentally that any one-dimensional algebra is
associative. There is considerably more variety, however, among the
algebras which can be encountered even for such a low dimension as two.

Other than associative algebras the best-known examples of algebras
are the Lie algebras which arise in the study of Lie groups. A \emph{Lie algebra}
$\LL$ over $F$ is an algebra over $F$ in which the multiplication is \emph{anticommutative},
that is,
\begin{myalign}
\tag{6}
&& x^2 = 0  & (implying $xy = -yx$),
\intertext{and the \emph{Jacobi identity}}
\tag{7}
&& (xy)z + (yz)x + (zx)y = 0   &for all $x, y, z$ in $\LL$
\intertext{is satisfied.  If $\A$ is any associative algebra over $F$, then the \emph{commutator}}
\tag{8}
&& [x,y] = xy - yx
\Intertext{satisfies }
\tag{6$'$}
&& [x,x] = 0
\Intertext{and }
\tag{7$'$}
&& \bigl[ [x,y],z \bigr]
+ \bigl[ [y,z],x \bigr]
+ \bigl[ [z,x],y \bigr] = 0.
\end{myalign}
Thus the algebra $\A^-$ obtained by defining a new multiplication \tagref(8) in the
same vector space as $\A$ is a Lie algebra over $F$. Also any subspace of $\A$
which is closed under commutation \tagref(8) gives a subalgebra of $\A^-$, hence a
Lie algebra over $F$. For example, if $\A$ is the associative algebra of
all $n \times n$ matrices, then the set $\LL$ of all skew-symmetric matrices in $\A$
is a Lie algebra of dimension $\frac{1}{2}n(n-1)$. The Birkhoff-Witt theorem states
\PG--File: 006.png---\********\******\********\********\-------------------
that any Lie algebra $\LL$ is isomorphic to a subalgebra of an (infinite-dimensional)
algebra $\A^-$ where $\A$ is associative. In the general literature
the notation $[x,y]$ (without regard to \tagref(8)) is frequently used, instead of
$xy$, to denote the product in an arbitrary Lie algebra.

In these notes we shall not make any systematic study of Lie algebras.
A number of such accounts exist (principally for characteristic $0$, where most
of the known results lie). Instead we shall be concerned upon occasion with
relationships between Lie algebras and other non\-associative algebras which
arise through such mechanisms as the \emph{derivation algebra}. Let $\A$ be any
algebra over $F$. By a \emph{derivation} of $\A$ is meant a linear operator $D$ on $\A$
satisfying
\begin{myalign}
\tag{9}  &&(xy)D = (xD)y + x(yD) &for all $x,y$ in $\A$.
\end{myalign}
The set $\D(\A)$ of all derivations of $\A$ is a subspace of the associative
algebra $\E$ of all linear operators on $\A$. Since the commutator $[D, D']$
of two derivations $D$, $D'$ is a derivation of $\A$, $\D(\A)$ is a subalgebra of
$\E^-$; that is, $\D(\A)$ is a Lie algebra, called the \emph{derivation algebra} of $\A$.

Just as one can introduce the commutator \tagref(8) as a new product to
obtain a Lie algebra $\A^-$ from an associative algebra $\A$, so one can
introduce a symmetrized product
\begin{myalign}
\tag{10} && x * y = xy + yx
\end{myalign}
in an associative algebra $\A$ to obtain a new algebra over $F$ where the
vector space operations coincide with those in $\A$ but where multiplication
is defined by the commutative product $x * y$ in \tagref(10). If one is content
to restrict attention to fields $F$ of characteristic not two (as we shall
be in many places in these notes) there is a certain advantage in writing
\begin{myalign}
\tag{10$'$} &&   x\dotm y = \tfrac12 (xy + yx)
\end{myalign}
to obtain an algebra $\A^+$ from an associative algebra $\A$ by defining products
by \tagref(10$'$) in the same vector space as $\A$. For $\A^+$ is isomorphic under the
\PG--File: 007.png---\*******\************\********\********\--------------
mapping $a \to \frac12 a$ to the algebra in which products are defined by \tagref(10).
At the same time powers of any element $x$ in $\A^+$ coincide with those in $\A$:
clearly $x \dotm x = x^2$, whence it is easy to see by induction on $n$ that
$x \dotm x \dotm \dots \dotm x \text{ ($n$ factors)} = (x \dotm \dots \dotm x) \dotm (x \dotm \dots \dotm x) =
x^i \dotm x^{n-i} = \frac12 (x^i x^{n-i} + x^{n-i} x^i) = x^n$.

If $\A$ is associative, then the multiplication in $\A^+$ is not only
commutative but also satisfies the identity
\begin{myalign}
\tag{11}  &&(x \dotm y)\dotm(x \dotm x) = x\dotm\left[y\dotm(x\dotm x)\right] &for all $x, y$ in $\A^+$.
\intertext{A (commutative) \emph{Jordan algebra} $\J$ is an algebra over a field $F$ in which
products are \emph{commutative}:}
\tag{12}  &&xy = yx               &for all $x, y$ in $\J$,
\intertext{and satisfy the \emph{Jordan identity}}
\tag{11$'$} &&(xy)x^2 = x(yx^2)    &for all $x, y$ in $\J$.
\end{myalign}
Thus, if $\A$ is associative, then $\A^+$ is a Jordan algebra. So is any
subalgebra of $\A^+$, that is, any subspace of $\A$ which is closed under the
symmetrized product \tagref(10$'$) and in which \tagref(10$'$) is used as a new multiplication
(for example, the set of all $n \times n$ symmetric matrices). An algebra $\J$ over
$F$ is called a \emph{special Jordan algebra} in case $\J$ is isomorphic to a subalgebra
of $\A^+$ for some associative $\A$. We shall see that not all Jordan algebras
are special.

Jordan algebras were introduced in the early 1930's by a physicist,
P.~Jordan, in an attempt to generalize the formalism of quantum mechanics.
Little appears to have resulted in this direction, but unanticipated
relationships between these algebras and Lie groups and the foundations of
geometry have been discovered.

The study of Jordan algebras which are not special depends upon
knowledge of a class of algebras which are more general, but in a certain
sense only slightly more general, than associative algebras. These are
\PG--File: 008.png---\*******\*******\********\********\-------------------
the \emph{alternative} algebras $\A$ defined by the identities
\begin{myalign}
\tag{13}  &x^2y &= x(xy)          &for all $x,y$ in $\A$
\Intertext{and}
\tag{14}  &yx^2 &= (yx)x          &for all $x,y$ in $\A$,
\end{myalign}
known respectively as the \emph{left} and \emph{right alternative laws}.  Clearly any
associative algebra is alternative. The class of $8$-dimensional \emph{Cayley
algebras} (or \emph{Cayley-Dickson algebras}, the prototype having been discovered
in 1845 by Cayley and later generalized by Dickson) is, as we shall see,
an important class of alternative algebras which are not associative.

To date these are the algebras (Lie, Jordan and alternative) about
which most is known. Numerous generalizations have recently been made,
usually by studying classes of algebras defined by weaker identities.
We shall see in \chaplink{2} some things which can be proved about completely
arbitrary algebras.
\PG--File: 009.png---\*******\************\********\********\--------------




\chapter{Arbitrary Nonassociative Algebras} % II.

Let $\A$ be an algebra over a field $F$. (The reader may make the
appropriate modifications for a ring $\R$.) The definitions of the terms
\emph{subalgebra}, \emph{left ideal}, \emph{right ideal}, (two-sided) \emph{ideal} $\I$, \emph{homomorphism},
\emph{kernel} of a homomorphism, \emph{residue class algebra} $\A/\I$ (\emph{difference algebra}
$\A-\I$), \emph{anti-isomorphism}, which are familiar from a study of associative
algebras, do not involve associativity of multiplication and are thus
immediately applicable to algebras in general. So is the notation $\B\C$
for the subspace of $\A$ spanned by all products $bc$ with $b$ in $\B$, $c$ in $\C$
($\B$, $\C$ being arbitrary nonempty subsets of $\A$); here we must of course
distinguish between $(\A\B)\C$ and $\A(\B\C)$, etc.

We have the \emph{fundamental theorem of homomorphism for algebras}:
If $\I$ is an ideal of $\A$, then $\A/\I$ is a homomorphic image of $\A$ under the
natural homomorphism
\begin{myalign}
\tag{1}  &&a \to  \overline{a} = a + \I,   &$a$ in $\A$, $a + \I$ in $\A/\I$.
\intertext{Conversely, if $\A'$ is a homomorphic image of $\A$ (under the homomorphism}
\tag{2}  &&a \to a',   &$a$ in $\A$, $a'$ in $\A'$),
\end{myalign}
then $\A'$ is isomorphic to $\A/\I$ where $\I$ is the kernel of the homomorphism.

If $\Ss'$ is a subalgebra (or ideal) of a homomorphic image $\A'$ of $\A$, then
the \emph{complete inverse image} of $\Ss'$ under the homomorphism \tagref(2)---that is, the
set $\Ss = \{ s \in \A \mid s' \in \Ss'\}$---is a subalgebra (or ideal) of $\A$ which contains
the kernel $\I$ of \tagref(2). If a class of algebras is defined by identities (as,
for example, Lie, Jordan or alternative algebras), then any subalgebra or
any homomorphic image belongs to the same class.

We have the customary isomorphism theorems:

\begin{Itemize}
\item[i] If $\I_1$ and $\I_2$ are ideals of $\A$ such that $\I_1$ contains $\I_2$, then
\PG--File: 010.png---\************\************\********\********\---------
$(\A/\I_2)/(\I_1/\I_2)$ and $\A/\I_1$ are isomorphic.

\item[ii] If $\I$ is an ideal of $\A$ and $\Ss$ is a subalgebra of $\A$, then $\I\cap\Ss$
is an ideal of $\Ss$, and $(\I+\Ss)/\I$ and $\Ss/(\I\cap\Ss)$ are isomorphic.
\end{Itemize}

Suppose that $\B$ and $\C$ are ideals of an algebra $\A$, and that as a vector
space $\A$ is the direct sum of $\B$ and $\C$ ($\A = \B + \C$, $\B\cap\C = 0$). Then $\A$ is
called the \emph{direct sum} $\A = \B\oplus\C$ of $\B$ and $\C$ as algebras. The vector space
properties insure that in a direct sum $\A = \B\oplus\C$ the components $b$, $c$ of
$a = b + c$ ($b$ in $\B$, $c$ in $\C$) are uniquely determined, and that addition and
multiplication by scalars are performed componentwise. It is the assumption
that $\B$ and $\C$ are ideals in $\A = \B\oplus\C$ that gives componentwise multiplication
as well:
\begin{myalign}(0em)
\tag{3}  &&(b_1 + c_1)(b_2 + c_2) = b_1b_2 + c_1c_2, &$b_i$ in $\B$, $c_i$ in $\C$.
\end{myalign}
For $b_1c_2$ is in both $\B$ and $\C$, hence in $\B\cap\C = 0$. Similarly $c_1b_2 = 0$,
so \tagref(3) holds, (Although $\oplus$ is commonly used to denote vector space direct
sum, it has been reserved in these notes for direct sum of ideals; where
appropriate the notation $\perp$ has been used for orthogonal direct sum relative
to a symmetric bilinear form.)

Given any two algebras $\B$, $\C$ over a field $F$, one can construct an
algebra $\A$ over $F$ such that $\A$ is the direct sum $\A = \B'\oplus\C'$ of ideals $\B'$,
$\C'$ which are isomorphic respectively to $\B$, $\C$. The construction of $\A$ is
familiar: the elements of $\A$ are the ordered pairs $(b, c)$ with $b$ in $\B$,
$c$ in $\C$; addition, multiplication by scalars, and multiplication are
defined componentwise:
\begin{myalign} % tag (4) omitted from typescript
&(b_1, c_1) + (b_2, c_2) &= (b_1 + b_2, c_1 + c_2),\\
\tag{4}&\alpha(b, c) &= (\alpha b, \alpha c),\\
&(b_1, c_1)(b_2, c_2) &= (b_1c_1, b_2c_2).
\end{myalign}
Then $\A$ is an algebra over $F$, the sets $\B'$ of all pairs $(b, 0)$ with $b$ in $\B$
and $\C'$ of all pairs $(0, c)$ with $c$ in $\C$ are ideals of $\A$ isomorphic respectively
\PG--File: 011.png---\************\************\******\********\-----------
to $\B$ and $\C$, and $\A = \B'\oplus\C'$. By the customary identification of $\B$ with
$\B'$, $\C$ with $\C'$, we can then write $\A = \B\oplus\C$, the direct sum of $\B$ and $\C$
as algebras.

As in the case of vector spaces, the notion of direct sum extends to
an arbitrary (indexed) set of summands. In these notes we shall have
occasion to use only finite direct sums $\A = \B_1\oplus\B_2\oplus\dotsb\oplus\B_t$.
Here $\A$ is the direct sum of the vector spaces $\B_i$, and multiplication in
$\A$ is given by
\begin{myalign}(-0.25em)
\tag{5}&&
  (b_1 + b_2 + \dots + b_t) (c_1 + c_2 + \dots + c_t)
= b_1c_1 + b_2c_2 + \dots + b_tc_t
\end{myalign}
for $b_i$, $c_i$ in $\B_i$. The $\B_i$ are ideals of $\A$. Note that (in the case of a
vector space direct sum) the latter statement is equivalent to the fact that
the $\B_i$ are subalgebras of $\A$ such that
\begin{myalign}
\tag{6}&&
  \B_i\B_j = 0 & for $i \ne j$.
\end{myalign}

An element $e$ (or $f$) in an algebra $\A$ over $F$ is called a \emph{left} (or \emph{right})
\emph{identity} (sometimes \emph{unity element}) in case $ea = a$ (or $af = a$) for all $a$ in $\A$.
If $\A$ contains both a left identity $e$ and a right identity $f$, then $e = f$
($= ef$) is a (two-sided) \emph{identity} $1$. If $\A$ does not contain an identity
element $1$, there is a standard construction for obtaining an algebra $\A_1$
which does contain $1$, such that $\A_1$ contains (an isomorphic copy of) $\A$ as
an ideal, and such that $\A_1/\A$ has dimension $1$ over $F$. We take $\A_1$ to be the
set of all ordered pairs $(\alpha, a)$ with $\alpha$ in $F$, $a$ in $\A$; addition and
multiplication by scalars are defined componentwise; multiplication is
defined by
\begin{myalign}[0em]
\tag{7}
  &&(\alpha, a)(\beta, b)
= (\alpha\beta,\ \beta a + \alpha b + ab), &$\alpha, \beta$ in $F$, $a, b$ in $\A$.\\
\end{myalign}
Then $\A_1$ is an algebra over $F$ with identity element $1 = (1, 0)$. The set
$\A'$ of all pairs $(0, a)$ in $\A_1$ with $a$ in $\A$ is an ideal of $\A_1$ which is
isomorphic to $\A$. As a vector space $\A_1$ is the direct sum of $\A'$ and the
$1$-dimensional space $F1 = \{\alpha 1 \mid \alpha \text{ in } F\}$. Identifying $\A'$ with its isomorphic
\PG--File: 012.png---\*******\*****\********\********\---------------------
image $\A$, we can write every element of $\A_1$ uniquely in the form $\alpha 1 + a$
with $\alpha$ in $F$, $a$ in $\A$, in which case the multiplication \tagref(7) becomes
\begin{myalign}
\tag{7$'$}&& (\alpha 1 + a)(\beta 1 + b) = (\alpha\beta)1 + (\beta a + \alpha b + ab).
\end{myalign}
We say that we have \emph{adjoined a unity element} to $\A$ to obtain $\A_1$. (If $\A$ is
associative, this familiar construction yields an associative algebra $\A_1$
with $1$. A similar statement is readily verifiable for (commutative)
Jordan algebras and for alternative algebras. It is of course not true
for Lie algebras, since $1^2 = 1 \ne 0$.)

Let $\B$ and $\A$ be algebras over a field $F$. The \emph{Kronecker product}
$\B\otimes_F\A$ (written $\B\otimes\A$ if there is no ambiguity) is the tensor product
$\B\otimes_F\A$ of the vector spaces $\B$, $\A$ (so that all elements are sums
$\sum b \otimes a$, $b$ in $\B$, $a$ in $\A$, multiplication being defined by distributivity and
\begin{myalign}[0em]
\tag{8}  &&(b_1 \otimes a_1)(b_2 \otimes a_2) = (b_1 b_2) \otimes (a_1 a_2), &$b_i$ in $\B$, $a_i$ in $\A$.\\
\end{myalign}
If $\B$ contains $1$, then the set of all $1 \otimes a$ in $\B\otimes\A$ is a subalgebra of
$\B\otimes\A$ which is isomorphic to $\A$, and which we can identify with $\A$ (similarly,
if $\A$ contains $1$, then $\B\otimes\A$ contains $\B$ as a subalgebra). If $\B$ and $\A$ are
finite-dimensional over $F$, then $\dim (\B\otimes\A) = (\dim\B)(\dim\A)$.

We shall on numerous occasions be concerned with the case where $\B$
is taken to be a field (an arbitrary extension $K$ of $F$). Then $K$ does contain
$1$, so $\A_K = K \otimes_F\A$ contains $\A$ (in the sense of isomorphism) as a subalgebra
over $F$. Moreover, $\A_K$ is readily seen to be an algebra over $K$, which is
called the \emph{scalar extension} of $\A$ to an algebra over $K$. The properties of
a tensor product insure that any basis for $\A$ over $F$ is a basis for $\A_K$
over $K$. In case $\A$ is finite-dimensional over $F$, this gives an easy
representation for the elements of $\A_K$. Let $u_1, \dotsc, u_n$ be any basis for
$\A$ over $F$.  Then the elements of $\A_K$ are the linear combinations
\begin{myalign}
\tag{9} &&\sum \alpha_i u_i\quad (=\sum \alpha_i \otimes u_i), &$\alpha_i$ in $K$,
\end{myalign}
where the coefficients $\alpha_i$ in \tagref(9) are uniquely determined. Addition and
\PG--File: 013.png---\*******\*****\********\********\---------------------
multiplication by scalars are performed componentwise. For multiplication
in $\A_K$ we use bilinearity and the multiplication table
\begin{myalign}
\tag{10} &&u_i u_j = \sum \gamma_{ijk}\; u_k, &$\gamma_{ijk}$ in $F$.
\end{myalign}
The elements of $\A$ are obtained by restricting the $\alpha_i$ in \tagref(9) to elements of
$F$.

For finite-dimensional $\A$, the scalar extension $\A_K$ ($K$ an arbitrary
extension of $F$) may be defined in a non-invariant way (without recourse
to tensor products) by use of a basis as above. Let $u_1, \dotsc, u_n$ be any
basis for $\A$ over $F$; multiplication in $\A$ is given by the multiplication
table \tagref(10). Let $\A_K$ be an $n$-dimensional algebra over $K$ with the same
multiplication table (this is valid since the $\gamma_{ijk}$, being in $F$, are in
$K$). What remains to be verified is that a different choice of basis for
$\A$ over $F$ would yield an algebra isomorphic (over $K$) to this one. (A non-invariant
definition of the Kronecker product of two finite-dimensional
algebras $\A$, $\B$ may similarly be given.)

For the classes of algebras mentioned in the \hyperlink{chapter.1}{Introduction} (Jordan
algebras of characteristic $\ne 2$, and Lie and alternative algebras of
arbitrary characteristic), one may verify that algebras remain in the
same class under scalar extension---a property which is not shared by
classes of algebras defined by more general identities (as, for example,
in \chaplink{5}).

Just as the \emph{commutator} $[x,y] = xy-yx$ measures commutativity (and
lack of it) in an algebra $\A$, the \emph{associator}
\begin{myalign}
\tag{11} &&(x,y,z) = (xy)z - x(yz)
\end{myalign}
of any three elements may be introduced as a measure of associativity
(and lack of it) in $\A$. Thus the definitions of alternative and Jordan
algebras may be written as
\begin{myalign}
&&(x,x,y) = (y,x,x) = 0     &for all $x,y$ in $\A$
\PGx--File: 014.png---\*******\*******\********\********\-------------------
\Intertext{and}
&&[x,y] = (x,y,x^2) = 0  &for all $x,y$ in $\A$.
\end{myalign}
Note that the associator $(x,y,z)$ is linear in each argument. One identity
which is sometimes useful and which holds in any algebra $\A$ is
\begin{myalign}
\tag{12}&& a(x,y,z) + (a,x,y)z = (ax,y,z) - (a,xy,z) + (a,x,yz)\\
&&& for all $a,x,y,z$ in $\A$. %[.** punctuation assumed: off edge of scan]
\end{myalign}

The \emph{nucleus} $\G$ of an algebra $\A$ is the set of elements $g$ in $\A$ which
associate with every pair of elements $x,y$ in $\A$ in the sense that
\begin{myalign}[0em]
\tag{13}&&(g,x,y) = (x,g,y) = (x,y,g) = 0  &for all $x,y$ in $\A$.\\
\end{myalign}
It is easy to verify that $\G$ is an associative subalgebra of $\A$. $\G$ is a
subspace by the linearity of the associator in each argument, and
$\allowbreaks(g_1g_2,x,y) = g_1(g_2,x,y) + (g_1,g_2,x)y + (g_1,g_2x,y) - (g_1,g_2,xy) = 0$
by \tagref(13), etc.

The \emph{center} $\C$ of $\A$ is the set of all $c$ in $\A$ which commute and associate
with all elements; that is, the set of all $c$ in the nucleus $\G$ with the
additional property that
\begin{myalign}
\tag{14} && xc = cx  &for all $x$ in $\A$.
\end{myalign}
This clearly generalizes the familiar notion of the center of an associative
algebra. Note that $\C$ is a commutative associative subalgebra of $\A$.

Let $a$ be any element of an algebra $\A$ over $F$. The \emph{right multiplication}
$R_a$ of $\A$ which is determined by $a$ is defined by
\begin{myalign}
\tag{15}   &&R_a : x \to xa &for all $x$ in $\A$.
\end{myalign}
Clearly $R_a$ is a linear operator on $\A$. Also the set $R(\A)$ of all right
multiplications of $\A$ is a subspace of the associative algebra $\E$ of all
linear operators on $\A$, since $a \to R_a$ is a linear mapping of $\A$ into $\E$.
(In the familiar case of an associative algebra, $R(\A)$ is a subalgebra of
$\E$, but this is not true in general.) Similarly the \emph{left multiplication}
$L_a$ defined by
\begin{myalign}
\tag{16}  &&L_a : x \to ax    &for all $x$ in $\A$
\end{myalign}
\PG--File: 015.png---\*******\********\****\********\----------------------
is a linear operator on $\A$, the mapping $a \to L_a$ is linear, and the set
$L(\A)$ of all left multiplications of $\A$ is a subspace of $\E$.

We denote by $\M(\A)$, or simply $\M$, the enveloping algebra of $R(\A) \cup L(\A)$;
that is, the (associative) subalgebra of $\E$ generated by right and left
multiplications of $\A$. $\M(\A)$ is the intersection of all subalgebras of $\E$
which contain both $R(\A)$ and $L(\A)$. The elements of $\M(\A)$ are of the form
$\sum S_1 \dotsm S_n$ where $S_i$ is either a right or left multiplication of $\A$. We
call the associative algebra $\M = \M(\A)$ the \emph{multiplication algebra} of $\A$.

It is sometimes useful to have a notation for the enveloping algebra
of the right and left multiplications (of $\A$) which correspond to the
elements of any subset $\B$ of $\A$; we shall write $\B^*$ for this subalgebra of $\M(\A)$.
That is, $\B^*$ is the set of all $\sum S_1 \dotsm S_n$, where $S_i$ is either $R_{b_i}$, the
right multiplication of $\A$ determined by $b_i$ in $\B$, or $L_{b_i}$. Clearly $\A^* = \M(\A)$,
but note the difference between $\B^*$ and $\M(\B)$ in case $\B$ is a proper subalgebra
of $\A$---they are associative algebras of operators on different spaces ($\A$
and $\B$ respectively).

An algebra $\A$ over $F$ is called \emph{simple} in case $0$ and $\A$ itself are the
only ideals of $\A$, and $\A$ is not a zero algebra (equivalently, in the presence
of the first assumption, $\A$ is not the zero algebra of dimension $1$). Since
an ideal of $\A$ is an invariant subspace under $\M = \M(\A)$, and conversely, it
follows that $\A$ is simple if and only if $\M \ne 0$ is an irreducible set of
linear operators on $\A$. Since $\A^2$ ($= \A\A$) is an ideal of $\A$, we have $\A^2 = \A$
in case $\A$ is simple.

An algebra $\A$ over $F$ is a \emph{division algebra} in case $\A \ne 0$ and the
equations
\begin{myalign}
\tag{17}&& ax = b,\quad ya = b &($a \ne 0$, $b$ in $\A$)
\end{myalign}
have unique solutions $x$, $y$ in $\A$; this is equivalent to saying that, for any
$a \ne 0$ in $\A$, $L_a$ and $R_a$ have inverses $L_a^{-1}$ and $R_a^{-1}$. Any division algebra is
\PG--File: 016.png---\*******\*****\********\********\---------------------
simple. For, if $\I \ne 0$ is merely a left ideal of $\A$, there is an element
$a \ne 0$ in $\I$ and $\A \subseteq \A a \subseteq \I$ by \tagref(17), or $\I = \A$; also clearly $\A^2 \ne 0$. (Any
associative division algebra $\A$ has an identity $1$, since \tagref(17) implies
that the non-zero elements form a multiplicative group. In general, a
division algebra need not contain an identity $1$.) If $\A$ has finite
dimension $n \ge 1$ over $F$, then $\A$ is a division algebra if and only if $\A$ is
\emph{without zero divisors} ($x \ne 0$ and $y \ne 0$ in $\A$ imply $xy \ne 0$), inasmuch as the
finite-dimensionality insures that $L_a$ (and similarly $R_a$), being (1--1) for
$a \ne 0$, has an inverse.

In order to make the observation that any simple ring is actually
an algebra, so the study of simple rings reduces to that of (possibly
infinite-dimensional) simple algebras, we take for granted that the appropriate
definitions for rings are apparent and we digress to consider any simple
ring $\R$. The (associative) multiplication ring $\M = \M(\R) \ne 0$ is irreducible as
a ring of endomorphisms of $\R$. Thus by Schur's Lemma the centralizer $\C'$
of $\M$ in the ring $\E$ of all endomorphisms of $\R$ is an associative division
ring. Since $\M$ is generated by left and right multiplications of $\R$, $\C'$
consists of those endomorphisms $T$ in $\E$ satisfying $R_y T = TR_y$, $L_xT = TL_x$, or
\begin{myalign}
\tag{18} &&(xy)T = (xT)y = x(yT)  &for all $x, y$ in $\R$.
\end{myalign}
Hence $S$, $T$ in $\C'$ imply $(xy)ST = \left((xS)y\right)T = (xS)(yT) = \left(x(yS)\right)T = (xT)(yS)$.
Interchanging $S$ and $T$, we have $(xy)ST = (xy)TS$, so that $zST = zTS$ for
all $z$ in $\R^2 = \R$. That is, $ST = TS$ for all $S$, $T$ in $\C'$; $\C'$ is a field which
we call the \emph{multiplication centralizer} of $\R$. Now the simple ring $\R$ may be
regarded in a natural way as an algebra over the field $\C'$. Denote $T$ in $\C'$ by
$\alpha$, and write $\alpha x = xT$ for any $x$ in $\R$. Then $\R$ is a (left) vector space over
$\C'$. Also \tagref(18) gives the defining relations $\alpha(xy) = (\alpha x)y = x(\alpha y)$ for an
algebra over $\C'$. As an algebra over $\C'$ (or any subfield $F$ of $\C'$), $\R$ is
simple since any ideal of $\R$ as an algebra is \textit{a priori} an ideal of $\R$ as a ring.
\PG--File: 017.png---\********\*****\******\********\----------------------

Moreover, $\M$ is a dense ring of linear transformations on $\R$ over $\C'$
(Jacobson, Lectures in Abstract Algebra, vol.~II, p.~274), so we have proved

\begin{theorem}[1]
Let $\R$ be a simple ring, and $\M$ be its multiplication ring.
Then the multiplication centralizer $\C'$ of $\M$ is a field, and $\R$ may be regarded
as a simple algebra over any subfield $F$ of $\C'$. $\M$ is a dense ring of linear
transformations on $\R$ over $\C'$.
\end{theorem}

Returning now to any simple algebra $\A$ over $F$, we recall that the
multiplication algebra $\M(\A)$ is irreducible as a set of linear operators on
the vector space $\A$ over $F$. But (Jacobson, ibid) this means that $\M(\A)$ is
irreducible as a set of endomorphisms of the additive group of $\A$, so that
$\A$ is a simple ring. That is, the notions of simple algebra and simple
ring coincide, and \hyperlink{Theorem:1}{Theorem~1} may be paraphrased for algebras as

\begin{theorem}[1$'$]
Let $\A$ be a simple algebra over $F$, and $\M$ be its
multiplication algebra. Then the multiplication centralizer $\C'$ of $\M$ is a
field (containing $F$), and $\A$ may be regarded as a simple algebra over $\C'$.
$\M$ is a dense ring of linear transformations on $\A$ over $\C'$.
\end{theorem}

Suppose that $\A$ has finite dimension $n$ over $F$. Then $\E$ has dimension
$n^2$ over $F$, and its subalgebra $\C'$ has finite dimension over $F$. That is, the
field $\C'$ is a finite extension of $F$ of degree $r = (\C':F)$ over $F$. Then
$n = mr$, and $\A$ has dimension $m$ over $\C'$. Since $\M$ is a dense ring of linear
transformations on (the finite-dimensional vector space) $\A$ over $\C'$, $\M$ is the
set of \emph{all} linear operators on $\A$ over $\C'$. Hence $\C'$ is contained in $\M$ in the
finite-dimensional case. That is, $\C'$ is the center of $\M$ and is called the
\emph{multiplication center} of $\A$.

\begin{corollary}
Let $\A$ be a simple algebra of finite dimension over $F$, and
$\M$ be its multiplication algebra. Then the center $\C'$ of $\M$ is a field, a
\PG--File: 018.png---\*********\******\****\********\----------------------
finite extension of $F$. $\A$ may be regarded as a simple algebra over $\C'$.
$\M$ is the algebra of all linear operators on $\A$ over $\C'$.
\end{corollary}

An algebra $\A$ over $F$ is called \emph{central simple} in case $\A_K$ is simple for
every extension $K$ of $F$. Every central simple algebra is simple (take $K=F$).

We omit the proof of the fact that any simple algebra $\A$ (of arbitrary
dimension), regarded as an algebra over its multiplication centralizer $\C'$ (so
that $\C'=F$) is central simple. The idea of the proof is to show that, for
any extension $K$ of $F$, the multiplication algebra $\M(\A_K)$ is a dense ring of
linear transformations on $\A_K$ over $K$, and hence is an irreducible set of
linear operators.

\begin{theorem}[2]
The center $\C$ of any simple algebra $\A$ over $F$ is either $0$
or a field. In the latter case $\A$ contains $1$, the multiplication centralizer
$\C'= \C^* =\{ R_c\mid c \in \C \}$, and $\A$ is a central simple algebra over $\C$.
\end{theorem}

\begin{proof}
Note that $c$ is in the center of any algebra $\A$ if and only if
$R_c=L_c$ and $[L_c, R_y]=R_cR_y-R_{cy}=R_yR_c-R_{yc}=0$ for all $y$ in $\A$ or,
more compactly,
\begin{myalign}
\tag{19} &&R_c=L_c,\quad R_cR_y=R_yR_c=R_{cy}&for all $y$ in $\A$.
\intertext{Hence \tagref(18) implies that}
\tag{20}&&cT \text{ is in } \C &for all $c$ in $\C$, $T$ in $\C'$.
\intertext{For \tagref(18) may be written as}
\tag{18$'$} &&R_yT=TR_y=R_{yT} &for all $y$ in $\A$
\intertext{or, equivalently, as}
\tag{18$''$} &&L_xT=L_{xT}=TL_x &for all $x$ in $\A$.
\intertext{Then \tagref(18$'$) and \tagref(18$''$) imply $R_{cT}=TR_c=TL_c=L_{cT}$, together with $R_{cT}R_y=
R_cTR_y=R_cR_{yT}=R_{c(yT)}=R_{(cT)y}$ and $R_yR_{cT}=R_yR_cT=R_cR_yT=R_cTR_y$ ($=R_{(cT)y}$),
That is, \tagref(20) holds. Note also that \tagref(19) implies}
\tag{21} &&L_xR_c=R_cL_x & for all $c$ in $\C$, $x$ in $\A$.
\end{myalign}

\PG--File: 019.png---\*********\*****\****\********\-----------------------
Since $R_{c_1}R_{c_2}=R_{c_1c_2}$ ($c_i$ in $C$) by \tagref(19), the subalgebra $\C^*$ of $\M(\A)$ is
just $\C^* = \{R_c \mid c \in \C\}$, and the mapping $c\to R_c$ is a homomorphism of $\C$ onto
$\C^*$. Also \tagref(19) and \tagref(21) imply that $R_c$ commutes with every element of $\M$ so
that $\C^* \subseteq \C'$. Moreover, $\C^*$ is an ideal of the (commutative) field $\C'$ since
\tagref(18$'$) and \tagref(20) imply that $TR_c = R_{cT}$ is in $\C^*$ for all $T$ in $\C'$, $c$ in $\C$. Hence
either $\C^*=0$ or $\C^*=C'$.

Now $\C^*=0$ implies $R_c=0$ for all $c$ in $\C$; hence $\C=0$. For, if there
is $c \ne 0$ in $\C$, then $\I=Fc \ne 0$ is an ideal of $\A$ since $\I\A=\A\I=0$. Then
$\I=\A$, $\A^2=0$, a contradiction.

In the remaining case $\C^*=\C'$, the identity operator $1_\A$ on $\A$ is in $\C'=\C^*$.
Hence there is an element $e$ in $\C$ such that $R_e=L_e=1_\A$, or $ae=ea=a$ for
all $a$ in $\A$; $\A$ has a unity element $1=e$. Then $c \to R_c$ is an isomorphism
between $\C$ and the field $\C'$. $\A$ is an algebra over the field $\C$, and as such
is central simple.
\end{proof}

For any algebra $\A$ over $F$, one obtains a \emph{derived series} of subalgebras
$\A^{(1)} \supseteq \A^{(2)} \supseteq \A^{(3)} \supseteq \dotsb $ by defining
 $\A^{(1)}=\A$, $\A^{(i+1)}=(\A^{(i)})^2$. $\A$ is
called \emph{solvable} in case $\A^{(r)}=0$ for some integer $r$.

\begin{proposition}[1]
If an algebra $\A$ contains a solvable ideal $\I$, and if
$\overline \A=\A/\I$ is solvable, then $\A$ is solvable.
\end{proposition}

\begin{proof}
Since \tagref(1) is a homomorphism, it follows that $\overline{\A^2}={\overline \A}\vphantom{\A}^2$ and that
$\overline{\A^{(i)}}={\overline \A}\vphantom{\A}^{(i)}$. Then ${\overline \A}\vphantom{\A}^{(r)}=0$ implies
 $\overline {\A^{(r)}}=0$, or $\A^{(r)} \subseteq \I $. But $\I^{(s)}=0$
for some $s$, so $\A^{(r+s)}=( \A^{(r)})^{(s)} \subseteq \I^{(s)}=0$. Hence $\A$ is solvable.
\end{proof}

\begin{proposition}[2]
If $\B$ and $\C$ are solvable ideals of an algebra $\A$, then $\B+\C$
is a solvable ideal of $\A$. Hence, if $\A$ is finite-dimensional, $\A$ has a unique
maximal solvable ideal $\N$. Moreover, the only solvable ideal of $\A/\N$ is $0$.
\PG--File: 020.png---\*************\*****\********\******\-----------------
\end{proposition}

\begin{proof} $\B + \C$ is an ideal because $\B$ and $\C$ are ideals. By the second
isomorphism theorem $(\B+\C)/\C\cong \B/(\B\cap \C)$. But $\B/(\B\cap \C)$ is a homomorphic
image of the solvable algebra $\B$, and is therefore clearly solvable. Then
$\B + \C$ is solvable by \hyperlink{Proposition:1}{Proposition~1}. It follows that, if $\A$ is finite-dimensional,
the solvable ideal of maximum dimension is unique (and contains every solvable
ideal of $\A$). Let $\N$ be this maximal solvable ideal, and $\overline\G$ be any solvable
ideal of $\overline \A = \A/\N$. The complete inverse image $\G$ of $\overline\G$ under the natural
homomorphism of $\A$ onto $\overline\A$ is an ideal of $\A$ such that $\G/\N = \overline\G$. Then $\G$ is
solvable by \hyperlink{Proposition:1}{Proposition~1}, so $\G\subseteq\N$. Hence $\G/\N = \overline\G = 0$.
\end{proof}

An algebra $\A$ is called \emph{nilpotent} in case there exists an integer $t$
such that any product $z_1z_2\dotsm z_t$ of $t$ elements in $\A$, no matter how associated,
is $0$. This clearly generalizes the concept of nilpotence as defined for
associative algebras. Also any nilpotent algebra is solvable.

\begin{theorem}[3] An ideal $\B$ of an algebra $\A$ is nilpotent if and only if the
(associative) subalgebra $\B^*$ of $\M(\A)$ is nilpotent.
\end{theorem}

\begin{proof} Suppose that every product of $t$ elements of $\B$, no matter how
associated, is $0$. Then the same is true for any product of more than $t$
elements of $\B$. Let $T = T_1\dotsm T_t$ be any product of $t$ elements of $\B^*$. Then
$T$ is a sum of terms each of which is a product of at least $t$ linear operators
$S_i$, each $S_i$ being either $L_{b_i}$ or $R_{b_i}$ ($b_i$ in $\B$). Since $\B$ is an ideal of $\A$,
$xS_1$ is in $\B$ for every $x$ in $\A$. Hence $xT$ is a sum of terms, each of which is
a product of at least $t$ elements in $\B$. Hence $xT = 0$ for all $x$ in $\A$, or $T = 0$,
$\B^*$ is nilpotent. For the converse we need only that $\B$ is a subalgebra of $\A$.
We show by induction on $n$ that any product of at least $2^n$ elements in $\B$, no
matter how associated, is of the form $bS_1 \dotsm S_n$ with $b$ in $\B$, $S_i$ in $\B^*$.
For $n = 1$, we take any product of at least $2$ elements in $\B$. There is a
final multiplication which is performed. Since $\B$ is a subalgebra, each
\PG--File: 021.png---\*******\*****\********\******\-----------------------
of the two factors is in $\B$: $bb_1 = bR_{b_1} = bS_1$.  Similarly in any product of
at least $2^{n+1}$ elements of $\B$, no matter how associated, there is a final
multiplication which is performed. At least one of the two factors is a
product of at least $2^n$ elements of $\B$, while the other factor $b'$ is in $\B$.
Hence by the assumption of the induction we have either $b'(bS_1\dotsm S_n) =
bS_1\dotsm S_nL_{b'} = bS_1\dotsm S_{n+1}$ or $(bS_1\dotsm S_n)b' = bS_1\dotsm S_nR_{b'} = bS_1\dotsm S_{n+1}$, as
desired. Hence, if any product $S_1\dotsm S_t$ of $t$ elements in $\B^*$ is $0$, any
product of $2^t$ elements of $\B$, no matter how associated, is $0$. That is, $\B$
is nilpotent.
\PG--File: 022.png---\*******\*****\********\******\-----------------------
\end{proof}




\chapter{Alternative Algebras} % III.


As indicated in the \hyperlink{chapter.1}{Introduction}, an \emph{alternative algebra} $\A$ over $F$ is
an algebra in which
\begin{myalign}
\tag{1} &  x^2 y &= x(xy)  &for all $x, y$ in $\A$
\Intertext{and}
\tag{2} &  yx^2 &= (yx)x   &for all $x, y$ in $\A$.
\end{myalign}
In terms of associators, \tagref(1) and \tagref(2) are equivalent to
\begin{myalign}
\tag{1$'$} &  (x,x,y) &= 0   &for all $x, y$ in $\A$
\Intertext{and}
\tag{2$'$} &  (y,x,x) &= 0   &for all $x, y$ in $\A$.
\end{myalign}
In terms of left and right multiplications, \tagref(1) and \tagref(2) are equivalent to
\begin{myalign}
\tag{1$''$} &  L_{x^2} &= {L_x}^2    &for all $x$ in $\A$
\Intertext{and}
\tag{2$''$} &  R_{x^2} &= {R_x}^2    &for all $x$ in $\A$.
\end{myalign}

The associator $(x_1,x_2,x_3)$ ``alternates'' in the sense that, for any
permutation $\sigma$ of $1, 2, 3$, we have $(x_{1\sigma},x_{2\sigma},x_{3\sigma}) = (\sgn \sigma)(x_1,x_2,x_3)$. To
establish this, it is sufficient to prove
\begin{myalign}
\tag{3} &  (x,y,z) &= -(y,x,z)  &for all $x, y, z$ in $\A$
\Intertext{and}
\tag{4} &  (x,y,z) &= (z,x,y)   &for all $x, y, z$ in $\A$.
\end{myalign}
Now \tagref(1$'$) implies that $\allowbreaks(x+y,x+y,z) = (x,x,z) + (x,y,z) + (y,x,z) +
(y,y,z) = (x,y,z) + (y,x,z) = 0$, implying \tagref(3). Similarly \tagref(2$'$) implies
$(x,y,z) = -(x,z,y)$ which gives $(x,z,y) = (y,x,z)$. Interchanging $y$ and $z$,
we have \tagref(4). The fact that the associator alternates is equivalent to
\PG--File: 023.png---\*******\*****\********\*******\----------------------
\begin{myalign}
\tag{5}&&{\begin{aligned}
  R_xR_y - R_{xy} &= L_{xy} - L_yL_x = L_yR_x - R_xL_y = \\
     L_xL_y - L_{yx} &= R_yL_x - L_xR_y = R_{yx} - R_yR_x
\end{aligned}}
\end{myalign}
for all $x, y$ in $\A$. It follows from \tagref(1$''$), \tagref(2$''$) and \tagref(5) that any scalar
extension $\A_K$ of an alternative algebra $\A$ is alternative.

Now \tagref(3) and \tagref(2$'$) imply
\begin{myalign}
\tag{6} &&  (x,y,x) = 0       &for all $x, y$ in $\A$;
\Intertext{that is,}
\tag{6$'$} && (xy)x = x(yx)   &for all $x, y$ in $\A$,
\Intertext{or}
\tag{6$''$} && L_xR_x = R_xL_x   &for all $x$ in $\A$.
\end{myalign}
Identity \tagref(6$'$) is called the \emph{flexible} law. All of the algebras mentioned in
the \hyperlink{chapter.1}{Introduction} (Lie, Jordan and alternative) are flexible. The linearized
form of the flexible law is
\begin{myalign}
\tag{6$'''$} && (x,y,z) + (z,y,x) = 0 &for all $x, y, z$ in $\A$.
\end{myalign}

We shall have occasion to use the Moufang identities
\begin{myalign}
\tag{7} &&     (xax)y = x\left[a(xy)\right],\\
\tag{8} &&     y(xax) = \left[(yx)a\right]x,\\
\tag{9} &&     (xy)(ax) = x(ya)x
\end{myalign}
for all $x, y, a$ in an alternative algebra $\A$ (where we may write $xax$ unambiguously
by \tagref(6$'$)). Now $\allowbreaks(xax)y - x\left[a(xy)\right] = (xa,x,y) + (x,a,xy) = (-x,xa,y) - (x,xy,a) =
-\left[x(xa)\right]y + x\left[(xa)y\right] - \left[x(xy)\right]a + x\left[(xy)a\right] = -(x^2a)y - (x^2y)a + x\left[(xa)y +
(xy)a\right] = -(x^2,a,y) - (x^2,y,a) - x^2(ay) - x^2(ya) + x\left[(xa)y + (xy)a\right] = x\left[-x(ay)\right. -
x(ya) + (xa)y + \left.(xy)a\right] = x\left[(x,a,y) + (x,y,a)\right] = 0$, establishing \tagref(7). Identity
\tagref(8) is the reciprocal relationship (obtained by passing to the anti-isomorphic
algebra, which is alternative since the defining identities are reciprocal).
Finally \tagref(7) implies $\allowbreaks(xy)(ax)-x(ya)x = (x,y,ax) + x\left[y(ax)-(ya)x\right] =
-(x,ax,y)-x(y,a,x) = -(xax)y + x\left[(ax)y\right.-\left.(y,a,x)\right] = -x\left[a(xy)\right.-(ax)y +
\PGx--File: 024.png---\*******\*****\********\*******\----------------------
\left.(y,a,x)\right] = -x\left[-(a,x,y) + (y,a,x)\right] = 0$, or \tagref(9) holds.

\begin{theorem}[of Artin]
The subalgebra generated by any two elements $x,y$
of an alternative algebra $\A$ is associative.
\end{theorem}

\begin{proof}
Define powers of a single element $x$ recursively by $x^1 = x$,
$x^{i+1} = xx^i$. Show first that the subalgebra $F[x]$ generated by a single
element $x$ is associative by proving
\begin{myalign}
\tag{10} &  x^ix^j &= x^{i+j}    &for all $x$ in $\A$ ($i, j = 1,2,3,\dots$).\\
\intertext{We prove this by induction on $i$, but shall require the case $j = 1$:}
\tag{11} &x^ix &= xx^i     &for all $x$ in $\A$ ($i = 1,2,\dots$).\\
\end{myalign}
Proving \tagref(11) by induction, we have $x^{i+1}x = (xx^i)x = x(x^ix) = x(xx^i) = xx^{i+1}$
by flexibility and the assumption of the induction. We have \tagref(10) for $i = 1,2$
by definition and \tagref(1). Assuming \tagref(10) for $i \geq 2$, we have $x^{i+1}x^j = (xx^i)x^j =
\left[x(xx^{i-1})\right]x^j = \left[x(x^{i-1}x)\right]x^j = x\left[x^{i-1}(xx^j)\right] = x(x^{i-1}x^{j+1}) = xx^{i+j} = x^{i+j+1}$ by
\tagref(11), \tagref(7) and the assumption of the induction. Hence $F[x]$ is associative.

Next we prove that
\begin{myalign}
\tag{12}&   x^i(x^jy) &= x^{i+j}y   &for all $x,y$ in $\A$ ($i,j = 1,2,3,\dots$).\\
\intertext{First we prove the case $j = 1$:}
\tag{13}&   x^i(xy) &= x^{i+1}y    &for all $x,y$ in $\A$ ($i = 1,2,3,\dots$).\\
\end{myalign}

The case $i = 1$ of \tagref(13) is given by \tagref(1); the case $i = 2$ is $x^2(xy) = x\left[x(xy)\right] =
(xxx)y = x^3y$ by \tagref(1) and \tagref(7). Then for $i \ge 2$, write the assumption \tagref(13) of the
induction with $xy$ for $y$ and $i$ for $i + 1$: $x^{i-1}\left[x(xy)\right] = x^i(xy)$. Then
$x^{i+1}(xy) = (xx^{i-1}x)(xy) = x\left[x^{i-1}\left\{x(xy)\right\}\right] = x\left[x^i(xy)\right] = (xx^ix)y = x^{i+2}y$ by
\tagref(7). We have proved the case $j = 1$ of \tagref(12). Then with $xy$ written for $y$ in
\tagref(12), the assumption of the induction is $x^{i+j}(xy) = x^i\left[x^j(xy)\right]$. It follows that
$x^i(x^{j+1}y) = x^i\left[x^j(xy)\right] = x^{i+j}(xy) = x^{i+j+1}y$ by \tagref(13). Now \tagref(12) holds identically
in y. Hence
\begin{myalign}
\tag{14}&&
  x^i(x^jy^k) = (x^ix^j)y^k.
\PGx--File: 025.png---\*************\********\********\*******\-------------
\intertext{Reciprocally}
\tag{15}&& (y^kx^j)x^i = y^k(x^jx^i).
\intertext{Since the distributive law holds in $\A$, it is sufficient now to show that}
\tag{16}&& (x^iy^k)x^j = x^i(y^kx^j)
\end{myalign}
in order to show that the subalgebra generated by $x,y$ is associative. But
\tagref(14) implies $(x^i,y^k,x^j)=-(x^i,x^j,y^k)=0$.
\end{proof}

An algebra $\A$ over $F$ is called \emph{power-associative} in case the subalgebra
$F[x]$ of $\A$ generated by any element $x$ in $\A$ is associative. Any alternative
algebra is power-associative; the \hyperlink{Theorem:of Artin}{Theorem} of Artin also implies
\begin{myalign}
\tag{17} && {R_x}^j=R_{x^j},\qquad {L_x}^j=L_{x^j} &for all $x$ in $\A$.
\end{myalign}

An element $x$ in a power-associative algebra $\A$ is called \emph{nilpotent} in
case there is an integer $r$ such that $x^r = 0$. An algebra (ideal) consisting
only of nilpotent elements is called a \emph{nilalgebra} (\emph{nilideal}).

\begin{theorem}[4]
Any alternative nilalgebra $\A$ of finite dimension over $F$
is nilpotent.
\end{theorem}

\begin{proof}
Any subalgebra $\B$ of $\A$ is generated by a finite number of elements
(for example, the elements in a basis for $\B$ over $F$). We prove by induction
on the number of generators of $\B$ that $\B^*$ is nilpotent for all subalgebras $\B$;
hence, in particular, for $\B = \A$. If $\B$ is generated by one element $x$, then
by \tagref(6$''$) and \tagref(17) any $T$ in $\B^*$ is a linear combination of operators of the
form
\begin{myalign}
\tag{18} &&{R_x}^{j_1},~ {L_x}^{j_2},~ {R_x}^{j_3}{L_x}^{j_4} &for $j_i \ge 1$.
\end{myalign}
Then, if $x^j = 0$, we have $T^{2j-1} = 0$,  $\B^*$ is nilpotent. Hence, by the
assumption of the induction, we may take a maximal proper subalgebra $\B$ of
$\A$ and know that $\B^*$ is nilpotent. But then there exists an element $x$ not in
$\B$ such that
\PG--File: 026.png---\*************\********\****\********\----------------
\begin{myalign}
\tag{19}&& x\B^* \subseteq\B.
\end{myalign}
For $\B^{*r} = 0$ implies that $\A\B^{*r}=0 \subseteq\B$, and there exists a smallest integer
$m \ge 1$ such that $\A\B^{*m} \subseteq\B$. If $m = 1$, take $x$ in $\A$ but not in $\B$; if $m > 1$, take
$x$ in $\A\B^{*m-1}$ but not in $\B$. Then \tagref(19) is satisfied. Since $\B$ is maximal, the
subalgebra generated by $\B$ and $x$ is $\A$ itself. It follows from \tagref(19) that
$\A = \B + F[x]$ so that $\M = \A^* = (\B + Fx)^*$. Put $y = b$ in \tagref(5) for any $b$ in $\B$.
Then \tagref(19) implies that
\begin{myalign}
\tag{20}&&
{\begin{aligned}
R_xR_b &= R_{b_1}-R_bR_x,\quad R_xL_b=L_bR_x+R_bR_x-R_{b_2},\\
L_xR_b &= R_bL_x+L_bL_x-L_{b_3},\quad L_xL_b=L_{b_1}-L_bL_x
\end{aligned}}
\end{myalign}
for $b_i$ in $\B$. Equations \tagref(20) show that, in each product of right and left
multiplications in $\B^*$ and $(Fx)^*$, the multiplication $R_x$ or $L_x$ may be
systematically passed from the left to the right of $R_b$ or $L_b$ in a fashion
which, although it may change signs and introduce new terms, preserves the
number of factors from $\B^*$ and does not increase the number of factors from
$(Fx)^*$. Hence any $T$ in $\A^*= (\B + Fx)^*$ may be written as a linear combination
of terms of the form \tagref(18) and others of the form
\begin{myalign}
&&B_1,\quad B_2{R_x}^{m_1},\quad B_3{L_x}^{m_2},\quad B_4{R_x}^{m_3}{L_x}^{m_4}
\end{myalign}
for $B_i$ in $\B^*$, $m_i \ge 1$. Then if $\B^{*r} = 0$ and $x^j = 0$, we have $T^{r(2j-1)}=0$;
for every term in the expansion of $T^{r(2j-1)}$ contains either an uninterrupted
sequence of at least $2j-1$ factors from $(Fx)^*$ or at least $r$ factors $B_i$. In the
latter case the $R_x$ or $L_x$ may be systematically passed from the left to the
right of $B_i$ (as above) preserving the number of factors from $\B^*$, resulting
in a sum of terms each containing a product $B_1 B_2 \dotsm B_r = 0$. Hence every
element $T$ of the finite-dimensional associative algebra $\A^*$ is nilpotent.
Hence $\A^*$ is nilpotent (Albert, Structure of Algebras, p.~23). Hence $\A$ is
nilpotent by \hyperlink{Theorem:3}{Theorem~3}.
\PG--File: 027.png---\********\******\****\********\-----------------------
\end{proof}

Any nilpotent algebra is solvable, and any solvable (power-assoc\-iative)
algebra is a nilalgebra. By \hyperlink{Theorem:4}{Theorem~4} the concepts of nilpotent algebra,
solvable algebra, and nilalgebra coincide for finite-dimensional alternative
algebras. Hence there is a unique maximal nilpotent ideal $\N$ ($=$ solvable
ideal $=$ nilideal) in any finite-dimensional alternative algebra $\A$; we call
$\N$ the \emph{radical} of $\A$. We have seen that the radical of $\A/\N$ is $0$.

We say that $\A$ is \emph{semisimple} in case the radical of $\A$ is $0$, and omit the
proof that any finite-dimensional semisimple alternative algebra $\A$ is the
direct sum $\A = \Ss_1 \oplus \dotsb \oplus \Ss_t$ of simple algebras $\Ss_i$. The proof is dependent
upon the properties of the \emph{Peirce decomposition} relative to an idempotent $e$.

An element $e$ of an (arbitrary) algebra $\A$ is called an \emph{idempotent} in
case $e^2 = e \ne 0$.

\begin{proposition}[3]
Any finite-dimensional power-associative algebra, which
is not a nilalgebra, contains an idempotent $e$ ($\ne 0$).
\end{proposition}

\begin{proof}
$\A$ contains an element $x$ which is not nilpotent. The subalgebra
$F[x]$ of $\A$ generated by $x$ is a finite-dimensional associative algebra which
is not a nilalgebra. Then $F[x]$ contains an idempotent $e$ ($\ne 0$) (Albert, ibid),
and therefore $\A$ does.
\end{proof}

By \tagref(1$''$) and \tagref(2$''$) $L_e$ and $R_e$ are idempotent operators on $\A$ which commute by
\tagref(6$''$) (``commuting projections''). It follows that $\A$ is the vector space direct
sum
\begin{myalign}[6em]
\tag{21}   &&\A = \A_{11} + \A_{10} + \A_{01} + \A_{00}
\intertext{where $\A_{ij}$ ($i, j = 0, 1$) is the subspace of $\A$ defined by}
\tag{22}  &&\A_{ij} = \{x_{ij} \mid ex_{ij} = ix_{ij},\; x_{ij} e = jx_{ij}\} &$i,j = 0, 1$.
\end{myalign}
Just as in the case of associative algebras, the decomposition of any element
$x$ in $\A$ according to the \emph{Peirce decomposition} \tagref(21) is
\begin{myalign}(0em)
\tag{23}&&  x = exe + (ex - exe) + (xe - exe) + (x - ex - xe + exe).
\end{myalign}
\PG--File: 028.png---\********\********\****\********\---------------------
We derive a few of the properties of the Peirce decomposition as follows:
\begin{myalign}
& (x_{ij}y_{ji})e &= (x_{ij},y_{ji},e ) + x_{ij}(y_{ji}e )\\
&                 &= - (x_{ij},e,y_{ji} ) + x_{ij}(y_{ji}e )\\
&                 &= - jx_{ij}y_{ji} + jx_{ij}y_{ji} + ix_{ij}y_{ji}\\
&                 &= ix_{ij}y_{ji}
\end{myalign}
and similarly $e (x_{ij}y_{ji}) = ix_{ij}y_{ji}$, so
\begin{myalign}
\tag{24}     &&\A_{ij}\A_{ji} \subseteq \A_{ii},   &     $i,j=0,1$.
\end{myalign}
That is, $\A_{11}$ and $\A_{00}$ are subalgebras of $\A$, while $\A_{10}\A_{01} \subseteq\A_{11}$,
 $\A_{01}\A_{10} \subseteq \A_{00}$.
Also $x_{11}y_{00} = (ex_{11}e)y_{00} = e\left[x_{11}(ey_{00})\right] = 0$ by \tagref(7), and similarly $y_{00}x_{11} = 0$.
Hence $\A_{11}$ and $\A_{00}$ are orthogonal subalgebras of $\A$.  Similarly $\A_{ii}\A_{ij} \subseteq \A_{ij}$,
$\A_{ij}\A_{jj} \subseteq \A_{ij}$, etc.

We wish to define the class of Cayley algebras mentioned in the
\hyperlink{chapter.1}{Introduction}. We construct these algebras in the following manner. The
procedure works slightly more smoothly if we assume that $F$ has characteristic
$\ne 2$, so we make this restriction here although it is not necessary.

An algebra $\A$ with $1$ over $F$ is called a \emph{quadratic algebra} in case
$\A \ne F1$ and for each $x$ in $\A$ we have
\begin{myalign}
\tag{25}   &&x^2 - t(x)x + n(x)1 = 0, &$t(x)$, $n(x)$ in $F$.
\end{myalign}
If $x$ is not in $F1$, the scalars $t(x)$, $n(x)$ in \tagref(25) are uniquely determined;
set $t(\alpha 1) = 2\alpha$, $n(\alpha 1) = \alpha^2$ to make the \emph{trace} $t(x)$ linear and the \emph{norm} $n(x)$
a quadratic form.

An \emph{involution} (\emph{involutorial anti-isomorphism}) of an algebra $\A$ is a linear
operator $x \to \overline x$ on $\A$ satisfying
\begin{myalign}[8em]
\tag{26}  &&\overline{xy} = \overline{y}\: \overline{x},  \quad\overline{\overline{x}}= x  &for all $x, y$ in $\A$.
\intertext{Here we are concerned with an involution satisfying}
\tag{27}  &&x + \overline{x} \in F1,  \quad x\overline{x} (=\overline{x}x) \in F1 &for all $x$ in $\A$.
\intertext{Clearly \tagref(27) implies \tagref(25) with}
\PGx--File: 029.png---\****\********\****\********\-------------------------
\tag{27$'$} &&x + \overline x = t(x)1, \quad x\overline x(=\overline{x}x) = n(x)1&for all $x$ in $\A$
\end{myalign}
(since $\overline{1} = 1$, we have $t(\alpha 1) = 2 \alpha$, $n(\alpha 1) = \alpha^2$ from \tagref(27)).

Let $\B$ be an algebra with $1$ having dimension $n$ over $F$ and such that $\B$
has an involution $x \to \overline{x}$ satisfying \tagref(27). We construct an algebra $\A$ of
dimension $2n$ over $F$ with the same properties and having $\B$ as subalgebra
(with $1 \in \B$) as follows: $\A$ consists of all ordered pairs $x = (b_1,b_2)$,
$b_i$ in $\B$, addition and multiplication by scalars defined componentwise, and
multiplication defined by
\begin{myalign}
\tag{28}&&(b_1,b_2)(b_3,b_4) = (b_1 b_3 + \mu b_4 \overline{b_2},\; \overline{b_1} b_4 + b_3 b_2)
\end{myalign}
for all $b_i$ in $\B$ and some $\mu \ne 0$ in $F$. Then $1 = (1,0)$ is a unity element
for $\A$, $\B' = \{(b,0) \mid b \in \B\}$ is a subalgebra of $\A$ isomorphic to $\B$, $v = (0, 1)$
is an element of $\A$ such that $v^2 = \mu1$ and $\A$ is the vector space direct sum
$\A = \B' + v\B'$ of the $n$-dimensional vector spaces $\B'$, $v\B'$. Identifying $\B'$ with
$\B$, the elements of $\A$ are of the form
\begin{myalign}
\tag{29} &&x = b_1 + v b_2 &($b_1$ in $\B$ uniquely determined by $x$),\\
\end{myalign}
and \tagref(28) becomes
\begin{myalign}
\tag{28$'$}&& (b_1 + v b_2)(b_3 + v b_4) = (b_1 b_3 + \mu b_4 \overline{b_2}) + v(\overline{b_1} b_4 + b_3 b_2 )
\end{myalign}
for all $b_i$ in $\B$ and some $\mu \ne 0$ in $F$. Defining
\begin{myalign}
\tag{30}&& \overline x = \overline{b_1} - v b_2,
\end{myalign}
we have $\overline {xy} = \overline y \:\overline x$ by \tagref(28$'$) since $b \to \overline b$ is an
involution of $\B$; hence $x \to \overline x$ is an involution of $\A$. Also
\begin{myalign}
&&x + \overline x = t(x) 1, \qquad x \overline x(= \overline xx) = n(x)1
\intertext{where, for $x$ in \tagref(29), we have}
\tag{31} &&t(x) = t(b_1), \qquad n(x) = n(b_1) - \mu n(b_2).
\end{myalign}

Assume that the norm on $\B$ is a nondegenerate quadratic form; that is, the associated symmetric bilinear form
\begin{myalign}
\tag{32}&& (a,b) = \tfrac12 \left[n(a+b) - n(a) - n(b)\right] \quad  (= \tfrac12 t(a \overline b))
\end{myalign}
is nondegenerate (if $(a,b) = 0$ for all $b$ in $\B$, then $a = 0$). Then the norm
\PG--File: 030.png---\*******\********\****\********\----------------------
$n(x)$ on $\A$ defined by \tagref(31) is nondegenerate. For $y = b_3 + vb_4$ implies that
$\allowbreaks (x,y) =
\frac12\left[n(x+y)-n(x)-n(y)\right] = \frac12\left[n(b_1 + b_3)\right. - \mu n(b_2 + b_4) - n(b_1) + \mu n(b_2) -
n(b_3) + \left.\mu n(b_4)\right] = (b_1,b_3) - \mu(b_2,b_4)$. Hence $(x,y)=0$ for all $y = b_3 +
vb_4$ implies $(b_1,b_3) = \mu(b_2,b_4)$ for all $b_3, b_4$ in $\B$. Then $b_4=0$ implies
$(b_1,b_3) = 0$ for all $b_3$ in $\B$, or $b_1 = 0$ since $n(b)$ is nondegenerate on $\B$;
similarly $b_3 = 0$ implies $(b_2,b_4) = 0$ (since $\mu \ne 0$) for all $b_4$ in $\B$, or
$b_2 = 0$. That is, $x = 0$; $n(x)$ is nondegenerate on $\A$.

When is $\A$ alternative? Since $\A$ is its own reciprocal algebra, it is
sufficient to verify the left alternative law \tagref(1$'$), which is equivalent
to $(x,\overline{x},y) = 0$ since $(x,\overline{x},y) = (x,t(x)1-x, y) = -(x,x,y)$.
Now $\allowbreaks % we use explicit sized barackets rather than \left and \right to facilitate linebreaks
(x,\overline{x},y) =
n(x)y - (b_1 + vb_2)\big[(\overline{b_1}b_3 - \mu b_4 \overline{b_2}) + v(b_1b_4 - b_3b_2)\big] = n(x)y - \big[b_1(\overline{b_1}b_3) -
\mu b_1(b_4\overline{b_2}) + \mu(b_1b_4)\overline{b_2} - \mu(b_3b_2)\overline{b_2}\big] - v\big[\overline{b_1}(b_1b_4) - \overline{b_1}(b_3b_2) + (\overline{b_1}b_3)b_2 -
\mu(b_4\overline{b_2})b_2\big] = n(x)y - \big[n(b_1)-\mu n(b_2)\big] (b_3 + vb_4) - \mu(b_1,b_4,\overline{b_2}) - v(\overline{b_1},b_3,b_2) =
-\mu(b_1,b_4,\overline{b_2}) - v(\overline{b_1},b_3,b_2)$ by a trivial extension of the \hyperlink{Theorem:of Artin}{Theorem} of Artin.
Hence $\A$ is alternative if and only if $\B$ is associative.

The algebra $F1$ is not a quadratic algebra, but the identity operator on
$F1$ is an involution satisfying \tagref(27); also $n(\alpha1)$ is nondegenerate on $F1$.
Hence we can use an iterative process (beginning with $\B = F1$) to obtain by
the above construction algebras of dimension $2^t$ over $F$; these depend
completely upon the $t$ nonzero scalars $\mu_1,\mu_2,\dotsc,\mu_t$ used in the successive
steps. The norm on each algebra is a nondegenerate quadratic form. The
$2$-dimensional algebras $\Z = F1 + v_1(F1)$ are either quadratic fields over $F$
($\mu_1$ a nonsquare in $F$) or isomorphic to $F \oplus F$ ($\mu_1$ a square in $F$). The
$4$-dimensional algebras $\Q = \Z + v_2\Z$ are associative central simple algebras
(called \emph{quaternion algebras}) over $F$; any $\Q$ which is not a division algebra
is (by Wedderburn's theorem on simple associative algebras) isomorphic to the
algebra of all $2\times2$ matrices with elements in $F$.
\PG--File: 031.png---\*************\********\****\********\----------------

We are concerned with the $8$-dimensional algebras $\C = \Q + v_3\Q$ which
are called \emph{Cayley algebras} over $F$. Since any $\Q$ is associative, Cayley
algebras are alternative. However, no Cayley algebra is associative.
For $\Q$ is not commutative and there exist $q_1$, $q_2$ in $\Q$ such that $[q_1,q_2]\ne 0$;
hence $(v_3,q_2,q_1) = (v_3q_2)q_1 - v_3(q_2q_1) = v_3[q_1,q_2] \ne 0$ by \tagref(28$'$).
Thus this iterative process of constructing alternative algebras stops after
three steps. The quadratic form $n(x)$ is nondegenerate; also it \emph{permits
composition} in the sense that
\begin{myalign}
\tag{33} &&n(xy)=n(x)n(y) &for all $x, y$ in $\C$.
\intertext{For $n(xy)1=(xy)(\overline{xy})=xy\overline{y}\:\overline{x}=n(y)x\overline{x}=n(x)n(y)1$. Also}
\tag{34} &&t\left((xy)z\right) =t\left(x(yz)\right) &for all $x, y, z$ in $\A$.
\end{myalign}
For $\allowbreaks(x,y,z) = -(z,y,x) = (\overline{z},\overline{y},\overline{x})$ implies $(xy)z + \overline{z}(\overline{y}\:\overline{x}) = x(yz)+\break % heavyhandedness is the only solution here
 (\overline{z}\:\overline{y})\overline{x}$,
so that \tagref(34) holds.

\begin{theorem}[5]
Two Cayley algebras $\C$ and $\C'$ are isomorphic if and only if
their corresponding norm forms $n(x)$ and $n'(x')$ are equivalent (that is, there
is a linear mapping $x\to xH$ of $\C$ into $\C'$ such that
\begin{myalign}
\tag{35} &&n'(xH) = n(x) &for all $x$ in $\C$;
\end{myalign}
$H$ is necessarily (1--1) since $n(x)$ is nondegenerate).
\end{theorem}

\begin{proof}
Suppose $\C$ and $\C'$ are isomorphic, the isomorphism being $H$.
Then \tagref(25) implies $(xH)^2 - t(x)(xH) + n(x)1' = 0$ where $1'=1H$ is the unity
element of $\C'$. But also $(xH)^2 - t'(xH)(xH) + n'(xH)1' = 0$. Hence
$\left[t'(xH)-t(x)\right](xH) + \left[n(x) - n'(xH)\right]1' = 0$. If $x\notin F1$, then $xH\notin F1'$ and
$n(x)=n'(xH)$. On the other hand $n(\alpha1)=\alpha^2=n'(\alpha1')$, and we have \tagref(35) for
all $x$ in $\C$.

For the converse we need to establish the fact that, if $\B$ is a proper
subalgebra of a Cayley algebra $\C$, if $\B$ contains the unity element $1$ of $\C$, and
if (relative to the nondegenerate symmetric bilinear form $(x,y)$ defined on $\C$ by
\tagref(32)) $\B$ is a non-isotropic subspace of $\C$ (that is, $\B\cap\B^\perp=0$), then there is a
\PG--File: 032.png---\************\********\****\********\-----------------
subalgebra $\A = \B + v\B$ (constructed as above). For the involution $x \to \overline x$
on $\C$ induces an involution on $\B$, since $\overline b=t(b)1-b$ is in $\B$ for all $b$ in
$\B$. Also $\B$ non-isotropic implies $\C=\B \perp\B^{\perp}$ with $\B^{\perp}$ non-isotropic (Jacobson,
Lectures in Abstract Algebra, vol.~II, p.~151; Artin, Geometric Algebra, p.~117).
Hence there is a non-isotropic vector $v$ in $\B^{\perp}$, $n(v)=-\mu \ne 0$.
Since $t(v) = t(v \overline1) = 2(v,1) = 0$, we have
\begin{myalign}
\tag{36} &&v^2=\mu 1,   &$\mu \ne 0$ in $F$.
\end{myalign}
Now $v\B \subseteq\B^\perp$ since \tagref(34) implies $(va,b)=\frac{1}{2} t\left((va)\overline b\right) = \frac{1}{2} t\left(v(a \overline b)\right)=(v,b \overline a)=0$
for all $a, b$ in $\B$. Hence $\B \perp v\B$. Also $v\B$ has the same dimension as $\B$ since
$b \to vb$ is (1--1). Suppose $vb = 0$; then $v(vb) = v^2b = \mu b = 0$, implying $b = 0$.
In order to show that $\A = \B \perp v\B$ is the algebra constructed above, it remains
to show that
\begin{myalign}
\tag{37} &&a(vb) = v(\overline ab ),\\
\tag{38} &&(va)b = b(va),\\
\tag{39} &&(va)(vb) = \mu b \overline a
\intertext{for all $a, b$ in $\B$. Now $t(v) = 0$ implies $\overline v=-v$; hence $v$ in $\B^{\perp}$ implies
$0 = 2(v,b) = t(v\overline b) = v\overline b + b\overline v = v\overline b -bv$, or}
\tag{40} &&bv = v \overline b            &for all $b$ in $\B$.
\end{myalign}
Hence $\allowbreaks(v,\overline a, b) + (\overline a,v,b)=0=(v \overline a)b-v(\overline a b)+(\overline a v)b=(v \overline a)b-v(\overline a b)
+(va)b-\overline a(vb)=\left[t(a)1- \overline a\right]vb-v(\overline a b)$, establishing % [** typo:  extablishing]
 \tagref(37). Applying the
involution to $\overline b(va)=v(ba)$, and using \tagref(40), we have \tagref(38). Finally $(va)(vb) =
(va)(\overline b v)= v(a\overline b)v=v^2(b\overline a)= \mu b\overline a$ by the Moufang identity \tagref(9).
 Hence $\A =\B \perp\B^{\perp}$
is the subalgebra specified. Since $\B$ and $\B^{\perp}$ are non-isotropic, so is $\A$.
[Remark: we have shown incidentally that if $\Q$ is any quaternion subalgebra
containing $1$ in a Cayley algebra $\C$, then $\Q$ may be used in the construction
of $\C$ as $\C = \Q + v\Q$.]

Now let $\C$ and $\C'$ have equivalent norm forms $n(x)$ and $n'(x')$. Let $\B$
(and $\B'$) be as above. If $\B$ and $\B'$ are isomorphic under $H_0$,
then the
\PG--File: 033.png---\*************\******\********\*******\---------------
restrictions of $n(x)$ and $n'(x')$ to $\B$ and $\B'$ are equivalent. Then by
Witt's theorem (Jacobson, ibid, p.~162; Artin, ibid, p.~121), since $n(x)$
and $n'(x')$ are equivalent, the restrictions of $n(x)$ and $n'(x')$ to $\B^\perp$ and
$\B'^\perp$ are equivalent. Choose $v$ in $\B^\perp$ with $n(v) \ne 0$; correspondingly we have
$v'$ in $\B'^\perp$ such that $n'(v') = n(v)$. Then $a + vb \to aH_0 + v'(bH_0)$ is an
isomorphism of $\B\perp v\B$ onto $\B' \perp v'\B'$ by the construction above. Hence if
we begin with $\B = F1$, $\B' = F1'$, repetition of the process gives successively
isomorphisms between $\Z$ and $\Z'$, $\Q$ and $\Q'$, $\C$ and $\C'$.
\end{proof}

A Cayley algebra $\C$ is a division algebra if and only if $n(x) \ne 0$ for
every  $x \ne 0$ in $\C$. For $x \ne 0$, $n(x) = 0$ imply $x\overline x = n(x)1 = 0$, $\C$ has zero
divisors. Conversely, if $n(x) \ne 0$, then $\overline x(xy) = (\overline xx)y = n(x)y$ for all $y$
implies $\frac{1}{n(x)}L_xL_{\overline x} = 1_\C$, $L_x^{-1} = \frac{1}{n(x)} L_{\overline x}$
 and similarly $R_x^{-1} = \frac{1}{n(x)} R_{\overline x}$;
hence if $n(x) \ne 0$ for all $x \ne 0$, then $\C$ is a division algebra.

[Remark: If $F$ is the field of all real numbers, the norm form $n(x) =
\sum {\alpha_i}^2$  for $x = \sum \alpha_i u_i$ clearly has the property above. Also there are alternative
algebras $F1$, $\Z$, $\Q$, $\C$ with this norm form (take $\mu_i = -1$ at each step). Hence
there are real alternative division algebras of dimensions $1$, $2$, $4$, $8$. It
has recently been proved (see reference \cite{Ref12} of the appended bibliography
of recent papers) that finite-dimensional real division algebras can have
only these dimensions. It is not true, however, that the only finite-dimensional
real division algebras are the four listed above; they are the
only alternative ones. For other examples of finite-dimensional real division
algebras (necessarily of these specified dimensions of course) see reference
[23] in the bibliography of the 1955 Bulletin \hyperlink{cite.Ref64}{article}.]

\begin{corollary} Any two Cayley algebras $\C$ and $\C'$ with divisors of zero
are isomorphic.
\PG--File: 034.png---\*************\********\****\********\----------------
\end{corollary}

\begin{proof} Show first that $\C$ has divisors of zero if and only if there
is $w \notin F1$ such that $w^2 = 1$. For $1 - w \ne 0$, $1 + w \ne 0$ imply $(1 - w)(1 + w) =
1 - w^2 = 0$ (note $t(w) = 0$ implies $\overline {1 \pm w} = 1 \mp w$ so that $n(1 \pm w ) =0$).
Conversely, if $\C$ has divisors of zero, there exists $x \ne 0$ in $\C$ with $n(x) = 0$.
Then $ x= \alpha 1 + u$, $u \in {(F1)}^ \perp =\{u \mid t(u)=0 \}$ implies $0 = n(x)1 = x\overline{x} =
(\alpha 1 + u)(\alpha 1 - u) = \alpha^2 1 - u^2$.  If $\alpha \ne 0$, then $w = \alpha^{-1}u$ satisfies $w^2 =
1$ ($w \notin F1$).  If $\alpha=0$, then $n(u) = 0$ so that $u$ is an isotropic vector in the
non-isotropic space $(F1)^\perp$. Hence there exists $w$ in $(F1)^\perp$ with $n(w) = -1$
(Jacobson, ibid, p.~154, ex.~3), or $w^2 = t(w)w - n(w)1 = 1$ ($w \notin F1$).

Now let $e_1  = \frac{1}{2}(1- w)$, $e_2 = 1- e_1 = \frac{1}{2}(1+ w)$. Then ${e_1}^2 = e_1$, ${e_2}^2 = e_2$,
$e_1e_2=e_2e_1 = 0$ ($e_1$ and $e_2$ are \emph{orthogonal idempotents}). Also $n(e_i)=0$ for $i=
1,2$. Hence every vector in $e_i\C$  is isotropic since $n(e_ix) = n(e_i)n(x) = 0$.
This means that $e_i\C$ is a totally isotropic subspace ($e_i\C \subseteq (e_i\C)^\perp$). Hence
$\dim(e_i\C) \le \frac{1}{2} \dim\C = 4$ (Jacobson, p.~170; Artin, p.~122). But $x=1x =e_1x+
e_2x$ for all $x$ in $\C$, so $\C=e_1\C+e_2\C$. Hence $\dim (e_i\C)=4$, and $n(x)$ has
maximal Witt index $= 4 = \frac{1}{2}\dim\C$. Similarly $n'(x')$ has maximal Witt index $= 4$.
Hence $n(x)$ and $n'(x')$ are equivalent (Artin, ibid). By \hyperlink{Theorem:5}{Theorem} 5, $\C$ and $\C'$
are isomorphic.
\end{proof}

Over any field $F$ there is a Cayley algebra without divisors of zero
(take $\mu=1$ so $v^2 = 1$). This unique Cayley algebra over $F$ is called the
\emph{split Cayley algebra} over $F$.

$F1$ is both the nucleus and center of any Cayley algebra. Also any
Cayley algebra is simple (hence central simple over $F$). (This is obvious
for all but the split Cayley algebra.) For, if $\I$ is any nonzero ideal of
$\C$, there is $x \ne 0$ in $\I$. But $x$ is contained is some quaternion subalgebra
$\Q$ of $\C$. Then $\Q\times \Q$ is an ideal of the simple algebra $\Q$. Hence $1\in\Q =
\Q\times\Q \subseteq\I$, and $\I = \C$.

We omit the proof of the fact that the only alternative central simple
\PG--File: 035.png---\*******\*****\****\********\-------------------------
algebras of finite dimension which are not associative are Cayley algebras.
(Actually the following stronger result is known: any simple alternative
ring, which is not a nilring and which is not associative, is a Cayley
algebra over its center; in the finite-dimensional case the restriction
eliminating nilalgebras is not required since \hyperlink{Theorem:4}{Theorem} 4 implies that $\A^2 \ne\A$
for a finite-dimensional alternative nilalgebra). Hence the simple components
$\Ss_i$ in a finite-dimensional semisimple alternative algebra are either
associative or Cayley algebras over their centers.

The derivation algebra $\D(\C)$ of any Cayley algebra of characteristic
$\ne 3$ is a central simple Lie algebra of dimension $14$, called an \emph{exceptional
Lie algebra of type G} (corresponding to the $14$-parameter complex exceptional
simple Lie group $G_2$). The related subject of automorphisms of Cayley
algebras is studied in \cite{Ref33}.
\PG--File: 036.png---\************\***\****\********\----------------------

\chapter{Jordan Algebras}% IV.

In the \hyperlink{chapter.1}{Introduction} we defined a (commutative) Jordan algebra $\J$ over
$F$ to be a commutative algebra in which the \emph{Jordan identity}
\begin{myalign}
\tag{1}       &&(xy)x^2 = x(yx^2)       &for all $x, y$ in $\J$
\end{myalign}
is satisfied. Linearization of \tagref(1) requires that we assume $F$ has
characteristic $\ne 2$; we make this assumption throughout IV\@. It follows from
\tagref(1) and the identities \tagref(2), \tagref(3) below that any scalar extension $\J_K$ of a
Jordan algebra $\J$ is a Jordan algebra.

Replacing $x$ in
\begin{myalign}
\tag{1$'$}    &&(x, y, x^2) = 0         &for all $x, y$ in $\J$
\intertext{by $x + \lambda z$ ($\lambda \in F$), the coefficient of $\lambda$ is $0$ since $F$ contains at least three
distinct elements, and we have}
\tag{2}       &&2(x, y, zx) + (z, y, x^2) = 0     &for all $x, y, z$ in $\J$.
\end{myalign}
Replacing $x$ in \tagref(2) by $x + \lambda w$ ($\lambda \in F$), we have similarly (after dividing by
$2$) the multilinear identity
\begin{myalign}[0em](0em)
\tag{3}      &&(x, y, wz) + (w, y, zx) + (z, y, xw) = 0  &for all $w, x, y, z$ in $\J$.\\
\intertext{Recalling that $L_a = R_a$ since $\J$ is commutative, we see that \tagref(3) is equivalent to}
\tag{3$'$}&&[R_x, R_{wz}] + [R_w, R_{zx}] + [R_z, R_{xw}] = 0  &for all $w, x, z$ in $\J$\\
\Intertext{and to}
\tag{3$''$} &&R_zR_{xy} - R_zR_yR_x + R_yR_{zx} - R_{y(zx)} + R_xR_{zy} - R_xR_yR_z = 0\\
& &&for all $x, y, z$ in $\J$.\\
\intertext{Interchange $x$ and $y$ in \tagref(3$''$) and subtract to obtain}
\tag{4}    &&\left[R_z, [R_x, R_y]\right] = R_{(x, z, y)} = R_{z[R_x, R_y]}   &for all $x, y, z$ in $\J$.\\
\end{myalign}
Now \tagref(4) says that, for all $x, y$ in $\J$, the operator $[R_x, R_y]$ is a derivation of $\J$,
since the defining condition for a derivation $D$ of an arbitrary algebra $\A$ may
be written as
\begin{myalign}
&&[R_z, D] = R_{zD}           &for all $z$ in $\A$.
\end{myalign}

\PG--File: 037.png---\************\********\********\*******\--------------
Our first objective is to prove that any Jordan algebra $\J$ is power-associative.
As in \chaplink{3} we define powers of $x$ by $x^1 = x$, $x^{i+1} = xx^i$,
and prove
\begin{myalign}
\tag{5} &&  x^ix^j = x^{i+j}      &for all $x$ in $\J$.
\end{myalign}
For any $x$ in $\J$, write $\G_x = R_x \cup R_{x^2}$. Then the enveloping algebra $\G_x^*$ is
commutative, since the generators $R_x$, $R_{x^2}$ commute by \tagref(1). For $i \ge 2$, we
put $y = x$, $z = x^{i-1}$ in \tagref(3$''$) to obtain
\begin{myalign}
\tag{6}&&  R_{x^{i+1}} = R_{x^{i-1}}R_{x^2} - R_{x^{i-1}}R_x^2 - R_x^2R_{x^{i-1}} + 2R_xR_{x^i}.
\end{myalign}
By induction on $i$ we see from \tagref(6) that $R_{x^i}$ is in $\G^*$ for $i = 3, 4,\dots$. Hence
\begin{myalign}
\tag{7} &&  R_{x^i}R_{x^j} = R_{x^j}R_{x^i}    &for $i, j = 1, 2, 3,\dots$
\end{myalign}
Then, in a proof of \tagref(5) by induction on $i$, we can assume that $x^ix^{j+1} =
x^{i+j+1}$; then $x^{i+1}x^j = (xx^i)x^j = xR_{x^i}R_{x^j} = xR_{x^j}R_{x^i} = x^{j+1}x^i = x^{i+j+1}$
as desired.

One can prove, by a method similar to the proof of \hyperlink{Theorem:4}{Theorem~4} in \chaplink{3}
(only considerably more complicated since the identities involved are more
complicated), that any finite-dimensional Jordan nilalgebra is nilpotent.
We omit the proof, which involves also a proof of the fact that
\begin{myalign}
\tag{8}&& \text{$R_x$ is nilpotent for any nilpotent $x$}\\
&&&in a finite-dimensional Jordan algebra.\\
\end{myalign}
As in \chaplink{3}, this means that there is a unique maximal nilpotent ($=$~solvable $=$~nil)
ideal $\N$ which is called the \emph{radical} of $\J$. Defining $\J$ to be \emph{semisimple} in case
$\N = 0$, we have seen that $\J/\N$ is semisimple. The proof that any semisimple
Jordan algebra $\Ss$ is a direct sum $\Ss = \Ss_1 \oplus \dotsb \oplus \Ss_t$ of simple $\Ss_i$ is quite
complicated for arbitrary $F$; we shall use a trace argument to give a proof
for $F$ of characteristic $0$.

Let $e$ be an idempotent in a Jordan algebra $\J$. Put $i = 2$ and $x = e$ in
\tagref(6) to obtain
\begin{myalign}
\tag{9} &&  2R_e^3 - 3R_e^2 + R_e = 0;
\end{myalign}
\PG--File: 038.png---\*********\********\********\*******\-----------------
that is, $f(R_e) = 0$ where $f(\lambda) = (\lambda-1)(2\lambda-1)\lambda$. Hence the minimal polynomial
for $R_e$ divides $f(\lambda)$, and the only possibilities for characteristic roots of
$R_e$ are $1$, $\frac12$, $0$ ($1$ must occur since $e$ is a characteristic vector belonging to
the characteristic root $1$: $eR_e = e^2 = e \ne 0$). Also the minimal polynomial
for $R_e$ has simple roots. Hence $\J$ is the vector space direct sum
\begin{myalign}
\tag{10}  &\J &=  \J_1 + \J_{1/2} + \J_0\\
\Intertext{where}
\tag{11} & \J_i&= \left\{ x_i \mid x_ie = ix_i \right\},    &   $i = 1, 1/2, 0$.
\end{myalign}
Taking a basis for $\J$ adapted to the \emph{Peirce decomposition} \tagref(10), we see that
$R_e$ has for its matrix relative to this basis the diagonal matrix
$\diag \{1,1, \dots, 1,1/2, 1/2, \dots, 1/2, 0, 0, \dots, 0\}$ where the number of $1$'s
is $\dim\J_1 > 0$ and the number of $1/2$'s is $\dim\J_{1/2}$. Hence
\begin{myalign}
\tag{12} &&  \trace R_e = \dim\J_1 + \tfrac12 \dim\J_{1/2}.
\end{myalign}
If $F$ has characteristic $0$, then $\trace R_e \ne 0$.

A symmetric bilinear form $(x,y)$ defined on an arbitrary algebra $\A$ is
called a \emph{trace form} (\emph{associative} or \emph{invariant} symmetric bilinear form) on $\A$ in
case
\begin{myalign}
\tag{13} &&  (xy,z) = (x,yz)   &for all $x,y,z$ in $\A$.
\end{myalign}
If $\I$ is any ideal of an algebra $\A$ on which such a bilinear form is defined,
then $\I^\perp$ is also an ideal of $\A$: for $x$ in $\I$, $y$ in $\I^\perp$, $a$ in $\A$ imply that $xa$ and
$ax$ are in $\I$ so that $(x,ay) = (xa,y) = 0$ and $(x,ya) = (ya,x) = (y,ax) = 0$ by
\tagref(13). In particular, the radical $\A^\perp =\{x \mid (x,y)=0 \text{ for all }y \in \A\}$ of
the trace form is an ideal of $\A$.

We also remark that it follows from \tagref(13) that $(xR_y,z) = (x,zL_y)$ and
$(xL_y,z) = (z,yx) = (zy,x) = (x,zR_y)$ so that, for right (or left)
multiplications $S_i$ determined by $b_i$,
\begin{myalign}
\tag{14} &&  (xS_1S_2 \dotsm S_h, y) = (x,yS_h'\dotsm S_2'S_1')
\end{myalign}
\PG--File: 039.png---\************\********\*****\********\----------------
where $S'_i$ is the left (or right) multiplication determined by $b_i$; then, if $\B$
is any subset of $\A$,
\begin{myalign}
\tag{15}
&&(xT, y) = (x, yT')
&for all $x, y$ in $\A$, $T$ in $\B^*$,\\
\end{myalign}
where $T'$ is in $\B^*$.

\begin{theorem}[6]
The radical $\N$ of any finite-dimensional Jordan algebra $\J$
over $F$ of characteristic $0$ is the radical $\J^\perp$ of the trace form
\begin{myalign}
\tag{16}
&&(x,y) = \trace R_{xy}
&for all $x, y$ in $\J$.
\end{myalign}
\end{theorem}

\begin{proof}
Without any assumption on the characteristic of $F$ it follows
from \tagref(4) that $(x,y)$ in \tagref(16) is a trace form: $(xy,z) - (x,yz) = \trace R_{(x,y,z)} = 0$
since the trace of any commutator is $0$. Hence $\J^\perp$ is an ideal of $\J$. If $\J$
were not a nilideal, then (by \hyperlink{Proposition:3}{Proposition~3}) $\J^\perp$ would contain an idempotent
$e$ ($\ne 0$) and, assuming characteristic $0$, $(e, e) = \trace R_e \ne 0$ by \tagref(12), a
contradiction. Hence $\J^\perp$ is a nilideal and $\J^\perp \subseteq\N$. Conversely, if $x$ is in $\N$,
then $xy$ is in $\N$ for every $y$ in $\A$, and $R_{xy}$ is nilpotent by~(8). Hence $(x, y) =
\trace R_{xy} = 0$ for all $y$ in $\A$; that is, $x$ is in $\J^\perp$. Hence $\N\subseteq\J^\perp$, $\N = \J^\perp$.
\end{proof}

\begin{theorem}[7]
Let $\A$ be a finite-dimensional algebra over $F$ (of arbitrary
characteristic) satisfying
\begin{Itemize}
\item[i\DPanchor{thm7:i}] there is a nondegenerate (associative) trace form $(x,y)$ defined on
$\A$, and
\item[ii\DPanchor{thm7:ii}] $\I^2 \ne 0$ for every ideal $\I \ne 0$ of $\A$.
\end{Itemize}
Then $\A$ is (uniquely) expressible as a direct sum $\A = \Ss_1 \oplus \dotsb \oplus \Ss_t$ of simple
ideals $\Ss_i$.
\end{theorem}

\begin{proof}
Let $\Ss$ ($\ne 0$) be a minimal ideal of $\A$. Since $(x,y)$ is a trace
form, $\Ss^\perp$ is an ideal of $\A$. Hence the intersection $\Ss\cap \Ss^\perp$ is either $0$ or
$\Ss$, since $\Ss$ is minimal. We show that $\Ss$ totally isotropic ($\Ss\subseteq \Ss^\perp$) leads to
a contradiction.
\PG--File: 040.png---\************\********\****\********\-----------------

For, since $\Ss^2 \ne 0$ by (\hyperlink{thm7:ii}{ii}), we know that the ideal of $\A$ generated by
$\Ss^2$ must be the minimal ideal $\Ss$. Thus $\Ss = \Ss^2 + \Ss^2\M$ where $\M$ is the multiplication
algebra of $\A$. Any element $s$ in $\Ss$ may be written in the form $s = \sum(a_ib_i)T_i$
for $a_i, b_i$ in $\Ss$, where $T_i = T'_i$ is the identity operator $1_\A$ or $T_i$ is in $\M$.
For every $y$ in $\A$ we have by \tagref(15) that $(s, y) = \sum \left((a_ib_i)T_i, y \right) = \sum (a_ib_i, yT'_i) =
\sum \left(a_i, b_i(yT'_i) \right) = 0$ since $b_i(yT'_i) \in \Ss \subseteq \Ss^\perp$. Then $s = 0$ since $(x, y)$ is
nondegenerate; $\Ss = 0$, a contradiction. Hence $\Ss \cap \Ss^\perp = 0$; that is, $\Ss$ is
non-isotropic. Hence $\A = \Ss \perp \Ss^\perp$ and $\Ss^\perp$ is non-isotropic. That is, $\A =
\Ss \oplus \Ss^\perp$, the direct sum of ideals $\Ss$, $\Ss^\perp$, and the restriction of $(x, y)$ to $\Ss^\perp$
is a nondegenerate (associative) trace form defined on $\Ss^\perp$. That is, (\hyperlink{thm7:i}{i})
holds for $\Ss^\perp$ as well as $\A$. Moreover, any ideal of the direct summand $\Ss$ or
$\Ss^\perp$ is an ideal of $\A$; hence $\Ss$ is simple and (\hyperlink{thm7:ii}{ii}) holds for $\Ss^\perp$. Induction on
the dimension of $\A$ completes the proof.
\end{proof}

\begin{corollary} Any (finite-dimensional) semisimple Jordan algebra $\J$ over\DPanchor{cor:thm:7}
$F$ of characteristic $0$ is (uniquely) expressible as a direct sum $\J = \Ss_1 \oplus \dotsb \oplus \Ss_t$
of simple ideals $\Ss_i$.
\end{corollary}

\begin{proof} By \hyperlink{Theorem:6}{Theorem 6} the (associative) trace form \tagref(16) is nondegenerate;
hence (\hyperlink{thm7:i}{i}) is satisfied. Also any ideal $\I$ such that $\I^2 = 0$ is nilpotent;
hence $\I = 0$, establishing (\hyperlink{thm7:ii}{ii}).
\end{proof}

As mentioned above, the corollary is actually true for $F$ of characteristic
$\ne 2$. What remains then, as far as the structure of semisimple Jordan algebras
is concerned, is a determination of the central simple algebras. The first
step in this is to show that every semisimple $\J$ (hence every simple $\J$) has
a unity element $1$. Again the argument we use here is valid only for
characteristic $0$, whereas the theorem is true in general.

We begin by returning to the Peirce decomposition \tagref(10) of any Jordan
algebra $\J$ relative to an idempotent $e$. The subspaces $\J_1$ and $\J_0$ are orthogonal
\PG--File: 041.png---\************\********\****\********\-----------------
subalgebras of $\J$ which are related to the subspace $\J_{1/2}$ as follows:
\begin{myalign}
\tag{17}  &&  \J_{1/2}\J_{1/2} \subseteq \J_1 + \J_0,\qquad \J_1\J_{1/2} \subseteq \J_{1/2},\qquad \J_0\J_{1/2} \subseteq \J_{1/2}.
\end{myalign}
For we may put $x = e$, $z = x_i \in \J_i$, $y = y_j \in \J_j$ in \tagref(2) to obtain $2i(e, y_j, x_i) +
(x_i, y_j, e) = 0$, or $(1 - 2i)\left[(x_iy_j)e - j(x_iy_j)\right] = 0$, so that
\begin{myalign}
\tag{18}     &&\J_i\J_j \subseteq \J_j   &if $i \ne 1/2$.
\end{myalign}
Hence ${\J_1}^2 \subseteq \J_1$, ${\J_0}^2 \subseteq \J_0$, $\J_1\J_0 = \J_0\J_1 \subseteq
\J_0 \cap \J_1 = 0$, so $\J_1$ and $\J_0$ are orthogonal
subalgebras by \tagref(18), and also the last two inclusions in \tagref(17) hold. Put $x =
x_{1/2}$, $z = y_{1/2}$, $y = w = e$ in \tagref(3) and write $x_{1/2}y_{1/2} = a = a_1 + a_{1/2} + a_0$ to
obtain $\frac{1}{2}(x_{1/2}, e, y_{1/2}) + (e, e, a) + \frac{1}{2}(y_{1/2}, e, x_{1/2}) = (e, e, a) = 0$. Hence
$ea - e(ea) = a_1 + \frac{1}{2}a_{1/2} - e(a_1 + \frac{1}{2}a_{1/2}) = a_1 + \frac{1}{2}a_{1/2} - a_1 - \frac{1}{4}a_{1/2} = \frac{1}{4}a_{1/2} = 0$.
Hence $x_{1/2}y_{1/2} = a_1 + a_0 \in \J_1 + \J_0$, establishing \tagref(17).

Now
\begin{myalign}
\tag{19}     &&\trace R_b = 0       &for all $b \in \J_{1/2}$.
\end{myalign}
For $b$ in $\J_{1/2}$ implies $\trace R_b = 2\trace R_{eb} = 2(e, b) = 2(e^2, b) = 2(e, eb) =
(e, b)$ by \tagref(16) and \tagref(13), so $\trace R_b = (e, b) = 0$. Writing $x = x_1 + x_{1/2} + x_0$,
$y = y_1 + y_{1/2} + y_0$ in accordance with \tagref(10), we have $xy = (x_1y_1 + x_{1/2}y_{1/2} +
x_0y_0) + (x_1y_{1/2} + x_{1/2}y_1 + x_{1/2}y_0 + x_0y_{1/2})$ with the last term in parentheses
in $\J_{1/2}$ by \tagref(17). Hence \tagref(19) implies that
\begin{myalign}
\tag{20}   &&   (x, y) = \trace R_{x_1y_1 + x_{1/2}y_{1/2} + x_0y_0}.
\end{myalign}
Now $x_{1/2}y_{1/2} = c = c_1 + c_0$ ($c_i$ in $\J_i$) implies $\trace R_{c_1} + \trace R_{c_0} =
\trace R_c = (x_{1/2}, y_{1/2}) = 2(ex_{1/2}, y_{1/2}) = 2(e, x_{1/2}y_{1/2}) =
2 \trace R_{e(c_1 + c_0)} \break % another heavyhanded solution
= 2 \trace R_{c_1}$, so that $\trace R_{c_1} = \trace R_{c_0}$. Then \tagref(20)
may be written as
\begin{myalign}
\tag{20$'$} &&      (x, y) = \trace R_{x_1y_1 + z_0}, &$z_0 = 2c_0 + x_0y_0$ in $\J_0$.
\end{myalign}

In any algebra $\A$ over $F$ an idempotent $e$ is called \emph{principal} in case
there is no idempotent $u$ in $\A$ which is orthogonal to $e$ ($u^2 = u \ne 0$, $ue = eu = 0$);
\PG--File: 042.png---\************\********\******\********\---------------
that is, there is no idempotent $u$ in the subspace $\A_0 =
\{x_0 \mid x_0 \in \A,\ ex_0 = x_0e = 0\}$. In a finite-dimensional Jordan algebra $\J$, this
means that $e$ is a principal idempotent of $\J$ if and only if the subalgebra
$\J_0$ (in the Peirce decomposition \tagref(10) relative to $e$) is a nilalgebra.

Now any finite-dimensional Jordan algebra $\J$ which is not a nilalgebra
contains a principal idempotent. For $\J$ contains an idempotent $e$ by
\hyperlink{Proposition:3}{Proposition~3}. If $e$ is not principal, there is an idempotent $u$ in $\J_0$,
$e' = e + u$ is idempotent, and $\J_{1, e'}$ (the $\J_1$ relative to $e'$) contains
properly $\J_{1, e} = \J_1$. For $x_1$ in $\J_{1, e}$ implies $x_1e' = x_1(e + u) = x_1e + x_1u =
x_1$, or $x_1$ is in $\J_{1, e'}$. That is, $\J_{1, e} \subseteq \J_{1, e'}$. But $u \in \J_{1, e'}$, $u \notin \J_{1, e}$.
Then $\dim\J_{1, e} < \dim\J_{1, e'}$, and this process of increasing dimensions must
terminate, yielding a principal idempotent.

\begin{theorem}[8]
Any semisimple (hence any simple) Jordan algebra $\J$ of finite
dimension over $F$ of characteristic $0$ has a unity element~$1$.
\end{theorem}

\begin{proof}
$\J$ has a principal idempotent $e$. Then $\J_0$ is a nilalgebra, so
that $(x, y) = \trace R_{x_1y_1}$ by \tagref(20$'$) since $\trace R_{z_0} = 0$ by \tagref(8). Hence $x$ in
$\J_{1/2} + \J_0$ implies $x_1 = 0$, $(x, y) = 0$ for all $y$ in $\J$, so $x$ is in $\J^\perp$.
That is, $\J_{1/2} + \J_0 \subseteq \J^\perp = \N = 0$, or $\J = \J_1$, $e = 1$.
\end{proof}

If $\J$ contains $1$ and $e_1 \ne 1$, then $e_2 = 1 - e_1$, is an idempotent, and the
Peirce decompositions relative to $e_1$ and $e_2$ coincide (with differing subscripts).
We introduce a new notation: $\J_{11} = \J_{1, e_1}$ ($= \J_{0, e_2}$), $\J_{12} = \J_{1/2, e_1}$
($= \J_{1/2, e_2}$), $\J_{22} = \J_{0, e_1}$ ($= \J_{1, e_2}$). More generally, if $1 = e_1 + e_2 + \dotsb +
e_t$ for pairwise orthogonal idempotents $e_i$, we have the refined Peirce
decomposition
\begin{myalign}
\tag{21}&&
  \J = \sum_{i \le j} \J_{ij}
\end{myalign}
of $\J$ as the vector space direct sum of subspaces $\J_{ii} = \J_{1, e_i}$ ($1 \le i \le t$),
\PG--File: 043.png---\************\*****\****\********\--------------------
$\J_{ij} = \J_{1/2, e_i} \cap \J_{1/2, e_j}$ ($1 \le i < j \le t$); that is,
\begin{myalign}
\tag{22}&&{\begin{aligned}
&\J_{ii} = \{x \mid x \in \J,\ xe_i = x\},\\
&\J_{ij} = \J_{ji} = \{x \mid x \in \J,\ xe_i = \tfrac{1}{2}x = xe_j\},\quad i \ne j.
\end{aligned}}
\end{myalign}
Multiplicative relationships among the $\J_{ij}$ are consequences of \tagref(17) and
the statement preceding it.

An idempotent $e$ in $\J$ is called \emph{primitive} in case $e$ is the only
idempotent in $\J_1$ (that is, $e$ cannot be written as the sum $e = u + v$ of
orthogonal idempotents), and \emph{absolutely primitive} in case it is primitive
in any scalar extension $\J_K$ of $\J$. A central simple Jordan algebra $\J$ is
called \emph{reduced} in case $1 = e_1 + \dotsb + e_t$ for pairwise orthogonal absolutely
primitive idempotents $e_i$ in $\J$. In this case it can be shown that the
subalgebras $\J_{ii}$ in the Peirce decomposition \tagref(22) are $1$-dimensional
($\J_{ii} = Fe_i$) and that the subspaces $\J_{ij}$ ($i \ne j$) all have the same dimension.
If $\J$ is a central simple algebra over $F$, there is a scalar extension $\J_K$
which is reduced (for example, take $K$ to be the algebraic closure of $F$),
and it can be shown that the number $t$ of pairwise orthogonal absolutely
primitive idempotents $e_i$ in $\J_K$ such that $1 = e_1 + \dotsb + e_t$ is unique; $t$ is
called the \emph{degree} of $\J$.

We list without proof all (finite-dimensional) central simple Jordan
algebras $\J$ of degree $t$ over $F$ of characteristic $\ne 2$. Recall from the
\hyperlink{chapter.1}{Introduction} that $\J$ is a \emph{special Jordan algebra} in case $\J$ is isomorphic to
a subalgebra of an algebra $\A^+$ where $\A$ is associative and multiplication in
$\A^+$ is defined by
\begin{myalign}
\tag{23} &&  x \dotm y = \tfrac{1}{2}(xy + yx).
\end{myalign}
We say that each algebra is of \emph{type} $\type A$, $\type B$, $\type C$, $\type D$, or $\type E$ listed below.

\begin{IItemize}
\item[$\type A_\text{I}$.]  $\J \cong \A^+$  with $\A$ any central simple associative algebra (necessarily
of dimension $t^2$ over $F$).
\PG--File: 044.png---\************\***\****\********\----------------------

\item[$\type A_\text{II}$.] Let $\A$ be any involutorial simple associative algebra over $F$, the
involution being of the second kind (so that the center $\Z$ of $\A$ is a quadratic
extension of $F$ and the involution induces a non-trivial automorphism on $\Z$
(Albert, Structure of Algebras, p.~153)). Then $\J \cong \HH(\A)$, the $t^2$-dimensional
subalgebra of self-adjoint elements in the $2t^2$-dimensional algebra $\A^+$. If
$\J$ is of type $A_\text{I}$ or $A_\text{II}$, and if $K$ is the algebraic closure of $F$, then
$\J_K \cong K_t^+$ where $K_t$ is the algebra of all $t \times t$ matrices with elements in $K$.

\item[$\type B$, $\type C$.] Let $\A$ be any involutorial central simple associative algebra over
$F$ (so the involution is of the first kind). Then $\J \cong \HH(\A)$, the subalgebra
of self-adjoint elements in $\A^+$. There are two types ($\type B$ and $\type C$) which may be
distinguished by passing to the algebraic closure $K$ of $F$, so that $\A_K$ is a
total matrix algebra. In case $\type B$ the (extended) involution on $\A_K$ is
transposition ($a \to a'$) so that $\A$ has dimension $t^2$ and $\J$ has dimension
$\frac{1}{2}t(t+1)$ over $F$. In case $\type C$ the (extended) involution on $\A_K$ is $a \to g^{-1}a'g$
where $g =\begin{pmatrix}
0 &1_t\\
-1_t &0
\end{pmatrix}$ so that $\A$ has dimension $4t^2$ and $\J$ has dimension $2t^2 - t$
over $F$.

\item[$\type D$.] Let $(x, y)$ be any nondegenerate symmetric bilinear form on a vector
space $\M$ of dimension $n \ge 2$. Then $\J$ is the vector space direct sum $\J = F1 + \M$,
multiplication in the $(n+1)$-dimensional algebra $\J$ being defined by $xy = (x,y)1$
for all $x, y$ in $\M$. Here $t = 2$ ($\dim J \ge 3$).

\item[$\type E$.] The algebra $\C_3$ of all $3 \times 3$ matrices with elements in a Cayley algebra
$\C$ over $F$ has the \emph{standard involution} $x \to\overline{x'}$ (conjugate transpose). The
$27$-dimensional subspace $\HH(\C_3)$ of self-adjoint elements
\begin{myalign}
\tag{24} && {\begin{pmatrix}
            \xi_1  &c       &\overline b\\
            \overline c  &\xi_2  &a\\
             b      &\overline a  &\xi_3
           \end{pmatrix}}, &$\xi_i$ in $F$, $a, b, c$ in $\C$,
\end{myalign}
\PG--File: 045.png---\************\********\********\*******\--------------
is a (central simple) Jordan algebra of degree $t = 3$ under the multiplication
\tagref(23) where $xy$ is the multiplication in $\C_3$ (which is not associative). Then
$\J$ is any algebra such that some scalar extension $\J_K \cong \HH(\C_3)_K$ ($= \HH((\C_K)_3)$). % missing closing ) added
\end{IItemize}

The possibility of additional algebras of degree $1$, mentioned in the
1955 Bulletin \hyperlink{cite.Ref64}{article}, has been eliminated in reference \cite{Ref32} of the
bibliography of more recent papers. The proof involves use of a two-variable
identity which is easily seen to be true for special Jordan algebras. But
any such identity is then true for arbitrary Jordan algebras since it has
been proved that the free Jordan algebra with two generators is special
\cite{Ref71, Ref38}, a result which is false for three generators \cite{Ref9}. The identity
in question is
\begin{myalign}
\tag{25} &&   \{aba\}^2 = \left\{a\{ba^2b\}a\right\}  &for all $a, b$ in $\J$,
\end{myalign}
where $\{abc\}$ is defined in a Jordan algebra $\J$ by $\allowbreaks\{abc\} = (ab)c + (bc) a -
(ac)b$, so that $\{aba\} = b(2{R_a}^2 - R_{a^2})$. Hence in $\A^+$ ($\A$ associative) we have $\{aba\} =
2(b\dotm a)\dotm a - b\dotm a^2 = aba$, so that $\{aba\}^2 = aba^2ba = \left\{a\{ba^2b\}a\right\}$. Then \tagref(25) is
satisfied in any special Jordan algebra (in particular, the free Jordan
algebra with two generators) and thus in any Jordan algebra.

Therefore all (finite-dimensional) separable Jordan algebras are known,
and the Wedderburn decomposition theorem stated in the 1955 Bulletin \hyperlink{cite.Ref64}{article}
is valid without restriction. Some of the computations employed in the
original proof may be eliminated \cite{Ref79}.

A central simple Jordan algebra of degree $2$ (that is, of type $\type D$) is a
commutative quadratic algebra with $1$ ($a^2 - t(a)a + n(a)1 = 0$) having
nondegenerate norm form $n(a)$, and conversely. For $a = \alpha 1 + x$, $x \in\M$, implies
$a^2 - t(a)a + n(a)1 = 0$ where $t(a) = 2\alpha$, $n(a) = \alpha^2 - (x, x)$, and $n(a)$ is
nondegenerate if and only if $(x, y)$ is.

The algebras of types $\type A$, $\type B$, $\type C$ are special Jordan algebras by definition.
An algebra of type $\type D$ is a subalgebra of $\A^+$, where $\A$ is the (associative)
\PG--File: 046.png---\************\*****\********\*******\-----------------
Clifford algebra of $(x, y)$ (Artin, \textit{Geometric Algebra}, p.~186). But algebras
of type $\type E$ are not special (as we show below), and are therefore called
\emph{exceptional} central simple Jordan algebras. Exceptional Jordan division
algebras exist (over suitable fields $F$; but not, for example, over a finite
field or the field of all real numbers)~\cite{Ref2}. If an exceptional central
simple Jordan algebra $\J$ is not a division algebra, then it is reduced, and
$\J$ is isomorphic to an algebra $\HH(\C_3, \Gamma)$ of self-adjoint elements in $\C_3$ under
a \emph{canonical involution} $x \to \Gamma^{-1}\overline{x'}\Gamma$ where $\Gamma = \diag\{\gamma_1, \gamma_2, \gamma_3\}$, $\gamma_i \ne 0$ in $F$.
Isomorphism of reduced exceptional simple Jordan algebras is studied in~\cite{Ref8}.

The unifying feature in the list of central simple Jordan algebras
above is that, for $t > 2$, a reduced central simple Jordan algebra is
isomorphic to the algebra $\HH(\D_t, \Gamma)$ defined as follows: $\D$ is an alternative
algebra (of dimension $1$, $2$, $4$, or $8$) with unity element $u$ and involution
$d \to\overline d$ satisfying $d + \overline d \in Fu$, $d\overline d = n(d)u$, $n(d)$ nondegenerate on $\D$; $\D_t$ is the
algebra of all $t \times t$ matrices with elements in $\D$; $\Gamma = \diag\{\gamma_1, \gamma_2, \dots, \gamma_t\}$,
$\gamma_i \ne 0$ in $F$. Then $x \to \Gamma^{-1}\overline{x'}\Gamma$ is a canonical involution in $\D_t$, and the
set $\HH(\D_t, \Gamma)$ of all self-adjoint elements in $\D_t$ is a subalgebra of $\D_t^+$ (that
is, we do not need $\A$ associative to define $\A^+$ by \tagref(23)). If $\D$ is associative,
then $\D_t = \D \otimes_F F_t$ is associative, and $\J \cong \HH(\D_t, \Gamma)$ is a special Jordan
algebra. If $\D$ is not associative, then $\J \cong \HH(\D_t, \Gamma)$ is not a Jordan algebra
unless $t = 3$. Hence we have $\J$ of type $\type B$ if $\D = F1$; $\J$ of type $\type A$ if $\D = \Z$
(type $\type A_\text{I}$ if $\Z = F \oplus F$; type $\type A_\text{II}$ if $\Z$ is a quadratic field over $F$); $\J$ of type
$\type C$ if $\D = \Q$; $\J$ of type $\type E$ if $t = 3$ and $\D = \C$. The corresponding dimensions for
$\J$ are clearly $t + \frac12t(t - 1)(\dim\D)$; that is, $\frac12t(t + 1)$ for type $\type B$, $t^2$ for
type $\type A$, $2t^2 - t$ for type $\type C$, and $27$ for type $\type E$.

\begin{theorem}[9] Any central simple Jordan algebra $\J$ of type $\type E$ is exceptional
(that is, is not a special Jordan algebra).
\PG--File: 047.png---\*************\*****\********\*******\----------------
\end{theorem}

\begin{proof} It is sufficient to prove that $\HH(\C_3)$ is not special. For, if
$\J$ were special, then $\J \cong \J'\subseteq \A^+$ with $\A$ associative implies $\J_K = K\otimes \J \cong K\otimes \J'\subseteq
K\otimes\A^+ = (K\otimes\A)^+ = {\A_K}^+$ so that $\HH((\C_K)_3)\cong\J_K$ is special, a contradiction.

Suppose that $\HH(\C_3)$ is special. There is an associative algebra $\A$ (of
possibly infinite dimension over $F$) such that $U$ is an isomorphism of $\HH(\C_3)$
into $\A^+$. For $i = 1,2,3$ define elements $e_i$ in $\A$ and $8$-dimensional subspaces
\[
\Ss_i=\{d_i\mid d\in \C\}
\]
of $\A$ by
\begin{myalign}
\tag{26} &&xU = \xi_1e_1 + \xi_2e_2 + \xi_3e_3 +a_1 + b_2 + c_3
\intertext{for $x$ in \tagref(24); that is, for $\xi_i$ in $F$ and $a, b, c$ in $\C$. (Note that our
notation is such that we will never use $e$ for an element of $\C$). Then}
\tag{27} && \Ss = Fe_1 + Fe_2 + Fe_3 + \Ss_1 + \Ss_2 + \Ss_3 = \HH(\C_3)U
\intertext{is a $27$-dimensional subspace of $\A$. $\Ss$ is a subalgebra of $\A^+$. The mapping
$V=U^{-1}$ defined on $\Ss$ (not on all of $\A$) is an isomorphism of $\Ss$ onto $\HH(\C_3)$:}
\tag{28} && (xU\dotm yU)V = x\dotm y &for all $x,y$ in $\HH(\C_3)$.
\intertext{For our proof we do not need all of the products in $\A^+$ of elements of $\Ss$.
However, performing the multiplications in $\HH(\C_3)$, we see that \tagref(28) yields}
\tag{29} &&{e_i}^2 = e_i~(\ne0),&$i=1,2,3$;\\
\tag{30} &&e_i\dotm e_j = 0, &$i\ne j$;\\
\tag{31} &&e_i\dotm a_i = 0, &$a$ in $\C$, $i=1,2,3$;\\
\tag{32} &&e_i\dotm a_j = \tfrac12 a_j, &$a$ in $\C$, $i\ne j$;\\
\tag{33} &&{u_i}^2 = e_j+e_k, &$i,j,k$ distinct,
\end{myalign}
where $u$ is the unity element in $\C$; and
\begin{myalign}
\tag{34}&& 2a_i\dotm b_j=(\overline b\,\overline{\vphantom{b}a})_k,\\ % line broken: too long
&&&$a,b$ in $\C$, $i,j,k$ a cyclic permutation of $1,2,3$.\\
\end{myalign}
Now \tagref(29) and \tagref(30) imply that $e_i$ ($i=1,2,3$) are pairwise orthogonal idempotents.
For $\A$ is associative, so $e_ie_j+e_je_i=0$ for $i\ne j$ implies ${e_i}^2e_j + e_ie_je_i=
0 = e_ie_je_i + e_j{e_i}^2$, or $e_ie_j=e_je_i$; hence $e_ie_j=0$ for $i\ne j$. By an
identical proof it follows from \tagref(31) that
\PG--File: 048.png---\********\******\****\********\-----------------------
\begin{myalign}
\tag{31$'$}  &&e_ia_i = a_ie_i = 0, &$i=1,2,3$.
\end{myalign}
For $i, j, k$ distinct, \tagref(32) implies $e_ia_j + a_je_i = a_j = e_ka_j + a_je_k$; then
$fa_j + a_jf = 2a_j$ for the idempotent $f = e_i + e_k$. Hence $f^2 a_j + fa_jf = 2fa_j$,
so $fa_jf = fa_j$ and similarly $fa_jf = a_jf$; that is, $fa_j = a_jf = a_j$:
\begin{myalign}
\tag{35}    &&(e_i + e_k)a_j = a_j = a_j(e_i + e_k), &$i, j, k$ distinct.
\intertext{Also \tagref(32) implies $e_ia_j = a_j - a_je_i$, so $e_ia_je_i = a_je_i - a_j{e_i}^2 = 0$:}
\tag{36}   &&e_ia_je_i = 0,  &$i \ne j$.
\end{myalign}

For any $a$ in $\C$, define
\begin{myalign}
\tag{37}   &&a' = e_1a_3u_3  &in $\A$.
\intertext{Then $(ab)' = e_1(ab)_3u_3 = e_1(\overline{b_1}\,\overline{\vphantom{b}a_2}
 + \overline{\vphantom{b}a_2}\:\overline{b_1})u_3
 = e_1\overline{\vphantom{b}a_2}\:\overline{b_1}u_3$ by \tagref(34) and \tagref(31$'$).
Also \tagref(34) implies $a_3u_1 + u_1a_3 = (\overline{u}\, \overline{a})_2 = \overline{a_2}$ and}
\tag{38}      &&u_2b_3 + b_3u_2 = \overline{b_1}.
\end{myalign}
Hence $(ab)' = e_1(a_3u_1 + u_1a_3)(u_2b_3 + b_3u_2)u_3 = e_1a_3u_1(u_2b_3 + b_3u_2)u_3$ by \tagref(31$'$).
Now $b_3u_2u_3 = b_3u_2(e_1 + e_3)u_3 = b_3u_2e_1u_3 = (\overline{b_1} - u_2b_3)e_1u_3 =
 - u_2b_3e_1u_3 =
-u_2(e_1 + e_3)b_3e_1u_3 = -u_2e_1b_3e_1u_3 = 0$ by \tagref(35), \tagref(31$'$), \tagref(38), and \tagref(36).
Also $\allowbreaks u_1u_2b_3 = u_1u_2(e_1 + e_2)b_3 = u_1u_2e_1b_3 = (u_3 - u_2u_1)e_1b_3 = u_3e_1b_3$ by \tagref(35),
\tagref(31$'$), \tagref(34). Hence $(ab)' = e_1a_3u_1u_2b_3u_3 = e_1a_3u_3e_1b_3u_3 = a'b'$.

Clearly the mapping $a \to a'$ is linear; hence it is a homomorphism of
$\C$ onto the subalgebra $\C'$ of $\A$ consisting of all $a'$. Since $\C$ is simple, the
kernel of this homomorphism is either $0$ or $\C$; in the latter case $0 = u' =
e_1{u_3}^2 = e_1(e_1 + e_2) = e_1 \ne 0$ by \tagref(33), and we have a contradiction. Hence
$a \to  a'$ is an isomorphism. But $\C'$ is associative, whereas $\C$ is not. Hence
$\HH(\C_3)$ is an exceptional Jordan algebra.
\end{proof}

Any central simple exceptional Jordan algebra $\J$ over $F$ is a \emph{cubic algebra}:
for any $x$ in $\J$,
\PG--File: 049.png---\********\*****\****\********\------------------------
\begin{myalign}
\tag{39}  &&x^3 - T(x)x^2 + Q(x)x - N(x)1 = 0,\\ % break not in original
&&&$T(x)$, $Q(x)$, $N(x)$ in $F$.\\
\end{myalign}
Here $x^2x = xx^2$ ($= x^3$) since $\J$ is commutative. It is sufficient to show \tagref(39)
for $\HH(\C_3)$. But $x$ in \tagref(24) implies \tagref(39) where
\begin{myalign}
\tag{40} &&T(x) = \xi_1 + \xi_2 + \xi_3,\\
\tag{41}&&\!{\begin{aligned} % there must be a better way to do this
         Q(x) &= \xi_1\xi_2 + \xi_2\xi_3 + \xi_3\xi_1 - n(a) - n(b) - n(c)\\
      &= \tfrac{1}{2}\left[\left(T(x)\right)^2 - T(x^2)\right],
\end{aligned}}\\
\tag{42} &&N(x) = \xi_1\xi_2\xi_3 - \xi_1n(a) - \xi_2n(b) - \xi_3n(c) + t(abc);
\intertext{if $F$ has characteristic $\ne 3$ (as well as $\ne 2$), formula \tagref(42) may be written
as}
\tag{42$'$}&& N(x) =\tfrac{1}{6}\left[\left(T(x)\right)^3 - 3T(x)T(x^2) + 2T(x^3)\right].
\end{myalign}
The cubic norm form \tagref(42) satisfies $N\left(\{xyx\}\right) = \left[N(x)\right]^2 N(y)$; that is, $N(x)$
permits a new type of composition \cite{Ref35, Ref69}.

In the 1955 Bulletin \hyperlink{cite.Ref64}{article}, only passing mention is made in \S7 of the
relationships between exceptional central simple Jordan algebras $\J$ and
exceptional simple Lie algebras and groups; relationships which stem from
the fact that the derivation algebra $\D(\J)$ is a central simple Lie algebra
of dimension $52$, called an \emph{exceptional Lie algebra of type $\type F$} (corresponding
to the $52$-parameter complex exceptional Lie group $F_4$)---a proof of this
for $F$ of characteristic $\ne 2$ appears in \cite{Ref36} (characteristic $\ne 2,\ 3$ in \cite{Ref70}).
Since 1955 the relationships, including a characterization of Cayley planes
by means of $\HH(\C_3,\Gamma)$, have been vigorously exploited \cite{Ref18, Ref19, Ref20, Ref35, Ref36,
Ref37, Ref39, Ref70, Ref76, Ref78, Ref81, Ref82}.
\PG--File: 050.png---\********\*****\****\********\------------------------



\chapter{Power-associative Algebras} % V.

We recall that an algebra $\A$ over $F$ is called \emph{power-associative} in case
the subalgebra $F[x]$ generated by any element $x$ of $\A$ is associative. We
have seen that this is equivalent to defining, for any $x$ in $\A$,
\begin{myalign}
    &&x^1 = x,\qquad x^{i+1} = xx^i     &for $i = 1,2,3,\ldots$, % NB \dots doesn't work here for some reason
\intertext{and requiring}
\tag{1}   &&x^i x^j = x^{i+j}           &for $i,j = 1,2,3,\ldots$
\end{myalign}
All algebras mentioned in the \hyperlink{chapter.1}{Introduction} are power-associative (Lie
algebras trivially, since $x^2 = 0$ implies $x^i = 0$ for $i = 2,3,\dots$). We shall
encounter in V new examples of power-associative algebras.

The most important tool in the study of noncommutative power-associative
algebras $\A$ is the passage to the commutative algebra $\A^+$. Let $F$ have
characteristic $\ne 2$ throughout V; we shall also require that $F$ contains at
least four distinct elements. The algebra $\A^+$ is the same vector space as $\A$
over $F$, but multiplication in $\A^+$ is defined by
\begin{myalign}
\tag{2}  &&x\dotm y = \tfrac{1}{2} (xy + yx)       &for $x, y$ in $\A$,
\end{myalign}
where $xy$ is the (nonassociative) product in $\A$. If $\A$ is power-associative,
then (as in the \hyperlink{chapter.1}{Introduction}) powers in $\A$ and $\A^+$ coincide, and it follows
that $\A^+$ is a commutative power-associative algebra.

Let $\A$ be power-associative. Then \tagref(2) implies
\begin{myalign}
\tag{3}   &&x^2 x = xx^2              &for all $x$ in $\A$
\Intertext{and}
\tag{4}   &&x^2 x^2 = x(xx^2)         &for all $x$ in $\A$.
\intertext{In terms of associators, we have}
\tag{3$'$}   &&(x,x,x) = 0            &for all $x$ in $\A$
\PGx--File: 051.png---\*******\*****\****\********\-------------------------
\Intertext{and}
\tag{4$'$}   &&(x,x,x^2) = 0         &for all $x$ in $\A$.
\intertext{Also \tagref(3) may be written in terms of a commutator as}
\tag{3$''$}   &&[x^2,x] = 0          &for all $x$ in $\A$.
\intertext{Using the linearization process employed in \chaplink{4}, we obtain from \tagref(3$''$), by
way of the intermediate identity}
\tag{3$'''$}  &&2[x\dotm y,x] + [x^2,y] = 0   &for all $x, y$ in $\A$,
\end{myalign}
the multilinear identity
\begin{myalign}[0em]
\tag{3m}   &&[x\dotm y,z] + [y\dotm z,x] + [z\dotm x,y] = 0   &for all $x, y, z$ in $\A$.\\
\intertext{Similarly, assuming four distinct elements in $F$, \tagref(4) is equivalent to}
\tag{4$''$}   &&2(x,x,x\dotm y) + (x,y,x^2) + (y,x,x^2) = 0 &for all $x, y$ in $\A$,\\
\Intertext{to}
\tag{4$'''$}&&{\begin{aligned}
&2(x,y,x\dotm z) + 2(x,z,x\dotm y) + 2(y,x,x\dotm z) + 2(x,x,y\dotm z)\\
&\quad+2(z,x,x\dotm y) + (y,z,x^2) + (z,y,x^2) = 0
\end{aligned}}\\
&&&for all $x, y, z$ in $\A$,\\
\Intertext{and finally to the multilinear identity}
\tag{4m}&&{\begin{aligned}
&(x,y,z\dotm w) + (z,y,w\dotm x) + (w,y,x\dotm z)\\
        +&(y,x,z\dotm w) + (z,x,w\dotm y) + (w,x,y\dotm z)\\
        +&(z,w,x\dotm y) + (x,w,y\dotm z) + (y,w,z\dotm x)\\
        +&(w,z,x\dotm y) + (x,z,y\dotm w) + (y,z,w\dotm x) = 0\end{aligned}}\\
 &&&for all $x, y, z, w$ in $\A$,\\
\end{myalign}
where in each row of the formula \tagref(4m) we have left one of the four elements
$x$, $y$, $z$, $w$ fixed in the middle position of the associator and permuted the
remaining three cyclically.

We omit the proof of the fact that, if $F$ has characteristic $0$, then
identities \tagref(3) and \tagref(4) are sufficient to insure that an algebra is power-associative;
the proof involves inductions employing the multilinear
identities \tagref(3m) and \tagref(4m). We omit similarly a proof of the fact that, if
$F$ has characteristic $\ne 2,3,5$, the single identity \tagref(4) in a commutative
algebra implies power-associativity. One consequence of this latter fact
\PG--File: 052.png---\*************\*****\***********\********\------------
is that in a number of proofs concerning power-associative algebras separate
consideration has to be given to the characteristic $3$ or $5$ case by bringing
in associativity of fifth or sixth powers and an assumption that $F$ contains
at least 6 distinct elements. We shall omit these details, simply by
assuming characteristic $\ne 3,5$ upon occasion.

An algebra $\A$ over $F$ is called \emph{strictly power-associative} in case every
scalar extension $\A_K$ is power-associative. If $\A$ is a commutative power-associative
algebra over $F$ of characteristic $\ne 2,3,5$, then $\A$ is strictly
power-associative. The assumption of strict power-associativity is
employed in the noncommutative case, and in the commutative case of
characteristic $3$ or $5$, when one wishes to use the method of extension of
the base field.

\ThoughtBreak
Let $\A$ be a finite-dimensional power-associative algebra over $F$.
Just as in the proofs of Propositions \hyperlink{Proposition:1}{1} and \hyperlink{Proposition:2}{2}, one may argue that $\A$ has a
unique maximal nilideal $\N$, and that $\A/\N$ has maximal nilideal $0$. For if
$\A$ is a power-associative algebra which contains a nilideal $\I$ such that
$\A/\I$ is a nilalgebra, then $\A$ is a nilalgebra. [If $x$ is in $\A$, then $\overline{x^s} = \overline{x}^s=0$
for some $s$, so that $x^s = y\in\I$ and $x^{rs}=(x^s)^r=y^r=0$ for some $r$.] Since
any homomorphic image of a nilalgebra is a nilalgebra, it follows from the
second isomorphism theorem that, if $\B$ and $\C$ are nilideals, then so is
$\B+\C$. For $(\B+\C)/\C\cong \B/(\B\cap\C)$ is a nilalgebra, so $\B+\C$ is. This
establishes the uniqueness of $\N$. It follows as in the proof of Proposition
\hyperlink{Proposition:2}{2} that $0$ is the only nilideal of $\A/\N$. $\N$ is called the \emph{nilradical} of $\A$, and
$\A$ is called \emph{semisimple} in case $\N=0$. Of course any anticommutative algebra
(for example, any Lie algebra) is a nilalgebra, and hence is its own
nilradical. Hence this concept of semisimplicity is trivial for anticommutative
algebras.
\PG--File: 053.png---\*************\********\********\*******\-------------

For the moment let $\A$ be a commutative power-associative algebra, and
let $e$ be an idempotent in $\A$. Putting $x=e$ in \tagref(4$''$) and using commutativity,
we have $y(2{R_e}^3-3{R_e}^2+R_e)=0$ for all $y$ in $\A$, or
\begin{myalign}
\tag{5} &&2{R_e}^3 -3{R_e}^2+R_e = 0
\end{myalign}
for any idempotent $e$ in a commutative power-associative algebra $\A$. As we
have seen in the case of Jordan algebras in \chaplink{4}, this gives a Peirce
decomposition
\begin{myalign}
\tag{6} &\A&=\A_1 + \A_{1/2} + \A_0
\intertext{of $\A$ as a vector space direct sum of subspaces $\A_i$ defined by}
\tag{7} &\A_i&=\{x_i \mid x_i e = ix_i\},&$i=1, 1/2, 0$; $\A$ commutative.\\
\end{myalign}
Now if $\A$ is any power-associative algebra, the algebra $\A^+$ is a commutative
power-associative algebra. Hence we have the Peirce decomposition \tagref(6) where
\begin{myalign}
\tag{7$'$}&& \A_i = \{x_i \mid ex_i+x_i e = 2ix_i\},&$i = 1, 1/2, 0$.
\end{myalign}
Put $y = z = e$ in \tagref(3m) and let $x=x_i \in\A_i$ as in \tagref(7$'$) to obtain $(2i-1)[x_i,e]=
0$; that is, $x_i e=ex_i$ if $i\ne 1/2$. Hence \tagref(7$'$) becomes
\begin{myalign}
\tag{7$''$}&&{\begin{aligned} % not in original
 \A_i&=\{x_i \mid ex_i = x_i e = ix_i\},\qquad i = 1,0;\\
 \A_{1/2} &=\{x_{1/2}\mid ex_{1/2}+x_{1/2}e=x_{1/2}\}
 \end{aligned}}
\end{myalign}
in the Peirce decomposition \tagref(6) of any power-associative algebra $\A$. As
we have just seen, the properties of commutative power-associative algebras
may be used (via $\A^+$) to obtain properties of arbitrary power-associative
algebras.

Let $\A$ be a commutative power-associative algebra with Peirce decomposition
\tagref(6), \tagref(7) relative to an idempotent $e$. Then $\A_1$ and $\A_0$ are orthogonal subalgebras
of $\A$ which are related to $\A_{1/2}$ as follows:
\begin{myalign}
&\A_{1/2}\A_{1/2}&\subseteq \A_1+\A_0,\\
\tag{8}& \A_1\A_{1/2}&\subseteq \A_{1/2}+\A_0,\\
&\A_0\A_{1/2}&\subseteq \A_1+\A_{1/2}.
\end{myalign}
\PG--File: 054.png---\*************\*****\********\*******\----------------
Note that the last two inclusion relations of \tagref(8) are weaker than for
Jordan algebras in \chaplink{4}. The proofs are similar to those in \chaplink{4}, and are
given by putting $x = e$, $y=y_j\in\A_j$, $z=x_i\in\A_i$ in \tagref(4$'''$). We omit the
details except to note that the characteristic $3$ case of orthogonality
requires associativity of fifth powers.

For $x\in\A_1$, $w\in\A_{1/2}$, we have $wx =(wx)_{1/2} + (wx)_0 \in\A_{1/2}+\A_0$ by \tagref(8).
Then $w\to(wx)_{1/2}$ is a linear operator on $\A_{1/2}$ which we denote by $S_x$:
\begin{myalign}
\tag{9} &&wS_x = (wx)_{1/2} &for $x\in\A_1$, $w\in\A_{1/2}$.\\
\intertext{If $\HH$ is the (associative) algebra of all linear operators on $\A_{1/2}$, then
$x\to2S_x$ is a homomorphism of $\A_1$ into the special Jordan algebra $\HH^+$,
for $x\to S_x$ is clearly linear and we verify}
\tag{10} &&S_{xy} = S_x S_y + S_y S_x &for all $x,y$ in $\A_1$
\end{myalign}
as follows: put $x\in\A_1$, $y\in\A_1$, $z = e$, $w\in\A_{1/2}$ in \tagref(4m) to obtain
\begin{myalign}(0em)
\tag{11}&& e\left[y(wx) + x(wy) + w(xy)\right] + w(xy) - 2x(yw) - 2y(xw) = 0,
\end{myalign}
since $e(yw)=\frac12(yw)_{1/2}$ implies $x\left[e(yw)\right]=\frac12x(yw)_{1/2}=\frac12x(yw)$ and $y\left[e(xw)\right]=
\frac12y(xw)$ by interchange of $x$ and $y$. By taking the $\A_{1/2}$ component in \tagref(11),
we have \tagref(10) after dividing by $3$. Similarly, defining the linear operator
$T_z$ on $\A_{1/2}$ for any $z$ in $\A_0$ by
\begin{myalign}
\tag{12}& wT_z &= (wz)_{1/2} &for $z\in\A_0$, $w\in\A_{1/2}$,
\Intertext{we have}
\tag{13}& T_{zy} &= T_z T_y + T_y T_z &for all $z,y$ in $\A_0$,
\Intertext{and}
\tag{14}& S_x T_z &= T_z S_x &for all $x$ in $\A_1$, $z$ in $\A_0$.\\
\end{myalign}
This is part of the basic machinery used in developing the structure
theory for commutative power-associative algebras as reported in the 1955 Bulletin
\hyperlink{cite.Ref64}{article}. The result that simple algebras (actually rings) of degree greater
\PG--File: 055.png---\*******\********\****\********\----------------------
than $2$ are Jordan has been extended by the same technique to flexible power-associative
rings (the conclusion being that $\A^+$ is Jordan) \cite{Ref58}. All
semisimple commutative power-associative algebras of characteristic $0$ are
Jordan algebras \cite{Ref51}. The determination of all simple commutative power-associative
algebras of degree $2$ and characteristic $p > 0$ is still not
complete \cite{Ref1, Ref24}.

Here we shall develop only as much of the technique as will be required
in the proof of the following generalization of Wedderburn's theorem that
every finite associative division ring is a field (Artin, Geometric Algebra,
p.~37). In \chaplink{4} it was mentioned that exceptional Jordan division algebras do
exist over suitable fields $F$; however, $F$ cannot be finite in that event and
we assume this particular case of the following theorem (as well as
Wedderburn's theorem) in the proof of

\begin{theorem}[10] Let $\D$ be a finite power-associative division ring of
characteristic $\ne 2,3,5$. Then $\D$ is a field.
\end{theorem}

For the proof we require an exercise and a lemma.

\begin{exercise} If $u$ and $v$ are orthogonal idempotents in a commutative
power-associative algebra $\A$, then
\begin{myalign}
\tag{15}&&  R_u R_v = R_v R_u.
\end{myalign}
(Hint: put $x = u$, $y = v$ in \tagref(4$'''$) and use \tagref(5). After considerable
manipulation, and use of characteristic $\ne 3,5$, this (together with what one
obtains by interchanging $u$ and $v$) will yield \tagref(15).)
\end{exercise}

\begin{lemma} Let $e$ be a principal idempotent in a commutative power-associative
algebra $\A$ (that is, $\A_0$ in \tagref(7) is a nilalgebra). Then every
element in $\A_{1/2}$ is nilpotent.
\PG--File: 056.png---\*******\********\****\********\----------------------
\end{lemma}

\begin{proof} To obtain \tagref(18$'$) below, one does not need to assume that $e$ is
principal. For any $w \in\A_{1/2}$ put $x = w$, $y = e$ in \tagref(4$''$) to obtain
\begin{myalign}
\tag{16}  && w^3 - w(ew^2)-ew^3 = 0     &for $w$ in $\A_{1/2}$
\end{myalign}
Let $x = (w^2)_1 \in\A_1$, $z = (w^2)_0 \in\A_0$, so that $w^2 = x + z$, $ew^2 = x$, $w(ew^2) =
wx = (wx)_{1/2} + (wx)_0$. Also $\allowbreaks w^3 = wx + wz = (wx)_{1/2} + (wx)_0 + (wz)_{1/2} + (wz)_1$
so that $ew^3 =  \frac{1}{2}(wx)_{1/2} +  \frac{1}{2}(wz)_{1/2} + (wz)_1$. Then \tagref(16) implies $(wx)_{1/2} =
(wz)_{1/2}$, or
\begin{myalign}
\tag{17}  && wS_x = wT_z  &for any $w$ in $\A_{1/2}$
\intertext{where $w^2 = x + z$, $x = (w^2)_1$, $z = (w^2)_0$. Now}
\tag{18}  &&   w{S_x}^k = w{T_z}^k  &for $k = 1,2,3,\ldots$
\intertext{For \tagref(17) is the case $k = 1$ of \tagref(18) and, assuming \tagref(18), we have $w{S_x}^{k+1} =
w{S_x}^kS_x = w{T_z}^kS_x = wS_x{T_z}^{k} = w{T_z}^{k+1}$ by \tagref(14). But \tagref(10) and \tagref(13) imply
$S_{x^k} = 2^{k-1}{S_x}^k$, $T_{z^k} = 2^{k-1}{T_z}^k$, so \tagref(18) yields}
\tag{18$'$} &&     wS_{x^k} = wT_{z^k}    &for $k = 1,2,3,\ldots$
\end{myalign}
where $w$ is any element of $\A_{1/2}$ and $w^2 = x + z$, $x \in\A_1$, $z \in\A_0$.

Now let $e$ be a principal idempotent in $\A$. Then every element $w$ in
$\A_{1/2}$ is nilpotent. For $z = (w^2)_0$ is nilpotent, and $z^k=0$ for some $k$.
By \tagref(18$'$) we have $w^{2k+1} = w(w^2)^k = w(x+z)^k = w(x^k + z^k) = wx^k = (wx^k)_{1/2} +
(wx^k)_0 = wS_{x^k} + (wx^k)_0 = (wx^k)_0$ in $\A_0$ is nilpotent; hence $w$ is
nilpotent. % possible missing text?
\end{proof}

We use the Lemma to prove that any finite-dimensional power-associative
division algebra $\D$ over a field $F$ has a unity element $1$. For $\D^+$ is a
(finite-dimensional) commutative power-associative algebra without nilpotent
elements $\ne 0$, so $\D^+$ contains a principal idempotent $e$ (as in \chaplink{4}, $\D^+$ contains
an idempotent $e$; if $e$ is not principal, there is an idempotent $u \in \D_0^+ = \D_{0,e}^+$,
\PG--File: 057.png---\*************\*****\****\********\-------------------
$e' = e + u$ is idempotent, $\dim\D_{1,e}^+ < \dim\D_{1,e'}^+$, and the increasing dimensions
must terminate). By the Lemma, since $0$ is the only nilpotent element in
$\D^+$, we have $\D_{1/2}^+ = \D_0^+ = 0,$ $\D^+ = \D_1^+$, $e$ is a unity element for $\D^+$. By \tagref(7$''$) $e$
is a unity element for $\D$.

\begin{proof}[Proof of Theorem 10]
We are assuming that $\D$ is a finite power-associative
division ring. We have seen in \chaplink{2} that this means that $\D$ is
a (finite-dimensional) division algebra over a (finite) field. Hence we
have just seen that $\D$ has a unity element $1$, so that $\D$ is an algebra over
its center. Thus we may as well take $\D$ to be an algebra over its center $F$,
a finite field. Hence $F$ is perfect (Zariski-Samuel, Commutative Algebra,
vol.~I, p.~65).

Now $\D^+$ is a Jordan algebra over $F$. For let $x, y$ be any elements of $\D^+$.
If $x \in F1$, the Jordan identity
\begin{myalign}
\tag{19} &&(x \dotm y)\dotm x^2 = x\dotm (y\dotm x^2) &for all $x, y$ in $\D^+$
\end{myalign}
holds trivially. Otherwise the (commutative associative) subalgebra $F[x]$
of $\D^+$ is a finite (necessarily separable) extension of $F$, so there is an
extension $K$ of $F$ such that $F[x]_K= K \oplus K  \oplus \dotsb  \oplus K$, $x$ is a linear combination
$x= \xi_1e_1 +\xi_2e_2 + \dotsb +\xi_ne_n$ of pairwise orthogonal idempotents $e_i$ in
$F[x]_K \subseteq (\D^+)_K$ with coefficients in $K$. In order to establish \tagref(19), it is
sufficient to establish
\begin{myalign}[0em]
\tag{19$'$} &&(e_i\dotm y)\dotm (e_j\dotm e_k) = e_i\dotm \left[y\dotm (e_j\dotm e_k)\right], &$i,j,k = 1,\ldots,n$.\\
\end{myalign}
For $j \ne k$, \tagref(19$'$) is obvious; for $j = k$, \tagref(19$'$) reduces to \tagref(15).

Now the radical of $\D^+$ (consisting of nilpotent elements) is $0$.
Although our proof of the \hyperlink{cor:thm:7}{Corollary} to Theorem 7 is valid only for
characteristic $0$, we remarked in \chaplink{4} that the conclusion is valid for
characteristic $\ne 0$. Hence $\D^+$ is a direct sum $\Ss_1 \oplus \dotsb \oplus \Ss_r$ of $r$ simple
ideals $\Ss_i$, each with unity element $e_i$. The existence of an idempotent
\PG--File: 058.png---\*************\*****\****\********\-------------------
$e \ne 1$ in $\D^+$ is sufficient to give zero divisors in $\D$, a contradiction, since
the product $e(1-e) = 0$ in $\D$. Hence $r = 1$ and $\D^+$ is a simple Jordan algebra
over $F$. Let $C$ be the center of $\D^+$. Then $C$ is a finite separable extension
of $F$, $C = F[z]$, $z \in C$ (Zariski-Samuel, ibid, p.~84). If $\D^+ = C = F[z]$, then
$\D = F[z]$ is a field, and the theorem is established. Hence we may assume
that $\D^+ \ne C$, so $\D^+$ is a central simple Jordan algebra of degree $t \ge 2$ over
the finite field $C$ and is of one of the types $\type A$--$\type E$ listed in \chaplink{4}. We are
assuming that type $\type E$ is known not to occur. The other types are eliminated
as follows.

Wedderburn's theorem implies that, over any finite field, there are no
associative central division algebras of dimension $> 1$. Hence, by Wedderburn's
theorem on simple associative algebras, every associative central simple
algebra over a finite field is a total matric algebra. Thus we have the
following possibilities:

\begin{IItemize}
\item[$\type A_\text{I}$.] $\D^+ \cong {C_t}^+$, $t \ge 2$. Then ${C_t}^+$ contains an idempotent $e_{11} \ne 1$,
a contradiction.

\item[$\type A_\text{II}$.] $\D^+$ is the set $\HH(\Z_t)$ of self-adjoint elements in $\Z_t$, $\Z$ a quadratic
extension of $C$, where the involution may be taken to be $a \to g^{-1}\overline {a'}g$ with $g$
a diagonal matrix. Hence $\HH(\Z_t)$ contains $e_{11} \ne 1$, a contradiction.

\item[$\type B$.] $\D^+ \cong\HH(C_t)$, the involution being $a \to g^{-1} a' g$ with $g$ diagonal;
hence $\HH(C_t)$ contains $e_{11} \ne 1$, a contradiction.

\item[$\type C$.] $\D^+ \cong\HH(C_{2t})$, the involution being $a \to g^{-1} a' g$, $g = \begin{pmatrix} 0 &1_t\\ -1_t &0 \end{pmatrix}$;
$\HH(C_{2t})$ contains the idempotent $e_{11}+ e_{t+1, t+1} \ne 1$, a contradiction.
\end{IItemize}

There remains the possibility that $\D^+$ might be of type $\type D$ (where the
dimension is necessarily $\ge 3$). The basis $u_1, \dotsc, u_n$ for $\M$ may be normalized
so that $(u_i, u_j)=0$ for $i \ne j$, $(u_i, u_i) = \alpha_i \ne 0$ in $C$; that is, ${u_i}^2 = \alpha_i 1$,
$u_i \dotm u_j=0$ for $i \ne j$. Each of the fields $C[u_i]$ is a quadratic extension of
\PG--File: 059.png---\*******\********\****\********\----------------------
$C$. But in the sense of isomorphism there is only one quadratic extension of
the finite field $C$ (Zariski-Samuel, ibid, pp.~73, 83); hence all $\alpha_i$ may be
taken to be the same nonsquare $\alpha$ in $C$. Let $\beta$ be any element of $C$. Then
$w = \beta u_1 + u_2 \notin F1$ implies $F[w]$ is isomorphic to $F[u_1]$, so $w^2 = (\beta^2 + 1)\alpha 1 =
\gamma^2 \alpha 1$, $\gamma$ in $C$; that is, for any $\beta \in C$, there is $\gamma \in C$ satisfying
\begin{myalign}
\tag{20}&&  \gamma^2 = \beta^2 + 1.
\end{myalign}
Now let $P$ be the prime field of characteristic $p$ contained in $C$. It
follows from \tagref(20) that all elements in $P$ are squares of elements in $C$. For
$1$ (also $0$) satisfies this condition, and it follows by induction from \tagref(20)
that all elements in $P$ do. In particular, $-1 = \beta^2$ for some $\beta \in C$. Then
$w^2 = (\beta u_1 + u_2)^2 = 0$, a contradiction.
\end{proof}

\begin{theorem}[11] Let $\A$ be a finite-dimensional power-associative algebra
over $F$ satisfying the following conditions:
\begin{myalign}[13em]
\text{(i)}\DPanchor{thm11:i}&&\text{there is an (associative) trace form $(x,y)$  defined on $\A$;}\\
\text{(ii)}\DPanchor{thm11:ii}&& (e,e) \ne 0  &for any idempotent $e$ in $\A$;\\
\text{(iii)}\DPanchor{thm11:iii}&& (x,y) = 0    &if $x \dotm y$ is nilpotent, $x, y$ in $\A$.\\
\end{myalign}
Then the nilradical $\N$ of $\A$ coincides with the nilradical of $\A^+$, and is the
radical $\A^\perp$ of the trace form $(x,y)$. The semisimple power-associative algebra
$\Ss = \A/\N$ satisfies (i)--(iii) with $(x,y)$ nondegenerate. For any such $\Ss$ we
have
\begin{myalign}
\text{(a)}\DPanchor{thm11:a}&& \text{$\Ss = \Ss_1 \oplus \dotsb \oplus \Ss_t$ for simple $\Ss_i$;}\\
\text{(b)}\DPanchor{thm11:b}&& \text{$\Ss$ is flexible.}
\intertext{If $F$ has characteristic $\ne 5$, then}
\text{(c)}\DPanchor{thm11:c}&& \text{$\Ss^+$ is a semisimple Jordan algebra;}\\
\text{(d)}\DPanchor{thm11:d}&& \text{$\Ss_i^+$ is a simple (Jordan) algebra, $i = 1,\dotsc,t$.}
\end{myalign}
\end{theorem}

\begin{proof} By (\hyperlink{thm11:i}{i}) we know from \chaplink{4} that $\A^\perp$ is an ideal of $\A$. If there
were an idempotent $e$ in $\A^\perp$, then (\hyperlink{thm11:ii}{ii}) would imply $(e,e) \ne 0$, a contradiction.
\PG--File: 060.png---\*******\*****\******\********\-----------------------
Hence $\A^\perp$ is a nilideal: $\A^\perp \subseteq\N$. Conversely, $x$ in $\N$ implies $x\dotm y$ is in $\N$
for all $y$ in $\A$, so that $(x,y) = 0$ for all $y$ in $\N$ by (\hyperlink{thm11:iii}{iii}), or $x$ is in $\A^\perp$.
Hence $\N \subseteq\A^\perp$, $\N = \A^\perp$. Any ideal of $\A$ is clearly an ideal of $\A^+$; hence
any nilideal of $\A$ is a nilideal of $\A^+$, and $\N$ is contained in the nilradical
$\N_1$ of $\A^+$. But $x$ in $\N_1$ implies $x \dotm y$ is in $\N_1$ for all $y$ in $\A^+$, or $(x,y) = 0$
by (\hyperlink{thm11:iii}{iii}) and we have $\N_1 \subseteq\A^\perp = \N$.

Now $(x,y)$ induces a nondegenerate symmetric bilinear form $(\overline x, \overline y)$ on
$\A/\A^\perp = \A/\N$ where $\overline x = x + \N$, etc.; that is, $(\overline x, \overline y) = (x,y)$. Then $(\overline x\:\overline y, \overline z) =
(\overline{xy}, \overline z) = (xy,z) = (x,yz) = (\overline x,\overline y\:\overline z)$, so $(\overline x,\overline y)$ is a trace form. To show (ii)
we take any idempotent $\overline e$ in $\A/\N$ and use the power-associativity of $\A$ to
``lift'' the idempotent to $\A$: $F[e]$ is a subalgebra of $\A$ which is not a nilalgebra,
so there is an idempotent $u \in F[e] \subseteq Fe + \N$, and $\overline u = \overline e$. Then $(\overline e,\overline e) = (\overline u,\overline u) =
(u,u) \ne 0$. Suppose $\overline x \dotm \overline y = \overline{x \dotm y}$ is nilpotent. Then some power of $x \dotm y$ is in $\N$,
$x \dotm y$ is nilpotent, and $(\overline x,\overline y) = (x,y) = 0$, establishing (\hyperlink{thm11:iii}{iii}).

Now let $\Ss$ satisfy (\hyperlink{thm11:i}{i})--(\hyperlink{thm11:iii}{iii}) with $(x,y)$ nondegenerate. Then the
nilradical of $\Ss$ is $0$, and the hypotheses of \hyperlink{Theorem:7}{Theorem~7} apply. For if $\I^2 = 0$
for an ideal $\I$ of $\Ss$, then $\I$ is a nilideal, $\I = 0$. We have $\Ss = \Ss_1 \oplus \dotsb \oplus \Ss_t$
for simple $\Ss_i$; also we know that the $\Ss_i$ are not nilalgebras (for then they
would be nilideals of $\Ss$), but this will follow from (\hyperlink{thm11:d}{d}).

Now \tagref(3$'''$) implies that $a \dotm z = 0$ where $a = 2[x \dotm y,x] + [x^2,y]$. Since
$a \dotm z$ is nilpotent, (\hyperlink{thm11:iii}{iii}) implies $\allowbreaks(a,z) = \left((xy)x,z\right) + \left((yx)x,z\right) - \left(x(xy),z\right)
- \left(x(yx),z\right) + (x^2y,z) - (yx^2,z) = 0$ for all $x, y, z$ in $\Ss$. The properties of
a trace form imply that
\begin{myalign}
\tag{21}&&
  (xy + yx,xz - zx) = (x^2,zy - yz).
\end{myalign}
Interchange $z$ and $y$ to obtain $(xz + zx,xy - yx) = (x^2,yz - zy) = (xy + yx,
zx - xz)$ by \tagref(21). Add $(xy + yx,xz + zx)$ to both sides of this to obtain
$(xy,xz + zx) = (xy + yx,zx)$. Then $(xy,xz) = (yx,zx)$, so that
\begin{myalign}
\tag{22}
  &&((xy)x,z) = (x(yx),z)   &for all $x,y,z$ in $\Ss$.
\end{myalign}
\PG--File: 061.png---\*******\********\********\*******\-------------------
Since $(x,y)$ is nondegenerate on $\Ss$, \tagref(22) implies $(xy)x = x(yx)$; that is, $\Ss$ is
flexible.

To prove (\hyperlink{thm11:c}{c}) we note first that $(x,y)$ is a trace form on $\Ss^+$:
\begin{myalign}
\tag{23}  &&   (x\dotm y,z) = (x,y\dotm z)  &   for all $x,y,z$ in $\Ss$.
\end{myalign}
Also it follows from \tagref(23), just as in formula \tagref[IV.](14) of \chaplink{4}, that
\begin{myalign}
\tag{24}  && (yS_1S_2\dotsm S_h,z) = (y,zS_h\dotsm S_2S_1)
\end{myalign}
where $S_i$ are right multiplications of the commutative algebra $\Ss^+$. In the
commutative power-associative algebra $\Ss^+$ formula \tagref(4$''$) becomes
\begin{myalign}
\tag{25} && 4x^2\dotm(x\dotm y) - 2x\dotm\left[x\dotm(x\dotm y)\right] - x\dotm(y\dotm x^2) - y\dotm x^3 = 0.
\end{myalign}
Applying the same procedure as above, we write $a$ for the left side of \tagref(25),
have $a\dotm z = 0$ for all $z$ in $\Ss^+$, so (\hyperlink{thm11:iii}{iii}) implies $\allowbreaks 4\left(x^2\dotm(x\dotm y),z\right) - 2\left(x\dotm\left[x\dotm(x\dotm y)\right],z\right)
-\left(x\dotm(y\dotm x^2),z\right) - (y\dotm x^3,z) = 0$ or
\begin{myalign}(0em)
\tag{26} && (y\dotm z,x^3) + 2\left(x\dotm[x\dotm(x\dotm y)],z\right) = 4(x\dotm y,x^2\dotm z)-(y\dotm x^2,x\dotm z).
\end{myalign}
By \tagref(24) the left-hand side of \tagref(26) is unaltered by interchange of $y$ and $z$.
Hence $4(x\dotm y, x^2\dotm z)-(y\dotm x^2,x\dotm z) = 4(x\dotm z,x^2\dotm y) - (z\dotm x^2,x\dotm y)$ so that (after
dividing by $5$) we have $(x\dotm y,x^2\dotm z) = (y\dotm x^2, x\dotm z)$. Hence $\left((x\dotm y)\dotm x^2,z\right) =
\left(x\dotm(y\dotm x^2),z\right)$ for all $z$ in $\Ss$, or $(x\dotm y)\dotm x^2 = x\dotm(y\dotm x^2)$, $\Ss^+$ is a Jordan algebra.
We know from \chaplink{4} that, since the nilradical of $\Ss^+$ is $0$, $\Ss^+$ is a direct sum
of simple ideals, but it is conceivable that these are not the ${\Ss_i}^+$ given by
(\hyperlink{thm11:a}{a}). To see that the simple components of $\Ss^+$ are the ${\Ss_i}^+$ given by (\hyperlink{thm11:a}{a}), we
need to establish (\hyperlink{thm11:d}{d}).

Let $\T$ be an ideal of ${\Ss_i}^+$; we need to show that $\T$ is an ideal of $\Ss_i$.
It follows from (\hyperlink{thm11:a}{a}) that $\T$ is an ideal of $\Ss^+$, and is therefore by (\hyperlink{thm11:c}{c}) a
direct sum of simple ideals of $\Ss^+$ each of which has a unity element. The
sum of these pairwise orthogonal idempotents in $\Ss^+$ is the unity element $e$ of
$\T$. Now $e$ is an idempotent in $\Ss^+$ (and $\Ss$), and the Peirce decomposition \tagref(7$''$)
characterizes $\T$ as
\begin{myalign}
\tag{27} &&  \T = \Ss_{1,e} = \{t \in \Ss \mid et = te = t\}.
\end{myalign}
\PG--File: 062.png---\*******\*****\********\*******\----------------------
Let $s$ be any element of $\Ss$. Then flexibility implies $(s,t,e) + (e,t,s) = 0$,
or
\begin{myalign}
\tag{28}  && (st)e - st + ts = e(ts) &for all $t \in\T$, $s \in\Ss$.
\end{myalign}
But $\T$ an ideal of $\Ss^+$ implies that $s\dotm t \in\T$, so that $st + ts = e(st + ts) =
e(st) + (st)e - st + ts$ by \tagref(27) and \tagref(28); that is, $e(st) + (st)e = 2st$, and
$st$ is in $\T=\Ss_{1,e}$ by \tagref(7$'$). But then $s\dotm t$ in $\T$ implies $ts$ is in $\T$ also;
$\T$ is an ideal of $\Ss$. Then $\T\subseteq\Ss_i$ is an ideal of $\Ss_i$. Hence the only ideals
of ${\Ss_i}^+$ are $0$ and ${\Ss_i}^+$. Since ${\Ss_i}^+$ cannot be a zero algebra, ${\Ss_i}^+$ is simple.
\end{proof}

We list without proof the central simple flexible algebras $\A$ over $F$
which are such that $\A^+$ is a (central) simple Jordan algebra. These are the
algebras which (over their centers) can appear as the simple components $\Ss_i$
in (\hyperlink{thm11:a}{a}) above:

\begin{Itemize}\let\LeftBracket\empty\let\RightBracket.
\item[1] $\A$ is a central simple (commutative) Jordan algebra.

\item[2] $\A$ is a \emph{quasiassociative} central simple algebra. That is, there is
a scalar extension $\A_K$, $K$ a quadratic extension of $F$, such that $\A_K$ is
isomorphic to an algebra $\B(\lambda)$ defined as follows: $\B$ is a central simple
associative algebra over $K$, $\lambda \ne \frac12$ is a fixed element of $K$, and $\B(\lambda)$ is the
same vector space over $K$ as $\B$ but multiplication in $\B(\lambda)$ is defined by
\begin{myalign}
\tag{29} &&  x \ast y = \lambda xy + (1-\lambda)yx &for all $x,y$ in $\B$
\end{myalign}
where $xy$ is the (associative) product in $\B$.

\item[3] $\A$ is a flexible quadratic algebra over $F$ with nondegenerate norm form.
\end{Itemize}

Note that, except for Lie algebras, all of the central simple algebras
which we have mentioned in these notes are listed here (associative algebras
are the case $\lambda=1$ (also $\lambda=0$) in 2; Cayley algebras are included in 3).

\ThoughtBreak
We should remark that, if an algebra $\A$ contains $1$, any trace form $(x,y)$
on $\A$ may be expressed in terms of a linear form $T(x)$. That is, we write
\begin{myalign}
\tag{30} &  T(x) &= (1,x)     &for all $x$ in $\A$,
\PGx--File: 063.png---\*******\********\********\*******\-------------------
\Intertext{and have}
\tag{31} &  (x,y) &= T(xy)    &for all $x,y$ in $\A$
\end{myalign}
since $(x1,y) = (1,xy)$. The symmetry and the associativity of the trace form
$(x,y)$ are equivalent to the vanishing of $T(x)$ on commutators and associators:
\begin{myalign}
\tag{32} &&{\begin{aligned} &T(xy) = T(yx),\\ &T((xy)z) = T(x(yz))
  \end{aligned}} &for all $x,y,z$ in $\A$. % not aligned in original
\end{myalign}
If $\A$ is power-associative, hypotheses (ii) and (iii) of \hyperlink{Theorem:11}{Theorem} 11 become
\begin{myalign}
\tag{33} &  T(e) &\ne 0       &for any idempotent $e$ in $\A$,\\
\Intertext{and}
\tag{34} &  T(z) &= 0         &for any nilpotent $z$ in $\A$,\\
\end{myalign}
the latter being evident as follows: \tagref(34) implies that, if $x\dotm y$ is nilpotent,
then $0=T(x\dotm y)=(1,x\dotm y)= \frac12(1,xy) + \frac12(1,yx) = \frac12(x,y) + \frac12(y,x) = (x,y)$
and, conversely, if $z = 1\dotm z$ is nilpotent, then (\hyperlink{thm11:iii}{iii}) implies $T(z) = (1,z) = 0$.

\ThoughtBreak
A natural generalization to noncommutative algebras of the class of
(commutative) Jordan algebras is the class of algebras $\J$ satisfying the
Jordan identity
\begin{myalign}
\tag{35} &&  (xy)x^2 = x(yx^2)    &for all $x,y$ in $\J$.
\end{myalign}
As in \chaplink{4}, we can linearize \tagref(35) to obtain
\begin{myalign}
\tag{35$'$} && (x,y,w\dotm z) + (w,y,z\dotm x) + (z,y,x\dotm w) = 0\\ % break not in original
&& &for all $w,x,y,z$ in $\J$.
\end{myalign}
If $\J$ contains $1$, then $w = 1$ in \tagref(35$'$) implies
\begin{myalign}
\tag{36}  &&(x,y,z) + (z,y,x) = 0 &for all $x,y,z$ in $\J$;
\intertext{that is, $\J$ is \emph{flexible}:}
\tag{37}  &&(xy)x = x(yx)    &for all $x,y$ in $\J$.
\end{myalign}
If a unity element $1$ is adjoined to $\J$ as in \chaplink{2}, then a necessary and
sufficient condition that \tagref(35$'$) be satisfied in the algebra with $1$ adjoined
is that both \tagref(35$'$) and \tagref(36) be satisfied in $\J$. Hence we define a \emph{noncommutative
Jordan algebra} to be an algebra satisfying both \tagref(35) and \tagref(37).

\begin{exercise} Prove: A flexible algebra $\J$ is a noncommutative Jordan
\PG--File: 064.png---\*******\********\********\*******\-------------------
algebra if and only if any one of the following is satisfied:
\begin{myalign}
\tag{38}  &&(x^2y)x = x^2(yx)    &for all $x,y$ in $\J$;\\
\tag{39}  &&x^2(xy) = x(x^2y)    &for all $x,y$ in $\J$;\\
\tag{40}  &&(yx)x^2 = (yx^2)x    &for all $x,y$ in $\J$;\\
\tag{41}  &&\J^+ \rlap{ is a (commutative) Jordan algebra.}
\end{myalign}
\end{exercise}

We see from \tagref(41) that any semisimple algebra (of characteristic $\ne 5$)
satisfying the hypotheses of \hyperlink{Theorem:11}{Theorem} 11 is a noncommutative Jordan algebra.

Since \tagref(35$'$) and \tagref(36) are multilinear, any scalar extension $\A_K$ of a
noncommutative Jordan algebra is a noncommutative Jordan algebra. It may
be verified directly that any noncommutative Jordan algebra is power-associative
(hence strictly power-associative).

Let $\J$ be any noncommutative Jordan algebra. By \tagref(41) $\J^+$ is a (commutative)
Jordan algebra, and we have seen in \chaplink{4} that a trace form on $\J^+$ may be given
in terms of right multiplications of $\J^+$. Our application of this to the
situation here works more smoothly if there is a unity element $1$ in $\J$, so
(if necessary) we adjoin one to $\J$ to obtain a noncommutative Jordan algebra
$\J_1$ with $1$ and having $\J$ as a subalgebra (actually ideal). Then by the proof
of \hyperlink{Theorem:6}{Theorem} 6 we know that
\begin{myalign}
\tag{42} && (x,y) = \trace R^+_{x\dotm y} = \tfrac12\trace (R_{x\dotm y} + L_{x\dotm y})\\ % break not in original
&& &for all $x,y$ in $\J_1$
\end{myalign}
is a trace form on ${\J_1}^+$ where $R^+$ indicates the right multiplication in ${\J_1}^+$;
hence \tagref(23) holds for all $x,y,z$ in $\J_1$, where $(x,y)$ is the symmetric bilinear
form \tagref(42). In terms of $T(x)$ defined by \tagref(30), we see that \tagref(23) is equivalent
to
\begin{myalign}
\tag{43}   &&T\left((x\dotm y)\dotm z\right)=T\left(x\dotm(y\dotm z)\right) &for all $x,y,z$ in $\J_1$.
\intertext{Now \tagref(36) implies}
\tag{44}   &&L_{xy} - L_yL_x + R_yR_x - R_{yx} = 0  &for all $x,y$ in $\J_1$.
\end{myalign}
Interchanging $x$ and $y$ in \tagref(44), and subtracting, we have
\PG--File: 065.png---\*******\********\********\*******\-------------------
\begin{myalign}[0em]
\tag{45}    && R_{[x,y]} + L_{[x,y]} = [R_y,R_x] + [L_x,L_y] &for all $x,y$ in $\J_1$.\\
\end{myalign}
Hence $T\left([x,y]\right) = \left(1,[x,y]\right) = \frac12\trace (R_{[x,y]} + L_{[x,y]}) = 0$ by \tagref(42) and \tagref(45).
Then $xy = x\dotm y + \frac12[x,y]$ implies $T(xy) = T(x\dotm y) = \frac12T(xy) + \frac12T(yx)$, or
\begin{myalign}
\tag{46}      &&T(xy) = T(yx) = (x,y)  &for all $x,y$ in $\J_1$
\end{myalign}
since $T(x\dotm y) = (1,x\dotm y) = (x,y)$ by \tagref(23). Now \tagref(43) and \tagref(46) imply that $(x,y)$
is a trace form on $\J_1$. For $0 = 4T\left[(x\dotm y)\dotm z-x\dotm(y\dotm z)\right] = T\left[(xy)z+(yx)z+
z(xy) + z(yx) - x(yz)-x(zy)-(yz)x-(zy)x\right] = 2T\left[(xy)z-x(yz)-(zy)x +
z(yx)\right] = 4T\left[(xy)z-x(yz)\right]$ by \tagref(36), so $T\left((xy)z\right) = T\left(x(yz)\right)$, or $(xy,z) = (x,yz)$
as desired. Then \tagref(42) induces a trace form on the subalgebra $\J$ of $\J_1$.

\begin{corollary}[to Theorem 11] Modulo its nilradical, any finite-dimensional
noncommutative Jordan algebra of characteristic $0$ is (uniquely) expressible
as a direct sum $\Ss_1\oplus\dotsb\oplus\Ss_t$ of simple ideals $\Ss_i$. Over their centers these
$\Ss_i$ are central simple algebras of the following types: (commutative) Jordan,
quasiassociative, or flexible quadratic.
\end{corollary}

\begin{proof} Only the verification for $(x,y)$ in \tagref(42) of hypotheses (\hyperlink{thm11:ii}{ii}) and
(\hyperlink{thm11:iii}{iii}) remains. But these are \tagref[IV.](12) and \tagref[IV.](8) of \chaplink{4}.
\end{proof}

It was remarked in \chaplink{4} that, although proof was given only for commutative
Jordan algebras of characteristic $0$, the results were valid for arbitrary
characteristic ($\ne 2$). The same statement cannot be made here. The trace
argument in \hyperlink{Theorem:11}{Theorem} 11 can be modified to give the direct sum decomposition
for semisimple algebras \cite{Ref58}. But new central simple algebras occur for
characteristic $p$ \cite{Ref52, Ref55}; central simple algebras which are not listed in
the Corollary above are necessarily of degree one \cite{Ref58} and are \emph{ramified} in
the sense of \cite{Ref35}.

A finite-dimensional power-associative algebra $\A$ with $1$ over $F$ is
called a \emph{nodal algebra} in case every element of $\A$ is of the form $\alpha 1 + z$
\PG--File: 066.png---\*******\************\****\********\------------------
where $\alpha \in F$ and $z$ is nilpotent, and $\A$ is not of the form $\A = F1 + \N$ for
$\N$ a nil subalgebra of $\A$. There are no such algebras which are alternative
(of arbitrary characteristic), commutative Jordan (of characteristic $\ne 2$)
\cite{Ref32}, or noncommutative Jordan of characteristic $0$. But nodal noncommutative
Jordan algebras of characteristic $p > 0$ do exist. Any nodal algebra has a
homomorphic image which is a simple nodal algebra.

Let $\J$ be a nodal noncommutative Jordan algebra over $F$. Since the
commutative Jordan algebra $\J^+$ is not a nodal algebra, $\J^+ =F1 + \N^+$ where
$\N^+$ is a nil subalgebra of $\J^+$; that is, $\J= F1 + \N$, where $\N$ is a subspace
of $\J$ consisting of all nilpotent elements of $\J$, and $x\dotm y \in\N$ for all $x,y \in\N$.
For any elements $x,y \in\N$ we have
\begin{myalign}
\tag{47}  && xy = \lambda 1 + z,  &$ \lambda \in F$, $z \in\N$.
\end{myalign}
There must exist $x, y$ in $\N$ with $\lambda\ne 0$ in \tagref(47). Since $\N^+$ is a nilpotent
commutative Jordan algebra, the powers of $\N^+$ lead to $0$; equivalently,
the subalgebra $(\N^+)^*$ of $\M(\J^+)$ is nilpotent. Now \tagref(47) implies $yx = -\lambda 1+
(2x\dotm y-z)$ and $(xy)x = \lambda x + zx = x(yx) = -\lambda x+2x(x\dotm y)-xz$, or
\begin{myalign}
\tag{48} &&  x(x\dotm y) = \lambda x + x\dotm z.
\end{myalign}
Now $\allowbreaks0=(x,x,y)+(y,x,x)=x^2y-x(\lambda 1+z)+(-\lambda 1+2x\dotm y-z)x-yx^2=
2x^2y-2\lambda x-2x\dotm z+4(x\dotm y)\dotm x-2x(x\dotm y)-2x^2\dotm y$ implies
\begin{myalign}
\tag{49} &&  x^2y=2\lambda x+2x\dotm z-2(x\dotm y)\dotm x+x^2\dotm y
\end{myalign}
by \tagref(48). Defining
\begin{myalign}
\tag{47$'$}&&   x_iy=\lambda_i1+z_i, &$\lambda_i\in F$, $z_i\in\N$,
\end{myalign}
linearization of \tagref(49) gives
\begin{myalign}
\tag{49$'$}&
   (x_1\dotm x_2)y&=\lambda_1x_2+\lambda_2x_1+x_1\dotm z_2+x_2\dotm z_1\\
&&\qquad -(x_1\dotm y)\dotm x_2-(x_2\dotm y)\dotm x_1+(x_1\dotm x_2)\dotm y.
\PGx--File: 067.png---\*************\*****\****\********\-------------------
\end{myalign}

\begin{theorem}[12] Let $\J$ be a simple nodal noncommutative Jordan algebra
over $F$. Then $F$ has characteristic $p$, $\J^+$ is the $p^n$-dimensional (commutative)
associative algebra $\J^+ = F[1, x_1,\dots,x_n]$, $x_i^p = 0$, $n \ge 2$, and multiplication
in $\J$ is given by
\begin{myalign}
\tag{50} && fg=f \dotm g+\sum_{i,j=1}^n\frac{\partial f}{\partial x_i}\dotm
 \frac{\partial g}{\partial x_j}\dotm c_{ij},&$c_{ij} = -c_{ji}$,
\end{myalign}
where at least one of the $c_{ij}$ ($= -c_{ji}$) has an inverse.
\end{theorem}

\begin{proof} Since $\J = F1 + \N$, every element $a$ in $\J$ is of the form
\begin{myalign}
\tag{51} &&  a=\alpha 1 + x,  &$\alpha \in F$, $x \in\N$.
\end{myalign}
By \tagref(51) every associator relative to the multiplication in $\J^+$ is an
associator
\begin{myalign}[0em]
\tag{52}&& [x_1,x_2,x_3] = (x_1 \dotm x_2)\dotm x_3 - x_1\dotm (x_2\dotm x_3), &$x_i \in\N$.\\
\end{myalign}
We shall first show that $\J^+$ is associative by showing that the subspace $\B$
spanned by all of the associators \tagref(52) is $0$. For any $y$ in $\N$, \tagref(49$'$) implies
that $(x_1\dotm x_2)y$ is in $\N$, so $\allowbreaks\left[(x_1 \dotm x_2 )\dotm x_3\right]y
 =\lambda_3x_1 \dotm x_2 + (x_1 \dotm x_2)\dotm z_3 +x_3\dotm \left[\lambda_1x_2\right. +
\lambda_2x_1 +x_1 \dotm z_2 + x_2\dotm z_1 -(x_1 \dotm y)\dotm x_2 -(x_2 \dotm y)\dotm x_1
 + \left.(x_1 \dotm x_2)\dotm y\right]-\left[(x_1 \dotm x_2)\dotm y\right]\dotm x_3 -
(x_3 \dotm y)\dotm (x_1 \dotm x_2) + \left[(x_1 \dotm x_2)\dotm x_3\right]\dotm y$ by \tagref(47$'$) and \tagref(49$'$).
 Interchange subscripts
$1$ and $3$, and subtract, to obtain $\allowbreaks[x_1,x_2,x_3]y = [x_1,x_2,x_3] + [x_1,z_2,x_3] +
[z_1,x_2,x_3] - [x_1\dotm y,x_2,x_3] - [x_1,x_2\dotm y,x_3] + [x_3\dotm y,x_2,x_1] + [x_1,x_2,x_3]\dotm y$, so
that we have the first inclusion in
\begin{myalign}
\tag{53} &&\B\N \subseteq\B + \B\dotm\N, \qquad \N\B \subseteq\B + \B\dotm\N.
\end{myalign}
The second part of \tagref(53) follows from $nb = -bn + 2b\dotm n$ for $b$ in $\B$, $n$ in $\N$.

Define an ascending series $\C_0 \subseteq\C_1 \subseteq\C_2 \dotsb$ of subspaces $\C_i$ of $\J$ by
\begin{myalign}
\tag{54}&& \C_0 = \B, \qquad \C_{i+1} = \C_i + \C_i\dotm\N.
\end{myalign}
Note that all the $\C_i$ are contained in $\N$ (actually in $\N \dotm\N \dotm\N$, since $\B$ is).
Since $(\N^+)^*$ is nilpotent, there is a positive integer $k$ such that $\C_{k+1} = \C_k$.
We prove by induction on $i$ that
\PG--File: 068.png---\*************\******\****\********\------------------
\begin{myalign}
\tag{55}&& \C_i\N \subseteq\C_{i+1}, \qquad \N\C_i \subseteq\C_{i+1}.
\end{myalign}
The case $i = 0$ of \tagref(55) is \tagref(53). We assume \tagref(55) and prove that $\C_{i+1}\N \subseteq\C_{i+2}$
as follows: by the assumption of the induction it is sufficient to show
\begin{myalign}
\tag{56}&&(\C_i \dotm\N)\N \subseteq\C_{i+1}+\C_{i+1}\dotm\N.
\end{myalign}
Now the flexible law \tagref(36) is equivalent to
\begin{myalign}[0em]
\tag{57} &&(x\dotm y)z = (yz)\dotm x + (yx)\dotm z - (y\dotm z)x &for all $x, y, z$ in $\J$.\\
\end{myalign}
Put $x$ in $\C_i$, $y$ and $z$ in $\N$ into \tagref(57), and use $yz=\mu1+w$, $\mu \in F$, $w \in\N$, to
see that each term of the right-hand side of \tagref(57) is in $\C_{i+1} + \C_{i+1}\dotm\N$ by the
assumption \tagref(55) of the induction. We have established \tagref(56), and therefore
$\C_{i+1}\N \subseteq\C_{i+2}$. Then, as above, $\N\C_{i+1} \subseteq\C_{i+1}\N + \C_{i+1}\dotm\N \subseteq\C_{i+2}$, and we have
established \tagref(55). For the positive integer $k$ such that $\C_{k+1} = \C_k$, we have
$\C_k$ an ideal of $\J$. For $\C_k\J=\C_k(F1 +\N) \subseteq\C_k$ by \tagref(55), and similarly $\J\C_k \subseteq\C_k$.
The ideal $\C_k$, being contained in $\N$, is not $\J$. Hence $\C_k = 0$, since $\J$ is
simple. But $\B \subseteq\C_k$, so $\B = 0$, $\J^+$ is associative.

An ideal $\I$ of an algebra $\A$ is called a \emph{characteristic} ideal (or $\D$-ideal)
in case $\I$ is mapped into itself by every derivation of $\A$. $\A$ is called \emph{$\D$-simple}
if $0$ and $\A$ are the only characteristic ideals of $\A$.

We show next that the commutative associative algebra $\J^+$ is $\D$-simple.
Interchange $x$ and $y$ in \tagref(36) to obtain
\begin{myalign}
\tag{36$'$} &&(y, x, z) + (z, x, y) = 0  &for all $x, y, z$ in $\J$;
\intertext{interchange $y$ and $z$ in \tagref(36) to obtain}
\tag{36$''$}  &&(x, z, y) + (y, z, x) = 0 &for all $x, y, z$ in $\J$;
\intertext{adding \tagref(36) and \tagref(36$'$), and subtracting \tagref(36$''$), we obtain the identity}
\tag{58} &&[x\dotm y, z] = [x, z]\dotm y + x\dotm [y, z] &for all $x, y, z$ in $\J$,
\intertext{which is valid in any flexible algebra. Identity \tagref(58) is equivalent to the
statement that}
\tag{59} &&D = R_z - L_z &for any $z$ in $\J$
\end{myalign}
\PG--File: 069.png---\*******\*******\****\********\-----------------------
is a derivation of $\J^+$. If $\I$ is an ideal of $\J^+$, then $x\dotm z$ is in $\I$ for
all $x$ in $\I$, $z$ in $\J$. If, furthermore, $\I$ is characteristic, then $[x,z] = xD$
is in $\I$, since $D$ in \tagref(59) is a derivation of $\J^+$. Hence $xz = x\dotm z + \frac{1}{2}[x,z]$
and $zx = x\dotm z - \frac{1}{2}[x,z]$ are in $\I$ for all $x$ in $\I$, $z$ in $\J$; that is, $\I$ is an
ideal of $\J$. Hence $\J$ simple implies that the commutative associative algebra
$\J^+$ is $\D$-simple.

It is a recently proved result in the theory of commutative associative
algebras (see \cite{Ref24}) that, if $\A$ is a finite-dimensional $\D$-simple commutative
associative algebra of the form $\A = F1 + \R$ where $\R$ is the radical of $\A$, then
(except for the trivial case $\A = F1$ which may occur at characteristic $0$,
and which does not give a nodal algebra) $F$ has characteristic $p$ and $\A$ is the
$p^n$-dimensional algebra $\A = F[1,x_1,\dots,x_n]$,  ${x_i}^p = 0$.

Now any derivation $D$ of such an algebra has the form
\begin{myalign}
\tag{60} &&f \to fD= \sum_{i=1}^n \frac{\partial f}{\partial x_i}\dotm a_i,  &$a_i \in\A$,
\intertext{where the $a_i$ of course depend on the derivation $D$. Then \tagref(59) implies that
$f \to [f,g]$ is a derivation of $\J^+$ for any $g$ in $\J$. By \tagref(60) we have}
\tag{61} &&[f,g]= \sum_{i=1}^n \frac{\partial f}{\partial x_i}\dotm a_i(g),  &$a_i(g) \in\J$.
\intertext{To evaluate the $a_i(g)$, note that $x_iD = [x_i,g] = a_i(g)$ and}
\tag{62} &&[g,x_i]= \sum_{j=1}^n \frac{\partial g}{\partial x_j}\dotm a_j(x_i).
\intertext{Then $a_j(x_i) = [x_j,x_i]$ implies
 $a_i(g) = -[g, x_i] = \displaystyle-\sum_{j=1}^n \frac{\partial g}{\partial x_j}\dotm [x_j, x_i]$, or}
\tag{63}   &&[f,g] = \sum_{i,j=1}^n \frac{\partial f}{\partial x_i}\dotm \frac{\partial g}{\partial x_j}\dotm [x_i,x_j]
\end{myalign}
by \tagref(61), so that $fg = f\dotm g + \frac{1}{2} [f,g]$ implies \tagref(50) where $c_{ij} = \frac{1}{2} [x_i,x_j]$.
If every $c_{ij}$ were in $\N$, then $\N$ would be a subalgebra of $\A$, a contradiction.
Hence at least one of the $c_{ij}$ is of the form \tagref(51) with $\alpha \ne 0$, so it has an
\PG--File: 070.png---\*******\*****\********\*******\----------------------
inverse, and $n \ge 2$.
\end{proof}

Not every algebra described in the conclusion of \hyperlink{Theorem:12}{Theorem} 12 is simple
(see \cite{Ref55}). However, all such algebras of dimension $p^2$ are, and for every
even $n$ there are simple algebras of dimension $p^n$. There are relationships
between the derivation algebras of nodal noncommutative Jordan algebras and
recently discovered (non-classical) simple Lie algebras of characteristic
$p$ \cite{Ref7, Ref11, Ref17, Ref68}. For a general discussion of Lie algebras of characteristic
$p$, see \cite{Ref61}.
\PG--File: 071.png---\*******\*****\********\*******\----------------------

\clearpage
The following list of papers on nonassociative algebras is intended
to bring up to date (May 1961) the selective bibliography which appears
at the end of the expository article \cite{Ref64}.


\begin{thebibliography}{99}

\bibitem{Ref1} %1.
 A.\,A. Albert, \textit{On partially stable algebras}, Trans.\ Amer.\ Math.\ Soc.\
      vol.~84 (1957), pp.~430--443; \textit{Addendum to the paper on partially
      stable algebras}, ibid, vol.~87 (1958), pp.~57--62.

\bibitem{Ref2} %2.
\bysame, \textit{A construction of exceptional Jordan division algebras},
      Ann.\ of Math.\ (2) vol.~67 (1958), pp.~1--28.

\bibitem{Ref3} %3.
\bysame, \textit{On the orthogonal equivalence of sets of real symmetric
      matrices}, J. Math.\ Mech.\ vol.~7 (1958), pp.~219--236.

\bibitem{Ref4} %4.
\bysame, \textit{Finite noncommutative division algebras}, Proc.\ Amer.\
      Math.\ Soc.\ vol.~9 (1958), pp.~928--932.

\bibitem{Ref5} %5.
\bysame, \textit{Finite division algebras and finite planes}, Proc.\
      Symposia Applied Math., vol.~X, Combinatorial Analysis, Amer.\ Math.\
      Soc.\ 1960, pp.~53--70.

\bibitem{Ref6} %6.
\bysame, \textit{A solvable exceptional Jordan algebra}, J. Math.\ Mech.\
      vol.~8 (1959), pp.~331--337. % [**period after "vol" missing][F1: inserted period]

\bibitem{Ref7} %7.
A.\,A. Albert and M.\,S. Frank, \textit{Simple Lie algebras of characteristic $p$},
      Univ.\ e Politec.\ Torino.\ Rend.\ Sem.\ Mat.\ vol.~14 (1954--5), pp.~117--139.

\bibitem{Ref8} %8.
A.\,A. Albert and N.~Jacobson, \textit{On reduced exceptional simple Jordan
      algebras}, Ann.\ of Math.\ (2) vol.~66 (1957), pp.~400--417.

\bibitem{Ref9} %9.
A.\,A. Albert and L.\,J. Paige, \textit{On a homomorphism property of certain
     Jordan algebras}, Trans.\ Amer.\ Math.\ Soc.\ vol.~93 (1959), pp.~20--29.

\bibitem{Ref10} %10.
S.\,A. Amitsur, \textit{A general theory of radicals. II.\ Radicals in rings
     and bicategories}, Amer.\ J. Math.\ vol.~76 (1954), pp.~100--125; \textit{III.\
     Applications}, ibid, pp.~126--136.

\bibitem{Ref11} %11.
Richard Block, \textit{New simple Lie algebras of prime characteristic}, Trans.\
    Amer.\ Math.\ Soc.\ vol.~89 (1958), pp.~421--449.

\bibitem{Ref12} %12.
R.~Bott and J.~Milnor, \textit{On the parallelizability of the spheres}, Bull.\
    Amer.\ Math.\ Soc.\ vol.~64 (1958), pp. 87--89.

\bibitem{Ref13} %13.
Bailey Brown and N.\,H. McCoy, \textit{Prime ideals in nonassociative rings},
    Trans.\ Amer.\ Math.\ Soc.\ vol.~89 (1958), pp.~245--255.

\PG--File: 072.png---\*******\********\********\*******\-------------------
\bibitem{Ref14} %14.
R.\,H. Bruck, \textit{Recent advances in the foundations of euclidean plane
       geometry}, Amer.\ Math.\ Monthly vol.~62 (1955), No.~7, part~II,
       pp.~2--17.

\bibitem{Ref15} %15.
P.\,M. Cohn, \textit{Two embedding theorems for Jordan algebras}, Proc.\ London
       Math.\ Soc.\ (3) vol.~9 (1959), pp.~503--524.

\bibitem{Ref16} %16.
J.~Dieudonn\'e, \textit{On semi-simple Lie algebras}, Proc.\ Amer.\ Math.\ Soc.\
       vol.~4 (1953), pp.~931--932.

\bibitem{Ref17} %17.
M.\,S. Frank, \textit{A new class of simple Lie algebras}, Proc.\ Nat.\ Acad.\
       Sci.\ U.S.A. vol.~40 (1954), pp.~713--719.

\bibitem{Ref18} %18.
Hans Freudenthal, \textit{Sur le groupe exceptionnel $E_7$}, Nederl.\ Akad.\
       Wet\-ensch.\ Proc.\ Ser.~A. vol.~56 (1953), pp.~81--89; \textit{Sur des invariants
       caract\'eristiques des groupes semi-simples}, ibid, pp.~90--94; \textit{Sur le
       groupe exceptionnel $E_8$}, ibid, pp.~95-98; \textit{Zur ebenen Oktavengeometrie},
       ibid, pp.~195--200.

\bibitem{Ref19} %19.
\bysame, \textit{Beziehungen der $E_7$ und $E_8$ zur Oktavenebene I.}
       Nederl.\ Akad.\ Wetensch.\ Proc.\ Ser.~A. vol.~57 (1954), pp.~218--230;
       \textit{II.} ibid, pp.~363--368; \textit{III.} ibid, vol.~58 (1955), pp.~151--157; \textit{IV.}
       ibid, pp.~277--285; \textit{V.} ibid, vol.~62 (1959), pp.~165--179; \textit{VI.} ibid,
       pp.~180--191; \textit{VII.} ibid, pp.~192--201; \textit{VIII.} ibid, pp.~447--465; \textit{IX.}
       ibid, pp.~466--474.

\bibitem{Ref20} %20.
\bysame, \textit{Lie groups and foundations of geometry}, Canadian
       Mathematical Congress Seminar, 1959 (mimeographed).

\bibitem{Ref21} %21.
Marshall Hall, Jr., \textit{Projective planes and related topics}, Calif.\ Inst.\
       of Tech, 1954, vi + 77pp.

\bibitem{Ref22} %22.
\bysame, \textit{An identity in Jordan rings}, Proc.\ Amer.\ Math.\ Soc.\
       vol.~7 (1956), pp.~990--998.

\bibitem{Ref23} %23.
L.\,R. Harper, Jr., \textit{Proof of an Identity on Jordan algebras}, Proc.\ Nat.\
       Acad.\ Sci.\ U.S.A. vol.~42 (1956), pp.~137--139.

\bibitem{Ref24} %24.
\bysame, \textit{On differentiably simple algebras}, Trans.\ Amer.\
       Math.\ Soc.\ vol.~100 (1961), pp.~63--72.

\bibitem{Ref25} %25.
Bruno Harris, \textit{Centralizers in Jordan algebras}, Pacific J. Math, vol.~8
       (1958), pp.~757--790.

\PG--File: 073.png---\*******\*******\********\*******\--------------------

\bibitem{Ref26}% 26.
\bysame, \textit{Derivations of Jordan algebras}, Pacific J. Math.\ vol.~9
(1959), pp.~495--512.

\bibitem{Ref27} %27.
I.\,N. Herstein, \textit{On the Lie and Jordan rings of a simple associative ring},
Amer.\ J. Math., vol.~77 (1955), pp.~279--285.

\bibitem{Ref28} %28.
\bysame, \textit{The Lie ring of a simple associative ring}, Duke Math.\ J.
vol.~22 (1955), pp.~471--476.

\bibitem{Ref29} %29.
\bysame, \textit{Jordan homomorphisms}, Trans.\ Amer.\ Math.\ Soc.\ vol~81
(1956), pp.~331--341.

\bibitem{Ref30} %30.
\bysame, \textit{Lie and Jordan systems in simple rings with involution},
Amer.\ J. Math.\ vol.~78 (1956), pp.~629--649.

\bibitem{Ref31} %31.
N.~Jacobson, \textit{Some aspects of the theory of representations of Jordan
algebras}, Proc.\ Internat.\ Congress of Math., 1954, Amsterdam, vol.~III,
pp.~28--33.

\bibitem{Ref32} %32.
\bysame, \textit{A theorem on the structure of Jordan algebras}, Proc.\ Nat.\
Acad.\ Sci.\ U.S.A. vol.~42 (1956), pp.~140--147.

\bibitem{Ref33} %33.
\bysame, \textit{Composition algebras and their automorphisms}, Rend.\ Circ.\
Mat.\ Palermo (2) vol.~7 (1958), pp.~55--80.

\bibitem{Ref34} %34.
\bysame, \textit{Nilpotent elements in semi-simple Jordan algebras}, Math.\
Ann.\ vol.~136 (1958), pp.~375--386.

\bibitem{Ref35} %35.
\bysame, \textit{Some groups of transformations defined by Jordan algebras.
I.}, J. Reine Angew.\ Math.\ vol.~201 (1959), pp.~178--195; \textit{II.}, ibid, %[**or , see next line][F1: changed "ibid." to "ibid,"]
vol.~204 (1960), pp.~74--98; \textit{III.}, ibid, vol.~207 (1961), pp.~61--85.

\bibitem{Ref36} %36.
\bysame, \textit{Exceptional Lie algebras}, dittoed, 57 pp.

\bibitem{Ref37} %37.
\bysame, \textit{Cayley planes}, dittoed, 28 pp.

\bibitem{Ref38} %38.
N.~Jacobson and L.\,J. Paige, \textit{On Jordan algebras with two generators},
J. Math.\ Mech.\ vol.~6 (1957), pp.~895--906.

\bibitem{Ref39} %39.
P.~Jordan, \textit{\"Uber eine nicht-desarguessche ebene projecktive Geometrie},
Abh.\ Math.\ Sem.\ Hamb.\ Univ.\ vol.~16 (1949), pp.~74--76.

\bibitem{Ref40} %40.
Irving Kaplansky, \textit{Lie algebras of characteristic $p$}, Trans.\ Amer.\ Math.\
Soc.\ vol.~89 (1958), pp.~149--183.

\bibitem{Ref41} %41.
Erwin Kleinfeld, \textit{Primitive alternative rings and semisimplicity}, Amer.\
J. Math.\ vol.~77 (1955), pp.~725--730.

\PG--File: 074.png---\*******\********\********\*******\-------------------
\bibitem{Ref42} %42.
\bysame, \textit{Generalization of a theorem on simple alternative rings},
       Portugal.\ Math.\ vol.~14 (1956), pp.~91--94.

\bibitem{Ref43} %43.
\bysame, \textit{Standard and accessible rings}, Canad.\ J. Math.\ vol.~8
       (1956), pp.~335--340.

\bibitem{Ref44} %44.
\bysame, \textit{Alternative nil rings}, Ann.\ of Math.\ (2) vol.~66 (1957),
       pp.~395--399.

\bibitem{Ref45} %45.
\bysame, \textit{Assosymmetric rings}, Proc.\ Amer.\ Math.\ Soc.\ vol~8 (1957),
       pp.~983--986.

\bibitem{Ref46} %46.
\bysame, \textit{A note on Moufang-Lie rings}, Proc.\ Amer.\ Math.\ Soc.\
       vol.~9 (1958), pp.~72--74.

\bibitem{Ref47} %47.
\bysame, \textit{Quasi-nil rings}, Proc.\ Amer.\ Math.\ Soc.\ vol.~10 (1959),
       pp.~477--479.

\bibitem{Ref48} %48.
\bysame, \textit{Simple algebras of type $(1,1)$ %[** I checked on Internet, it seems to be (1,1) and not (l,l)]
 are associative}, Canadian
       J. Math.\ vol.~13 (1961), pp.~129--148.

\bibitem{Ref49} %49.
Max Koecher, \textit{Analysis in reelen Jordan-Algebren}, Nachr.\ Acad.\ Wiss.\
       G\"ottingen Math.-Phys.\ Kl.\ IIa 1958, pp.~67--74.

\bibitem{Ref50} %50.
L.\,A. Kokoris, \textit{Power-associative rings of characteristic two}, Proc.\ Amer.\
       Math.\ Soc.\ vol.~6 (1955), pp.~705--710.

\bibitem{Ref51} %51.
\bysame, \textit{Simple power-associative algebras of degree two}, Ann.\ of
       Math.\ (2) vol.~64 (1956), pp.~544--550.

\bibitem{Ref52} %52.
\bysame, \textit{Simple nodal noncommutative Jordan algebras}, Proc.\ Amer.\
       Math.\ Soc.\ vol.~9 (1958), pp.~652--654.

\bibitem{Ref53} %53.
\bysame, \textit{On nilstable algebras}, Proc.\ Amer.\ Math.\ Soc.\ vol.~9
      (1958), pp.~697--701.

\bibitem{Ref54} %54.
\bysame, \textit{On rings of $(\gamma,\delta)$-type}, Proc.\ Amer.\ Math.\ Soc.\ vol.~9
      (1958), pp.~897--904.

\bibitem{Ref55} %55.
\bysame, \textit{Nodal noncommutative Jordan algebras}, Canadian J. Math.\
      vol.~12 (1960), pp.~488--492.

\bibitem{Ref56} %56.
I.\,G. Macdonald, \textit{Jordan algebras with three generators}, Proc.\ London
      Math.\ Soc.\ (3), vol.~10 (1960), pp.~395--408.

\bibitem{Ref57} %57.
R.\,H. Oehmke, \textit{A class of noncommutative power-associative algebras},
      Trans.\ Amer.\ Math.\ Soc.\ vol.~87 (1958), pp.~226--236.

\PG--File: 075.png---\*******\************\********\*******\---------------
\bibitem{Ref58} %58.
\bysame, \textit{On flexible algebras}, Ann.\ of Math.\ (2) vol.~68 (1958),
       pp.~221--230.

\bibitem{Ref59} %59.
\bysame, \textit{On flexible power-associative algebras of degree two},
       Proc.\ Amer.\ Math.\ Soc.\ vol.~12 (1961), pp.~151--158.

\bibitem{Ref60} %60.
Gunter Pickert, \textit{Projektive Ebenen}, Berlin, 1955.

\bibitem{Ref61} %6l.
Report of a Conference on Linear Algebras, Nat.\ Acad.\ of Sci.\ -- Nat.\
       Res.\ Council, Publ.\ 502, v + 60 pp. (1957) (N.~Jacobson, \textit{Jordan
       algebras}, pp.~12--19; Erwin Kleinfeld, \textit{On alternative and right
       alternative rings}, pp.~20--23; G.\,B. Seligman, \textit{A survey of Lie
       algebras of characteristic $p$}, pp.~24--32.)

\bibitem{Ref62} %62.
R.\,L. San Soucie, \textit{Right alternative rings of characteristic two},
       Proc.\ Amer.\ Math.\ Soc.\ vol.~6 (1955), pp.~716--719.

\bibitem{Ref63} %63.
\bysame, \textit{Weakly standard rings}, Amer.\ J. Math.\ vol.~79 (1957),
       pp.~80--86.

\bibitem{Ref64} %64.
R.\,D. Schafer, \textit{Structure and representation of nonassociative algebras},
       Bull.\ Amer.\ Math.\ Soc.\ vol.~61 (1955), pp.~469--484.

\bibitem{Ref65} %65.
\bysame, \textit{On noncommutative Jordan algebras}, Proc.\ Amer.\ Math.\ Soc.\ %, [F1: inconsistent comma removed]
       vol.~9 (1958), pp.~110--117.

\bibitem{Ref66} %66.
\bysame, \textit{Restricted noncommutative Jordan algebras of characteristic
       $p$}, Proc.\ Amer.\ Math.\ Soc.\ vol.~9 (1958), pp.~141--144.

\bibitem{Ref67} %67.
\bysame, \textit{On cubic forms permitting composition}, Proc.\ Amer.\ Math.\
       Soc.\ vol.~10 (1959), pp.~917--925.

\bibitem{Ref68} %68.
\bysame, \textit{Nodal noncommutative Jordan algebras and simple Lie
       algebras of characteristic $p$}, Trans.\ Amer.\ Math.\ Soc.\ vol.~94 (1960),
       pp.~310--326.

\bibitem{Ref69} %69.
\bysame, \textit{Cubic forms permitting a new type of composition}, J. Math.\
       Mech.\ vol.~10 (1961), pp.~159--174.

\bibitem{Ref70} %70.
G.\,B. Seligman, \textit{On automorphisms of Lie algebras of classical type},
       Trans.\ Amer.\ Math.\ Soc.\ vol.~92 (1959), pp.~430--448; \textit{II}, ibid, vol.~94
       (1960), pp.~452--482; \textit{III}, ibid, vol.~97 (1960), pp.~286--316.

\bibitem{Ref71} %71.
A.\,I. Shirshov, \textit{On special $J$-rings}, Mat.\ Sbornik N.~S. vol.~38 (80)
       (1956), pp.~149--166 (Russian).

\bibitem{Ref72} %72.
\bysame, \textit{Some questions in the theory of rings close to associative},
       Uspehi Mat.\ Nauk, vol.~13 (1958), no.~6 (84), pp.~3--20 (Russian).

\PG--File: 076.png---\*******\************\********\*******\---------------
\bibitem{Ref73} %73.
M.\,F. Smiley, \textit{Jordan homomorphisms onto prime rings}, Trans.\ Amer.\ Math.\
       Soc.\ vol.~84 (1957), pp.~426--429.

\bibitem{Ref74} %74.
\bysame, \textit{Jordan homomorphisms and right alternative rings}, Proc.\
       Amer.\ Math.\ Soc.\ vol.~8 (1957), pp.~668--671.

\bibitem{Ref75} %75.
T.\,A. Springer, \textit{On a class of Jordan algebras}, Nederl.\ Akad.\ Wetensch.\
       Proc.\ Ser.~A. vol.~62 (1959), pp.~254--264.

\bibitem{Ref76} %76.
\bysame, \textit{The projective octave plane}, Nederl.\ Akad.\ Wetensch.\
       Proc.\ Ser.~A. vol.~63 (1960), pp.~74--101.

\bibitem{Ref77} %77.
\bysame, \textit{The classification of reduced exceptional simple
       Jordan algebras}, Nederl.\ Akad.\ Wetensch.\ Proc.\ Ser.~A. vol.~63
       (1960), pp.~414--422.

\bibitem{Ref78} %78.
Taeil Suh, \textit{On isomorphisms of little projective groups of Cayley planes},
       Yale dissertation, 1960.

\bibitem{Ref79} %79.
E.\,J. Taft, \textit{Invariant Wedderburn factors}, Illinois J. Math.\ vol.~1
       (1957), pp.~565--573.

\bibitem{Ref80} %80.
\bysame, \textit{The Whitehead first lemma for alternative algebras},
       Proc.\ Amer.\ Math.\ Soc.\ vol.~8 (1957), pp.~950--956.

\bibitem{Ref81} %81.
J.~Tits, \textit{Le plan projectif des octaves et les groupes de Lie exceptionnels},
       Acad.\ Roy.\ Belgique Bull.\ Cl.\ Sci.\ (5) vol.~39 (1953), pp.~309--329;
       \textit{Le plan projectif des octaves et les groupes exceptionnels $E_6$ et $E_7$},
       ibid, vol.~40 (1954), pp.~29--40.

\bibitem{Ref82} %82.
\bysame, \textit{Sur la trialit\'e et les alg\`ebres d'octaves}, Acad.\ Roy.\ Belg.\
       Bull.\ Cl.\ Sci.\ (5) vol.~44 (1958), pp.~332--350.

\bibitem{Ref83} %83.
M.\,L. Tomber, \textit{Lie algebras of types $A$, $B$, $C$, $D$, and $F$}, Trans.\ Amer.\
       Math.\ Soc.\ vol.~88 (1958), pp.~99--106.

\bibitem{Ref84} %84.
F.~van der Blij and T.\,A. Springer, \textit{The arithmetics of octaves and of the
       group $G_2$}, Nederl.\ Akad.\ Wetensch.\ Proc.\ Ser.~A. vol.~62 (1959),
       pp.~406--418.

\bibitem{Ref85} %85.
L.\,M. Weiner, \textit{Lie admissible algebras}, Univ.\ Nac.\ Tucum\'an.\ Rev.\ Ser.~A.
       vol.~II (1957), pp.~10--24.

\end{thebibliography}

% we *do* want the licence to start recto, to emphasise it is an addition
\cleartorecto
\pagestyle{licence}
\setlength\parskip{0pt}\raggedbottom
\phantomsection
\par
\DPpdfbookmark[0]{Licensing Information}{PGlicence}
\begin{PGboilerplate}[\tiny]| 8pt for B5
End of the Project Gutenberg EBook of An Introduction to Nonassociative
Algebras, by R. D. Schafer

*** END OF THIS PROJECT GUTENBERG EBOOK NONASSOCIATIVE ALGEBRAS ***

***** This file should be named 25156-pdf.pdf or 25156-pdf.zip *****
This and all associated files of various formats will be found in:
        http://www.gutenberg.org/2/5/1/5/25156/

Produced by David Starner, David Wilson, Suzanne Lybarger
and the Online Distributed Proofreading Team at
http://www.pgdp.net


Updated editions will replace the previous one--the old editions
will be renamed.

Creating the works from public domain print editions means that no
one owns a United States copyright in these works, so the Foundation
(and you!) can copy and distribute it in the United States without
permission and without paying copyright royalties.  Special rules,
set forth in the General Terms of Use part of this license, apply to
copying and distributing Project Gutenberg-tm electronic works to
protect the PROJECT GUTENBERG-tm concept and trademark.  Project
Gutenberg is a registered trademark, and may not be used if you
charge for the eBooks, unless you receive specific permission.  If you
do not charge anything for copies of this eBook, complying with the
rules is very easy.  You may use this eBook for nearly any purpose
such as creation of derivative works, reports, performances and
research.  They may be modified and printed and given away--you may do
practically ANYTHING with public domain eBooks.  Redistribution is
subject to the trademark license, especially commercial
redistribution.



*** START: FULL LICENSE ***

THE FULL PROJECT GUTENBERG LICENSE
PLEASE READ THIS BEFORE YOU DISTRIBUTE OR USE THIS WORK

To protect the Project Gutenberg-tm mission of promoting the free
distribution of electronic works, by using or distributing this work
(or any other work associated in any way with the phrase "Project
Gutenberg"), you agree to comply with all the terms of the Full Project
Gutenberg-tm License (available with this file or online at
http://gutenberg.org/license).


Section 1.  General Terms of Use and Redistributing Project Gutenberg-tm
electronic works

1.A.  By reading or using any part of this Project Gutenberg-tm
electronic work, you indicate that you have read, understand, agree to
and accept all the terms of this license and intellectual property
(trademark/copyright) agreement.  If you do not agree to abide by all
the terms of this agreement, you must cease using and return or destroy
all copies of Project Gutenberg-tm electronic works in your possession.
If you paid a fee for obtaining a copy of or access to a Project
Gutenberg-tm electronic work and you do not agree to be bound by the
terms of this agreement, you may obtain a refund from the person or
entity to whom you paid the fee as set forth in paragraph 1.E.8.

1.B.  "Project Gutenberg" is a registered trademark.  It may only be
used on or associated in any way with an electronic work by people who
agree to be bound by the terms of this agreement.  There are a few
things that you can do with most Project Gutenberg-tm electronic works
even without complying with the full terms of this agreement.  See
paragraph 1.C below.  There are a lot of things you can do with Project
Gutenberg-tm electronic works if you follow the terms of this agreement
and help preserve free future access to Project Gutenberg-tm electronic
works.  See paragraph 1.E below.

1.C.  The Project Gutenberg Literary Archive Foundation ("the Foundation"
or PGLAF), owns a compilation copyright in the collection of Project
Gutenberg-tm electronic works.  Nearly all the individual works in the
collection are in the public domain in the United States.  If an
individual work is in the public domain in the United States and you are
located in the United States, we do not claim a right to prevent you from
copying, distributing, performing, displaying or creating derivative
works based on the work as long as all references to Project Gutenberg
are removed.  Of course, we hope that you will support the Project
Gutenberg-tm mission of promoting free access to electronic works by
freely sharing Project Gutenberg-tm works in compliance with the terms of
this agreement for keeping the Project Gutenberg-tm name associated with
the work.  You can easily comply with the terms of this agreement by
keeping this work in the same format with its attached full Project
Gutenberg-tm License when you share it without charge with others.

1.D.  The copyright laws of the place where you are located also govern
what you can do with this work.  Copyright laws in most countries are in
a constant state of change.  If you are outside the United States, check
the laws of your country in addition to the terms of this agreement
before downloading, copying, displaying, performing, distributing or
creating derivative works based on this work or any other Project
Gutenberg-tm work.  The Foundation makes no representations concerning
the copyright status of any work in any country outside the United
States.

1.E.  Unless you have removed all references to Project Gutenberg:

1.E.1.  The following sentence, with active links to, or other immediate
access to, the full Project Gutenberg-tm License must appear prominently
whenever any copy of a Project Gutenberg-tm work (any work on which the
phrase "Project Gutenberg" appears, or with which the phrase "Project
Gutenberg" is associated) is accessed, displayed, performed, viewed,
copied or distributed:

This eBook is for the use of anyone anywhere at no cost and with
almost no restrictions whatsoever.  You may copy it, give it away or
re-use it under the terms of the Project Gutenberg License included
with this eBook or online at www.gutenberg.org

1.E.2.  If an individual Project Gutenberg-tm electronic work is derived
from the public domain (does not contain a notice indicating that it is
posted with permission of the copyright holder), the work can be copied
and distributed to anyone in the United States without paying any fees
or charges.  If you are redistributing or providing access to a work
with the phrase "Project Gutenberg" associated with or appearing on the
work, you must comply either with the requirements of paragraphs 1.E.1
through 1.E.7 or obtain permission for the use of the work and the
Project Gutenberg-tm trademark as set forth in paragraphs 1.E.8 or
1.E.9.

1.E.3.  If an individual Project Gutenberg-tm electronic work is posted
with the permission of the copyright holder, your use and distribution
must comply with both paragraphs 1.E.1 through 1.E.7 and any additional
terms imposed by the copyright holder.  Additional terms will be linked
to the Project Gutenberg-tm License for all works posted with the
permission of the copyright holder found at the beginning of this work.

1.E.4.  Do not unlink or detach or remove the full Project Gutenberg-tm
License terms from this work, or any files containing a part of this
work or any other work associated with Project Gutenberg-tm.

1.E.5.  Do not copy, display, perform, distribute or redistribute this
electronic work, or any part of this electronic work, without
prominently displaying the sentence set forth in paragraph 1.E.1 with
active links or immediate access to the full terms of the Project
Gutenberg-tm License.

1.E.6.  You may convert to and distribute this work in any binary,
compressed, marked up, nonproprietary or proprietary form, including any
word processing or hypertext form.  However, if you provide access to or
distribute copies of a Project Gutenberg-tm work in a format other than
"Plain Vanilla ASCII" or other format used in the official version
posted on the official Project Gutenberg-tm web site (www.gutenberg.org),
you must, at no additional cost, fee or expense to the user, provide a
copy, a means of exporting a copy, or a means of obtaining a copy upon
request, of the work in its original "Plain Vanilla ASCII" or other
form.  Any alternate format must include the full Project Gutenberg-tm
License as specified in paragraph 1.E.1.

1.E.7.  Do not charge a fee for access to, viewing, displaying,
performing, copying or distributing any Project Gutenberg-tm works
unless you comply with paragraph 1.E.8 or 1.E.9.

1.E.8.  You may charge a reasonable fee for copies of or providing
access to or distributing Project Gutenberg-tm electronic works provided
that

- You pay a royalty fee of 20% of the gross profits you derive from
     the use of Project Gutenberg-tm works calculated using the method
     you already use to calculate your applicable taxes.  The fee is
     owed to the owner of the Project Gutenberg-tm trademark, but he
     has agreed to donate royalties under this paragraph to the
     Project Gutenberg Literary Archive Foundation.  Royalty payments
     must be paid within 60 days following each date on which you
     prepare (or are legally required to prepare) your periodic tax
     returns.  Royalty payments should be clearly marked as such and
     sent to the Project Gutenberg Literary Archive Foundation at the
     address specified in Section 4, "Information about donations to
     the Project Gutenberg Literary Archive Foundation."

- You provide a full refund of any money paid by a user who notifies
     you in writing (or by e-mail) within 30 days of receipt that s/he
     does not agree to the terms of the full Project Gutenberg-tm
     License.  You must require such a user to return or
     destroy all copies of the works possessed in a physical medium
     and discontinue all use of and all access to other copies of
     Project Gutenberg-tm works.

- You provide, in accordance with paragraph 1.F.3, a full refund of any
     money paid for a work or a replacement copy, if a defect in the
     electronic work is discovered and reported to you within 90 days
     of receipt of the work.

- You comply with all other terms of this agreement for free
     distribution of Project Gutenberg-tm works.

1.E.9.  If you wish to charge a fee or distribute a Project Gutenberg-tm
electronic work or group of works on different terms than are set
forth in this agreement, you must obtain permission in writing from
both the Project Gutenberg Literary Archive Foundation and Michael
Hart, the owner of the Project Gutenberg-tm trademark.  Contact the
Foundation as set forth in Section 3 below.

1.F.

1.F.1.  Project Gutenberg volunteers and employees expend considerable
effort to identify, do copyright research on, transcribe and proofread
public domain works in creating the Project Gutenberg-tm
collection.  Despite these efforts, Project Gutenberg-tm electronic
works, and the medium on which they may be stored, may contain
"Defects," such as, but not limited to, incomplete, inaccurate or
corrupt data, transcription errors, a copyright or other intellectual
property infringement, a defective or damaged disk or other medium, a
computer virus, or computer codes that damage or cannot be read by
your equipment.

1.F.2.  LIMITED WARRANTY, DISCLAIMER OF DAMAGES - Except for the "Right
of Replacement or Refund" described in paragraph 1.F.3, the Project
Gutenberg Literary Archive Foundation, the owner of the Project
Gutenberg-tm trademark, and any other party distributing a Project
Gutenberg-tm electronic work under this agreement, disclaim all
liability to you for damages, costs and expenses, including legal
fees.  YOU AGREE THAT YOU HAVE NO REMEDIES FOR NEGLIGENCE, STRICT
LIABILITY, BREACH OF WARRANTY OR BREACH OF CONTRACT EXCEPT THOSE
PROVIDED IN PARAGRAPH F3.  YOU AGREE THAT THE FOUNDATION, THE
TRADEMARK OWNER, AND ANY DISTRIBUTOR UNDER THIS AGREEMENT WILL NOT BE
LIABLE TO YOU FOR ACTUAL, DIRECT, INDIRECT, CONSEQUENTIAL, PUNITIVE OR
INCIDENTAL DAMAGES EVEN IF YOU GIVE NOTICE OF THE POSSIBILITY OF SUCH
DAMAGE.

1.F.3.  LIMITED RIGHT OF REPLACEMENT OR REFUND - If you discover a
defect in this electronic work within 90 days of receiving it, you can
receive a refund of the money (if any) you paid for it by sending a
written explanation to the person you received the work from.  If you
received the work on a physical medium, you must return the medium with
your written explanation.  The person or entity that provided you with
the defective work may elect to provide a replacement copy in lieu of a
refund.  If you received the work electronically, the person or entity
providing it to you may choose to give you a second opportunity to
receive the work electronically in lieu of a refund.  If the second copy
is also defective, you may demand a refund in writing without further
opportunities to fix the problem.

1.F.4.  Except for the limited right of replacement or refund set forth
in paragraph 1.F.3, this work is provided to you 'AS-IS' WITH NO OTHER
WARRANTIES OF ANY KIND, EXPRESS OR IMPLIED, INCLUDING BUT NOT LIMITED TO
WARRANTIES OF MERCHANTIBILITY OR FITNESS FOR ANY PURPOSE.

1.F.5.  Some states do not allow disclaimers of certain implied
warranties or the exclusion or limitation of certain types of damages.
If any disclaimer or limitation set forth in this agreement violates the
law of the state applicable to this agreement, the agreement shall be
interpreted to make the maximum disclaimer or limitation permitted by
the applicable state law.  The invalidity or unenforceability of any
provision of this agreement shall not void the remaining provisions.

1.F.6.  INDEMNITY - You agree to indemnify and hold the Foundation, the
trademark owner, any agent or employee of the Foundation, anyone
providing copies of Project Gutenberg-tm electronic works in accordance
with this agreement, and any volunteers associated with the production,
promotion and distribution of Project Gutenberg-tm electronic works,
harmless from all liability, costs and expenses, including legal fees,
that arise directly or indirectly from any of the following which you do
or cause to occur: (a) distribution of this or any Project Gutenberg-tm
work, (b) alteration, modification, or additions or deletions to any
Project Gutenberg-tm work, and (c) any Defect you cause.


Section  2.  Information about the Mission of Project Gutenberg-tm

Project Gutenberg-tm is synonymous with the free distribution of
electronic works in formats readable by the widest variety of computers
including obsolete, old, middle-aged and new computers.  It exists
because of the efforts of hundreds of volunteers and donations from
people in all walks of life.

Volunteers and financial support to provide volunteers with the
assistance they need, is critical to reaching Project Gutenberg-tm's
goals and ensuring that the Project Gutenberg-tm collection will
remain freely available for generations to come.  In 2001, the Project
Gutenberg Literary Archive Foundation was created to provide a secure
and permanent future for Project Gutenberg-tm and future generations.
To learn more about the Project Gutenberg Literary Archive Foundation
and how your efforts and donations can help, see Sections 3 and 4
and the Foundation web page at http://www.pglaf.org.


Section 3.  Information about the Project Gutenberg Literary Archive
Foundation

The Project Gutenberg Literary Archive Foundation is a non profit
501(c)(3) educational corporation organized under the laws of the
state of Mississippi and granted tax exempt status by the Internal
Revenue Service.  The Foundation's EIN or federal tax identification
number is 64-6221541.  Its 501(c)(3) letter is posted at
http://pglaf.org/fundraising.  Contributions to the Project Gutenberg
Literary Archive Foundation are tax deductible to the full extent
permitted by U.S. federal laws and your state's laws.

The Foundation's principal office is located at 4557 Melan Dr. S.
Fairbanks, AK, 99712., but its volunteers and employees are scattered
throughout numerous locations.  Its business office is located at
809 North 1500 West, Salt Lake City, UT 84116, (801) 596-1887, email
business@pglaf.org.  Email contact links and up to date contact
information can be found at the Foundation's web site and official
page at http://pglaf.org

For additional contact information:
     Dr. Gregory B. Newby
     Chief Executive and Director
     gbnewby@pglaf.org


Section 4.  Information about Donations to the Project Gutenberg
Literary Archive Foundation

Project Gutenberg-tm depends upon and cannot survive without wide
spread public support and donations to carry out its mission of
increasing the number of public domain and licensed works that can be
freely distributed in machine readable form accessible by the widest
array of equipment including outdated equipment.  Many small donations
($1 to $5,000) are particularly important to maintaining tax exempt
status with the IRS.

The Foundation is committed to complying with the laws regulating
charities and charitable donations in all 50 states of the United
States.  Compliance requirements are not uniform and it takes a
considerable effort, much paperwork and many fees to meet and keep up
with these requirements.  We do not solicit donations in locations
where we have not received written confirmation of compliance.  To
SEND DONATIONS or determine the status of compliance for any
particular state visit http://pglaf.org

While we cannot and do not solicit contributions from states where we
have not met the solicitation requirements, we know of no prohibition
against accepting unsolicited donations from donors in such states who
approach us with offers to donate.

International donations are gratefully accepted, but we cannot make
any statements concerning tax treatment of donations received from
outside the United States.  U.S. laws alone swamp our small staff.

Please check the Project Gutenberg Web pages for current donation
methods and addresses.  Donations are accepted in a number of other
ways including checks, online payments and credit card donations.
To donate, please visit: http://pglaf.org/donate


Section 5.  General Information About Project Gutenberg-tm electronic
works.

Professor Michael S. Hart is the originator of the Project Gutenberg-tm
concept of a library of electronic works that could be freely shared
with anyone.  For thirty years, he produced and distributed Project
Gutenberg-tm eBooks with only a loose network of volunteer support.


Project Gutenberg-tm eBooks are often created from several printed
editions, all of which are confirmed as Public Domain in the U.S.
unless a copyright notice is included.  Thus, we do not necessarily
keep eBooks in compliance with any particular paper edition.


Most people start at our Web site which has the main PG search facility:

     http://www.gutenberg.org

This Web site includes information about Project Gutenberg-tm,
including how to make donations to the Project Gutenberg Literary
Archive Foundation, how to help produce our new eBooks, and how to
subscribe to our email newsletter to hear about new eBooks.
\end{PGboilerplate}
% %%%%%%%%%%%%%%%%%%%%%%%%%%%%%%%%%%%%%%%%%%%%%%%%%%%%%%%%%%%%%%%%%%%%%%% %
%                                                                         %
% End of the Project Gutenberg EBook of An Introduction to Nonassociative %
% Algebras, by R. D. Schafer                                              %
%                                                                         %
% *** END OF THIS PROJECT GUTENBERG EBOOK NONASSOCIATIVE ALGEBRAS ***     %
%                                                                         %
% ***** This file should be named 25156-t.tex or 25156-t.zip *****        %
% This and all associated files of various formats will be found in:      %
%         http://www.gutenberg.org/2/5/1/5/25156/                         %
%                                                                         %
% %%%%%%%%%%%%%%%%%%%%%%%%%%%%%%%%%%%%%%%%%%%%%%%%%%%%%%%%%%%%%%%%%%%%%%% %

\end{document}

###
$PageSeparator = qr/^\\PGx?-/;
@ControlwordReplace = (
  ['\\dotsc', '...'],
  ['\\bysame','----'],
  ['\\maketitle','\n\n\n\nTranscriber\'s Notes\n']
                      );
@MathEnvironments = (
  ['\\begin{myalign}','\\end{myalign}','<aligned/tagged equation>']
                    );
@ControlwordArguments = (
  ['\\tagref', 0, 0, '', ''],
  ['\\item', 0, 1, '(', ')'],
  ['\\DPpdfbookmark', 0, 0, '', '', 1, 0, '', '', 1, 0, '', ''],
  ['\\DPanchor', 1, 0, '', ''],
  ['\\DPlabel', 1, 0, '', ''],
  ['\\begin\\{theorem\\}', 0, 1, 'Theorem ', '. '],
  ['\\begin\\{corollary\\}', 0, 1, 'Corollary ', '. '],
  ['\\begin\\{proposition\\}', 0, 1, 'Proposition ', '. '],
  ['\\begin\\{lemma\\}', 0, 1, 'Lemma', '. '],
  ['\\begin\\{exercise\\}', 0, 1, 'Exercise', '. '],
  ['\\begin\\{proof\\}', 0, '[Proof]', '', ': '],
  ['\\begin\\{thebibliography\\}', 1, 0, '', ''],
  ['\\cite', 1, 1, '[', ']'],
  ['\\chaplink', 1, 1, '<Ch.~', '>'],
                        );

$CustomClean = 'print "\\nCustom cleaning in progress...";
my $cline = 0;
while ($cline <=$#file) {
  $file[$cline] =~ s/\\|.*$//g; # strip comments in PGboilerplate
  $cline++
}
print "done\\n";';

###
This is pdfeTeX, Version 3.141592-1.21a-2.2 (Web2C 7.5.4) (format=pdflatex 2007.10.4)  24 APR 2008 11:30
entering extended mode
**25156-t.tex
(./25156-t.tex
LaTeX2e <2003/12/01>
Babel <v3.8d> and hyphenation patterns for american, french, german, ngerman, b
ahasa, basque, bulgarian, catalan, croatian, czech, danish, dutch, esperanto, e
stonian, finnish, greek, icelandic, irish, italian, latin, magyar, norsk, polis
h, portuges, romanian, russian, serbian, slovak, slovene, spanish, swedish, tur
kish, ukrainian, nohyphenation, loaded.
\PGheader=\toks14
(/usr/share/texmf-tetex/tex/latex/memoir/memoir.cls
Document Class: memoir 2004/04/05 v1.61 configurable document class
\onelineskip=\skip41
\lxvchars=\skip42
\xlvchars=\skip43
\@memcnta=\count79
\stockheight=\skip44
\stockwidth=\skip45
\trimtop=\skip46
\trimedge=\skip47
(/usr/share/texmf-tetex/tex/latex/memoir/mem12.clo
File: mem12.clo 2004/03/12 v0.3 memoir class 12pt size option
)
\spinemargin=\skip48
\foremargin=\skip49
\uppermargin=\skip50
\lowermargin=\skip51
\headdrop=\skip52
\normalrulethickness=\skip53
\headwidth=\skip54
\c@storedpagenumber=\count80
\thanksmarkwidth=\skip55
\thanksmarksep=\skip56
\droptitle=\skip57
\abstitleskip=\skip58
\absleftindent=\skip59
\absrightindent=\skip60
\absparindent=\skip61
\absparsep=\skip62
\c@part=\count81
\c@chapter=\count82
\c@section=\count83
\c@subsection=\count84
\c@subsubsection=\count85
\c@paragraph=\count86
\c@subparagraph=\count87
\beforechapskip=\skip63
\midchapskip=\skip64
\afterchapskip=\skip65
\chapindent=\skip66
\bottomsectionskip=\skip67
\secindent=\skip68
\beforesecskip=\skip69
\aftersecskip=\skip70
\subsecindent=\skip71
\beforesubsecskip=\skip72
\aftersubsecskip=\skip73
\subsubsecindent=\skip74
\beforesubsubsecskip=\skip75
\aftersubsubsecskip=\skip76
\paraindent=\skip77
\beforeparaskip=\skip78
\afterparaskip=\skip79
\subparaindent=\skip80
\beforesubparaskip=\skip81
\aftersubparaskip=\skip82
\pfbreakskip=\skip83
\c@@ppsavesec=\count88
\c@@ppsaveapp=\count89
\ragrparindent=\dimen102
\parsepi=\skip84
\topsepi=\skip85
\itemsepi=\skip86
\parsepii=\skip87
\topsepii=\skip88
\topsepiii=\skip89
\m@msavetopsep=\skip90
\m@msavepartopsep=\skip91
\@enLab=\toks15
\c@vslineno=\count90
\c@poemline=\count91
\c@modulo@vs=\count92
\vleftskip=\skip92
\vrightskip=\skip93
\stanzaskip=\skip94
\versewidth=\skip95
\vgap=\skip96
\vindent=\skip97
\c@verse=\count93
\c@chrsinstr=\count94
\beforepoemtitleskip=\skip98
\afterpoemtitleskip=\skip99
\col@sep=\dimen103
\extrarowheight=\dimen104
\NC@list=\toks16
\extratabsurround=\skip100
\backup@length=\skip101
\TX@col@width=\dimen105
\TX@old@table=\dimen106
\TX@old@col=\dimen107
\TX@target=\dimen108
\TX@delta=\dimen109
\TX@cols=\count95
\TX@ftn=\toks17
\heavyrulewidth=\dimen110
\lightrulewidth=\dimen111
\cmidrulewidth=\dimen112
\belowrulesep=\dimen113
\belowbottomsep=\dimen114
\aboverulesep=\dimen115
\abovetopsep=\dimen116
\cmidrulesep=\dimen117
\cmidrulekern=\dimen118
\defaultaddspace=\dimen119
\@cmidla=\count96
\@cmidlb=\count97
\@aboverulesep=\dimen120
\@belowrulesep=\dimen121
\@thisruleclass=\count98
\@lastruleclass=\count99
\@thisrulewidth=\dimen122
\ctableftskip=\skip102
\ctabrightskip=\skip103
\abovecolumnspenalty=\count100
\@linestogo=\count101
\@cellstogo=\count102
\@cellsincolumn=\count103
\crtok=\toks18
\@mincolumnwidth=\dimen123
\c@newflo@tctr=\count104
\@contcwidth=\skip104
\@contindw=\skip105
\abovecaptionskip=\skip106
\belowcaptionskip=\skip107
\subfloattopskip=\skip108
\subfloatcapskip=\skip109
\subfloatcaptopadj=\skip110
\subfloatbottomskip=\skip111
\subfloatlabelskip=\skip112
\subfloatcapmargin=\dimen124
\c@@contsubnum=\count105
\beforeepigraphskip=\skip113
\afterepigraphskip=\skip114
\epigraphwidth=\skip115
\epigraphrule=\skip116
LaTeX Info: Redefining \emph on input line 4419.
LaTeX Info: Redefining \em on input line 4420.
\tocentryskip=\skip117
\tocbaseline=\skip118
\cftparskip=\skip119
\cftbeforepartskip=\skip120
\cftpartindent=\skip121
\cftpartnumwidth=\skip122
\cftbeforechapterskip=\skip123
\cftchapterindent=\skip124
\cftchapternumwidth=\skip125
\cftbeforesectionskip=\skip126
\cftsectionindent=\skip127
\cftsectionnumwidth=\skip128
\cftbeforesubsectionskip=\skip129
\cftsubsectionindent=\skip130
\cftsubsectionnumwidth=\skip131
\cftbeforesubsubsectionskip=\skip132
\cftsubsubsectionindent=\skip133
\cftsubsubsectionnumwidth=\skip134
\cftbeforeparagraphskip=\skip135
\cftparagraphindent=\skip136
\cftparagraphnumwidth=\skip137
\cftbeforesubparagraphskip=\skip138
\cftsubparagraphindent=\skip139
\cftsubparagraphnumwidth=\skip140
\c@maxsecnumdepth=\count106
\bibindent=\dimen125
\bibitemsep=\skip141
\indexcolsep=\skip142
\indexrule=\skip143
\indexmarkstyle=\toks19
\@indexbox=\insert233
\sideparvshift=\skip144
\sideins=\insert232
\sidebarhsep=\skip145
\sidebarvsep=\skip146
\sidebarwidth=\skip147
\footmarkwidth=\skip148
\footmarksep=\skip149
\footparindent=\skip150
\footinsdim=\skip151
\footinsv@r=\insert231
\@mpfootinsv@r=\insert230
\m@m@k=\count107
\m@m@h=\dimen126
\m@mipn@skip=\skip152
\c@sheetsequence=\count108
\c@lastsheet=\count109
\c@lastpage=\count110
\every@verbatim=\toks20
\afterevery@verbatim=\toks21
\verbatim@line=\toks22
\tab@position=\count111
\verbatim@in@stream=\read1
\verbatimindent=\skip153
\verbatim@out=\write3
\bvboxsep=\skip154
\c@bvlinectr=\count112
\bvnumlength=\skip155
\FrameRule=\dimen127
\FrameSep=\dimen128
\c@cp@cntr=\count113
LaTeX Info: Redefining \: on input line 7541.
LaTeX Info: Redefining \! on input line 7543.
\c@ism@mctr=\count114
\c@xsm@mctr=\count115
\c@csm@mctr=\count116
\c@ksm@mctr=\count117
\c@xksm@mctr=\count118
\c@cksm@mctr=\count119
\c@msm@mctr=\count120
\c@xmsm@mctr=\count121
\c@cmsm@mctr=\count122
\c@bsm@mctr=\count123
\c@workm@mctr=\count124
\c@figure=\count125
\c@lofdepth=\count126
\c@lofdepth=\count126
\cftbeforefigureskip=\skip156
\cftfigureindent=\skip157
\cftfigurenumwidth=\skip158
\c@table=\count126
\c@lotdepth=\count127
\c@lotdepth=\count127
\cftbeforetableskip=\skip159
\cfttableindent=\skip160
\cfttablenumwidth=\skip161
(/usr/share/texmf-tetex/tex/latex/memoir/mempatch.sty
File: mempatch.sty 2005/02/01 v3.5 Patches for memoir class v1.61
\abs@leftindent=\dimen129
LaTeX Info: Redefining \em on input line 142.
LaTeX Info: Redefining \em on input line 384.
LaTeX Info: Redefining \emph on input line 392.
))

LaTeX Warning: You have requested, on input line 106, version
               `2005/09/25' of document class memoir,
               but only version
               `2004/04/05 v1.61 configurable document class'
               is available.


******************************************************
Stock height and width: 711.3189pt by 500.7685pt
Top and edge trims: 0.0pt and 0.0pt
Page height and width: 711.3189pt by 500.7685pt
Text height and width: 560.51923pt by 361.0pt
Spine and edge margins: 65.44142pt and 73.97733pt
Upper and lower margins: 88.2037pt and 62.59596pt
Headheight and headsep: 14.0pt and 28.45274pt
Footskip: 17.07182pt
Columnsep and columnseprule: 10.0pt and 0.0pt
Marginparsep and marginparwidth: 7.0pt and 50.0pt
******************************************************

LaTeX Info: Redefining \ttfamily on input line 119.
(/usr/share/texmf-tetex/tex/latex/amsmath/amsmath.sty
Package: amsmath 2000/07/18 v2.13 AMS math features
\@mathmargin=\skip162
For additional information on amsmath, use the `?' option.
(/usr/share/texmf-tetex/tex/latex/amsmath/amstext.sty
Package: amstext 2000/06/29 v2.01
(/usr/share/texmf-tetex/tex/latex/amsmath/amsgen.sty
File: amsgen.sty 1999/11/30 v2.0
\@emptytoks=\toks23
\ex@=\dimen130
)) (/usr/share/texmf-tetex/tex/latex/amsmath/amsbsy.sty
Package: amsbsy 1999/11/29 v1.2d
\pmbraise@=\dimen131
) (/usr/share/texmf-tetex/tex/latex/amsmath/amsopn.sty
Package: amsopn 1999/12/14 v2.01 operator names
)
\inf@bad=\count127
LaTeX Info: Redefining \frac on input line 211.
\uproot@=\count128
\leftroot@=\count129
LaTeX Info: Redefining \overline on input line 307.
\classnum@=\count130
\DOTSCASE@=\count131
LaTeX Info: Redefining \ldots on input line 379.
LaTeX Info: Redefining \dots on input line 382.
LaTeX Info: Redefining \cdots on input line 467.
\Mathstrutbox@=\box26
\strutbox@=\box27
\big@size=\dimen132
LaTeX Font Info:    Redeclaring font encoding OML on input line 567.
LaTeX Font Info:    Redeclaring font encoding OMS on input line 568.
\macc@depth=\count132
\c@MaxMatrixCols=\count133
\dotsspace@=\muskip10
\c@parentequation=\count134
\dspbrk@lvl=\count135
\tag@help=\toks24
\row@=\count136
\column@=\count137
\maxfields@=\count138
\andhelp@=\toks25
\eqnshift@=\dimen133
\alignsep@=\dimen134
\tagshift@=\dimen135
\tagwidth@=\dimen136
\totwidth@=\dimen137
\lineht@=\dimen138
\@envbody=\toks26
\multlinegap=\skip163
\multlinetaggap=\skip164
\mathdisplay@stack=\toks27
LaTeX Info: Redefining \[ on input line 2666.
LaTeX Info: Redefining \] on input line 2667.
) (/usr/share/texmf-tetex/tex/latex/amsfonts/amssymb.sty
Package: amssymb 2002/01/22 v2.2d
(/usr/share/texmf-tetex/tex/latex/amsfonts/amsfonts.sty
Package: amsfonts 2001/10/25 v2.2f
LaTeX Font Info:    Try loading font information for OMX+cmex on input line 76.

(/usr/share/texmf-tetex/tex/latex/base/omxcmex.fd
File: omxcmex.fd 1999/05/25 v2.5h Standard LaTeX font definitions
)
\symAMSa=\mathgroup4
\symAMSb=\mathgroup5
LaTeX Font Info:    Overwriting math alphabet `\mathfrak' in version `bold'
(Font)                  U/euf/m/n --> U/euf/b/n on input line 132.
)) (/usr/share/texmf-tetex/tex/latex/amscls/amsthm.sty
Package: amsthm 2004/08/06 v2.20
\thm@style=\toks28
\thm@bodyfont=\toks29
\thm@headfont=\toks30
\thm@notefont=\toks31
\thm@headpunct=\toks32
\thm@preskip=\skip165
\thm@postskip=\skip166
\thm@headsep=\skip167
\dth@everypar=\toks33
)
\@Centering=\skip168
(/usr/share/texmf-tetex/tex/latex/tools/indentfirst.sty
Package: indentfirst 1995/11/23 v1.03 Indent first paragraph (DPC)
) (/usr/share/texmf-tetex/tex/latex/hyperref/hyperref.sty
Package: hyperref 2003/11/30 v6.74m Hypertext links for LaTeX
(/usr/share/texmf-tetex/tex/latex/graphics/keyval.sty
Package: keyval 1999/03/16 v1.13 key=value parser (DPC)
\KV@toks@=\toks34
)
\@linkdim=\dimen139
\Hy@linkcounter=\count139
\Hy@pagecounter=\count140
(/usr/share/texmf-tetex/tex/latex/hyperref/pd1enc.def
File: pd1enc.def 2003/11/30 v6.74m Hyperref: PDFDocEncoding definition (HO)
) (/usr/share/texmf-tetex/tex/latex/hyperref/hyperref.cfg
File: hyperref.cfg 2002/06/06 v1.2 hyperref configuration of TeXLive and teTeX
)
Package hyperref Info: Option `draft' set `true' on input line 1830.
Package hyperref Info: Option `colorlinks' set `true' on input line 1830.
Package hyperref Info: Hyper figures OFF on input line 1880.
Package hyperref Info: Link nesting OFF on input line 1885.
Package hyperref Info: Hyper index ON on input line 1888.
Package hyperref Info: Plain pages ON on input line 1893.
Package hyperref Info: Backreferencing OFF on input line 1900.
Implicit mode ON; LaTeX internals redefined
Package hyperref Info: Bookmarks ON on input line 2004.
(/usr/share/texmf-tetex/tex/latex/url/url.sty
\Urlmuskip=\muskip11
Package: url 2004/03/15  ver 3.1  Verb mode for urls, etc.
)
LaTeX Info: Redefining \url on input line 2143.
\Fld@menulength=\count141
\Field@Width=\dimen140
\Fld@charsize=\dimen141
\Choice@toks=\toks35
\Field@toks=\toks36
Package hyperref Info: Hyper figures OFF on input line 2618.
Package hyperref Info: Link nesting OFF on input line 2623.
Package hyperref Info: Hyper index ON on input line 2626.
Package hyperref Info: backreferencing OFF on input line 2633.
Package hyperref Info: Link coloring ON on input line 2636.
\c@Item=\count142
)
*hyperref using default driver hpdftex*
(/usr/share/texmf-tetex/tex/latex/hyperref/hpdftex.def
File: hpdftex.def 2003/11/30 v6.74m Hyperref driver for pdfTeX
(/usr/share/texmf-tetex/tex/latex/psnfss/pifont.sty
Package: pifont 2004/09/15 PSNFSS-v9.2 Pi font support (SPQR) 
LaTeX Font Info:    Try loading font information for U+pzd on input line 63.
(/usr/share/texmf-tetex/tex/latex/psnfss/upzd.fd
File: upzd.fd 2001/06/04 font definitions for U/pzd.
)
LaTeX Font Info:    Try loading font information for U+psy on input line 64.
(/usr/share/texmf-tetex/tex/latex/psnfss/upsy.fd
File: upsy.fd 2001/06/04 font definitions for U/psy.
))
\Fld@listcount=\count143
\@outlinefile=\write4
) (/usr/share/texmf-tetex/tex/latex/memoir/memhfixc.sty
Package: memhfixc 2004/05/13 v1.6 package fixes for memoir class
)
Package hyperref Info: Option `pdfdisplaydoctitle' set `true' on input line 410
.
Package hyperref Info: Option `bookmarksopen' set `true' on input line 410.
Package hyperref Info: Option `linktocpage' set `false' on input line 410.
Package hyperref Info: Option `plainpages' set `false' on input line 410.
(./25156-t.aux)
\openout1 = `25156-t.aux'.

LaTeX Font Info:    Checking defaults for OML/cmm/m/it on input line 474.
LaTeX Font Info:    ... okay on input line 474.
LaTeX Font Info:    Checking defaults for T1/cmr/m/n on input line 474.
LaTeX Font Info:    ... okay on input line 474.
LaTeX Font Info:    Checking defaults for OT1/cmr/m/n on input line 474.
LaTeX Font Info:    ... okay on input line 474.
LaTeX Font Info:    Checking defaults for OMS/cmsy/m/n on input line 474.
LaTeX Font Info:    ... okay on input line 474.
LaTeX Font Info:    Checking defaults for OMX/cmex/m/n on input line 474.
LaTeX Font Info:    ... okay on input line 474.
LaTeX Font Info:    Checking defaults for U/cmr/m/n on input line 474.
LaTeX Font Info:    ... okay on input line 474.
LaTeX Font Info:    Checking defaults for PD1/pdf/m/n on input line 474.
LaTeX Font Info:    ... okay on input line 474.
\c@lofdepth=\count144
\c@lotdepth=\count145
(/usr/share/texmf-tetex/tex/latex/graphics/color.sty
Package: color 1999/02/16 v1.0i Standard LaTeX Color (DPC)
(/usr/share/texmf-tetex/tex/latex/graphics/color.cfg
File: color.cfg 2005/02/03 v1.3 color configuration of teTeX/TeXLive
)
Package color Info: Driver file: pdftex.def on input line 125.
(/usr/share/texmf-tetex/tex/latex/graphics/pdftex.def
File: pdftex.def 2002/06/19 v0.03k graphics/color for pdftex
\Gread@gobject=\count146
(/usr/share/texmf-tetex/tex/context/base/supp-pdf.tex (/usr/share/texmf-tetex/t
ex/context/base/supp-mis.tex
loading : Context Support Macros / Miscellaneous (2004.10.26)
\protectiondepth=\count147
\scratchcounter=\count148
\scratchtoks=\toks37
\scratchdimen=\dimen142
\scratchskip=\skip169
\scratchmuskip=\muskip12
\scratchbox=\box28
\scratchread=\read2
\scratchwrite=\write5
\zeropoint=\dimen143
\onepoint=\dimen144
\onebasepoint=\dimen145
\minusone=\count149
\thousandpoint=\dimen146
\onerealpoint=\dimen147
\emptytoks=\toks38
\nextbox=\box29
\nextdepth=\dimen148
\everyline=\toks39
\!!counta=\count150
\!!countb=\count151
\recursecounter=\count152
)
loading : Context Support Macros / PDF (2004.03.26)
\nofMPsegments=\count153
\nofMParguments=\count154
\MPscratchCnt=\count155
\MPscratchDim=\dimen149
\MPnumerator=\count156
\everyMPtoPDFconversion=\toks40
)))
Package hyperref Info: Link coloring ON on input line 474.
(/usr/share/texmf-tetex/tex/latex/hyperref/nameref.sty
Package: nameref 2003/12/03 v2.21 Cross-referencing by name of section
\c@section@level=\count157
)
LaTeX Info: Redefining \ref on input line 474.
LaTeX Info: Redefining \pageref on input line 474.
(./25156-t.out) (./25156-t.out)
\openout4 = `25156-t.out'.

Redoing nameref's sectioning
Redoing nameref's label
LaTeX Font Info:    Try loading font information for T1+pcr on input line 512.
(/usr/share/texmf-tetex/tex/latex/psnfss/t1pcr.fd
File: t1pcr.fd 2001/06/04 font definitions for T1/pcr.
) [1

{/var/lib/texmf/fonts/map/pdftex/updmap/pdftex.map}] [2

] [1

]
LaTeX Font Info:    External font `cmex10' loaded for size
(Font)              <9> on input line 543.
LaTeX Font Info:    External font `cmex10' loaded for size
(Font)              <6> on input line 543.
LaTeX Font Info:    External font `cmex10' loaded for size
(Font)              <5> on input line 543.
[2]
LaTeX Font Info:    Try loading font information for OMS+cmr on input line 562.

(/usr/share/texmf-tetex/tex/latex/base/omscmr.fd
File: omscmr.fd 1999/05/25 v2.5h Standard LaTeX font definitions
)
LaTeX Font Info:    Font shape `OMS/cmr/m/n' in size <12> not available
(Font)              Font shape `OMS/cmsy/m/n' tried instead on input line 562.
[3

] [4

]
LaTeX Font Info:    External font `cmex10' loaded for size
(Font)              <12> on input line 584.
LaTeX Font Info:    External font `cmex10' loaded for size
(Font)              <8> on input line 584.
[1

] [2] [3] [4] [5] [6

] [7] [8] [9] [10] [11] [12] [13] [14] [15] [16] [17

] [18] [19] [20] [21] [22] [23] [24] [25] [26] [27] [28] [29

] [30] [31] [32] [33]
Overfull \hbox (3.01158pt too wide) in paragraph at lines 1995--2001
[] \OML/cmm/m/it/12 R[] \OT1/cmr/m/n/12 = (\OML/cmm/m/it/12 x[]; y[]\OT1/cmr/m/
n/12 ) = 2(\OML/cmm/m/it/12 ex[]; y[]\OT1/cmr/m/n/12 ) = 2(\OML/cmm/m/it/12 e; 
x[]y[]\OT1/cmr/m/n/12 ) = 2 [] \OML/cmm/m/it/12 R[]
 []

[34] [35] [36] [37] [38] [39] [40] [41] [42

] [43] [44] [45] [46] [47] [48] [49] [50] [51] [52] [53] [54] [55] [56] [57] [5
8] [59] [60]
bibitemlist
LaTeX Font Info:    External font `cmex10' loaded for size
(Font)              <10.95> on input line 3099.
[61

] [62] [63] [64] [65] [66] [67] [68

] [69] [70] [71] [72] [73] [74] [75] (./25156-t.aux)

 *File List*
  memoir.cls    2004/04/05 v1.61 configurable document class
   mem12.clo    2004/03/12 v0.3 memoir class 12pt size option
mempatch.sty    2005/02/01 v3.5 Patches for memoir class v1.61
 amsmath.sty    2000/07/18 v2.13 AMS math features
 amstext.sty    2000/06/29 v2.01
  amsgen.sty    1999/11/30 v2.0
  amsbsy.sty    1999/11/29 v1.2d
  amsopn.sty    1999/12/14 v2.01 operator names
 amssymb.sty    2002/01/22 v2.2d
amsfonts.sty    2001/10/25 v2.2f
 omxcmex.fd    1999/05/25 v2.5h Standard LaTeX font definitions
  amsthm.sty    2004/08/06 v2.20
indentfirst.sty    1995/11/23 v1.03 Indent first paragraph (DPC)
hyperref.sty    2003/11/30 v6.74m Hypertext links for LaTeX
  keyval.sty    1999/03/16 v1.13 key=value parser (DPC)
  pd1enc.def    2003/11/30 v6.74m Hyperref: PDFDocEncoding definition (HO)
hyperref.cfg    2002/06/06 v1.2 hyperref configuration of TeXLive and teTeX
     url.sty    2004/03/15  ver 3.1  Verb mode for urls, etc.
 hpdftex.def    2003/11/30 v6.74m Hyperref driver for pdfTeX
  pifont.sty    2004/09/15 PSNFSS-v9.2 Pi font support (SPQR) 
    upzd.fd    2001/06/04 font definitions for U/pzd.
    upsy.fd    2001/06/04 font definitions for U/psy.
memhfixc.sty    2004/05/13 v1.6 package fixes for memoir class
   color.sty    1999/02/16 v1.0i Standard LaTeX Color (DPC)
   color.cfg    2005/02/03 v1.3 color configuration of teTeX/TeXLive
  pdftex.def    2002/06/19 v0.03k graphics/color for pdftex
supp-pdf.tex
 nameref.sty    2003/12/03 v2.21 Cross-referencing by name of section
 25156-t.out
 25156-t.out
   t1pcr.fd    2001/06/04 font definitions for T1/pcr.
  omscmr.fd    1999/05/25 v2.5h Standard LaTeX font definitions
 ***********

 ) 
Here is how much of TeX's memory you used:
 6146 strings out of 94500
 76437 string characters out of 1175771
 165784 words of memory out of 1000000
 8849 multiletter control sequences out of 10000+50000
 16938 words of font info for 62 fonts, out of 500000 for 2000
 580 hyphenation exceptions out of 8191
 26i,14n,24p,281b,435s stack positions out of 1500i,500n,5000p,200000b,5000s
PDF statistics:
 1191 PDF objects out of 300000
 423 named destinations out of 131072
 81 words of extra memory for PDF output out of 65536
</usr/share/texmf-tetex/fonts/type1/bluesky/cm/cmmi10.pfb></usr/share/texmf-t
etex/fonts/type1/bluesky/cm/cmsy6.pfb></usr/share/texmf-tetex/fonts/type1/blues
ky/cm/cmmi6.pfb></usr/share/texmf-tetex/fonts/type1/bluesky/cm/cmex10.pfb></usr
/share/texmf-tetex/fonts/type1/bluesky/cm/cmmi8.pfb></usr/share/texmf-tetex/fon
ts/type1/bluesky/cm/cmr8.pfb></usr/share/texmf-tetex/fonts/type1/bluesky/cm/cms
y8.pfb></usr/share/texmf-tetex/fonts/type1/bluesky/cm/cmmi12.pfb></usr/share/te
xmf-tetex/fonts/type1/bluesky/euler/eufm10.pfb></usr/share/texmf-tetex/fonts/ty
pe1/bluesky/cm/cmsy10.pfb></usr/share/texmf-tetex/fonts/type1/bluesky/cm/cmti12
.pfb></usr/share/texmf-tetex/fonts/type1/bluesky/cm/cmr6.pfb></usr/share/texmf-
tetex/fonts/type1/bluesky/cm/cmr9.pfb></usr/share/texmf-tetex/fonts/type1/blues
ky/cm/cmr10.pfb></usr/share/texmf-tetex/fonts/type1/bluesky/cm/cmti10.pfb></usr
/share/texmf-tetex/fonts/type1/bluesky/cm/cmcsc10.pfb></usr/share/texmf-tetex/f
onts/type1/bluesky/cm/cmr12.pfb></usr/share/texmf-tetex/fonts/type1/bluesky/cm/
cmbx12.pfb>{/usr/share/texmf-tetex/fonts/enc/dvips/psnfss/8r.enc}</usr/share/te
xmf-tetex/fonts/type1/urw/courier/ucrr8a.pfb>
Output written on 25156-t.pdf (81 pages, 587689 bytes).
